\section{Introduction}
Agile methodologies adoption has increased a lot \cite{laurie_williams, Wang_Conboy, Salo_Abrahamsson}. Many teams believe they are agile although they are not \cite{ambysoft}, while 13\% of companies are at odds with core agile values \cite{versionOne}

The reasons for this mainly occur during the agility transition. Some of them are:

\begin{itemize}
	\item agile methodologies are easier to misunderstand \cite{6427226}
	\item agile methodologies are often applied to the wrong context \cite{cefam}
	\item organizations modify agile practices before implementing them \cite{1579312, 1629340}
	\item people in the same team have different perception of agile practices \cite{ambler}
\end{itemize}

Based on the above, it is evident there is a huge need for tools measuring agility to verify the level of transition.

\textbf{Agility measurement tools problem} \\
For over a decade, researchers have been constantly coming up with models and frameworks in an effort to provide a solution. Unfortunately, the multiple tools have created a saturation in the field, resulting in them being used only by the organizations that participated in the empirical studies for their creation \cite{samireh_jalali_dissertation, jalali_angelis}.


\clearpage