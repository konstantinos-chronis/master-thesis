%http://www.latextemplates.com/template/koma-script-presentation

%----------------------------------------------------------------------------------------
%	PACKAGES AND OTHER DOCUMENT CONFIGURATIONS
%----------------------------------------------------------------------------------------

\documentclass[
paper=128mm:96mm, % The same paper size as used in the beamer class
fontsize=11pt, % Font size
pagesize, % Write page size to dvi or pdf
parskip=half-, % Paragraphs separated by half a line
]{scrartcl} % KOMA script (article)

\linespread{1.12} % Increase line spacing for readability

%------------------------------------------------

\usepackage{xcolor}	 % Required for custom colors
\usepackage{enumitem}  %for latin numbers
\usepackage{multirow} % for tables
\usepackage{hhline}% used for skipping lines in multirows
\usepackage{bbding} % different symbols in itemize

% Define a few colors for making text stand out within the presentation
\definecolor{mygreen}{RGB}{44,85,17}
\definecolor{myblue}{RGB}{34,31,217}
\definecolor{mybrown}{RGB}{194,164,113}
\definecolor{myred}{RGB}{255,66,56}
% Use these colors within the presentation by enclosing text in the commands below
\newcommand*{\mygreen}[1]{\textcolor{mygreen}{#1}}
\newcommand*{\myblue}[1]{\textcolor{myblue}{#1}}
\newcommand*{\mybrown}[1]{\textcolor{mybrown}{#1}}
\newcommand*{\myred}[1]{\textcolor{myred}{#1}}
%------------------------------------------------

%------------------------------------------------
% Margins
\usepackage[ % Page margins settings
includeheadfoot,
top=3.5mm,
bottom=3.5mm,
left=5.5mm,
right=5.5mm,
headsep=6.5mm,
footskip=8.5mm
]{geometry}
%------------------------------------------------

%------------------------------------------------
% Fonts
\usepackage[T1]{fontenc}	 % For correct hyphenation and T1 encoding
\usepackage{lmodern} % Default font: latin modern font
%\usepackage{fourier} % Alternative font: utopia
%\usepackage{charter} % Alternative font: low-resolution roman font
\renewcommand{\familydefault}{\sfdefault} % Sans serif - this may need to be commented to see the alternative fonts
%------------------------------------------------

%------------------------------------------------
% Various required packages
\usepackage{amsthm} % Required for theorem environments
\usepackage{bm} % Required for bold math symbols (used in the footer of the slides)
\usepackage{graphicx} % Required for including images in figures
\usepackage{tikz} % Required for colored boxes
\usepackage{booktabs} % Required for horizontal rules in tables
\usepackage{multicol} % Required for creating multiple columns in slides
\usepackage{lastpage} % For printing the total number of pages at the bottom of each slide
\usepackage[english]{babel} % Document language - required for customizing section titles
\usepackage{microtype} % Better typography
\usepackage{tocstyle} % Required for customizing the table of contents
%------------------------------------------------

%------------------------------------------------
% Slide layout configuration
\usepackage{scrpage2} % Required for customization of the header and footer
\pagestyle{scrheadings} % Activates the pagestyle from scrpage2 for custom headers and footers
\clearscrheadfoot % Remove the default header and footer
\setkomafont{pageheadfoot}{\normalfont\color{black}\sffamily} % Font settings for the header and footer

% Sets vertical centering of slide contents with increased space between paragraphs/lists
\makeatletter
\renewcommand*{\@textbottom}{\vskip \z@ \@plus 1fil}
\newcommand*{\@texttop}{\vskip \z@ \@plus .5fil}
\addtolength{\parskip}{\z@\@plus .25fil}
\makeatother

% Remove page numbers and the dots leading to them from the outline slide
\makeatletter
\newtocstyle[noonewithdot]{nodotnopagenumber}{\settocfeature{pagenumberbox}{\@gobble}}
\makeatother
\usetocstyle{nodotnopagenumber}

\AtBeginDocument{\renewcaptionname{english}{\contentsname}{\Large Outline}} % Change the name of the table of contents
%------------------------------------------------

%------------------------------------------------
% Header configuration - if you don't want a header remove this block
\ihead{
\hspace{-2mm}
\begin{tikzpicture}[remember picture,overlay]
\node [xshift=\paperwidth/2,yshift=-\headheight] (mybar) at (current page.north west)[rectangle,fill,inner sep=0pt,minimum width=\paperwidth,minimum height=2\headheight,top color=mygreen!64,bottom color=mygreen]{}; % Colored bar
\node[below of=mybar,yshift=3.3mm,rectangle,shade,inner sep=0pt,minimum width=128mm,minimum height =1.5mm,top color=black!50,bottom color=white]{}; % Shadow under the colored bar
shadow
\end{tikzpicture}
\color{white}\runninghead} % Header text defined by the \runninghead command below and colored white for contrast
%------------------------------------------------

%------------------------------------------------
% Footer configuration
\newlength{\footheight}
\setlength{\footheight}{8mm} % Height of the footer
\addtokomafont{pagefoot}{\footnotesize} % Small font size for the footnote

\ifoot{% Left side
\hspace{-2mm}
\begin{tikzpicture}[remember picture,overlay]
\node [xshift=\paperwidth/2,yshift=\footheight] at (current page.south west)[rectangle,fill,inner sep=0pt,minimum width=\paperwidth,minimum height=3pt,top color=mygreen,bottom color=mygreen]{}; % Green bar
\end{tikzpicture}
\myauthor\ \raisebox{0.2mm}{$\bm{\vert}$}\ \myuni % Left side text
}

\ofoot[\pagemark/\pageref{LastPage}\hspace{-2mm}]{\pagemark/\pageref{LastPage}\hspace{-2mm}} % Right side
%------------------------------------------------

%------------------------------------------------
% Section spacing - deeper section titles are given less space due to lesser importance
\usepackage{titlesec} % Required for customizing section spacing
\titlespacing{\section}{0mm}{0mm}{0mm} % Lengths are: left, before, after
\titlespacing{\subsection}{0mm}{0mm}{-1mm} % Lengths are: left, before, after
\titlespacing{\subsubsection}{0mm}{0mm}{-2mm} % Lengths are: left, before, after
\setcounter{secnumdepth}{0} % How deep sections are numbered, set to no numbering by default - change to 1 for numbering sections, 2 for numbering sections and subsections, etc
%------------------------------------------------

%------------------------------------------------
% Theorem style
\newtheoremstyle{mythmstyle} % Defines a new theorem style used in this template
{0.5em} % Space above
{0.5em} % Space below
{} % Body font
{} % Indent amount
{\sffamily\bfseries} % Head font
{} % Punctuation after head
{\newline} % Space after head
{\thmname{#1}\ \thmnote{(#3)}} % Head spec
	
\theoremstyle{mythmstyle} % Change the default style of the theorem to the one defined above
\newtheorem{theorem}{Theorem}[section] % Label for theorems
\newtheorem{remark}[theorem]{Remark} % Label for remarks
\newtheorem{algorithm}[theorem]{Algorithm} % Label for algorithms
\makeatletter % Correct qed adjustment
%------------------------------------------------

%------------------------------------------------
% The code for the box which can be used to highlight an element of a slide (such as a theorem)
\newcommand*{\mybox}[2]{ % The box takes two arguments: width and content
\par\noindent
\begin{tikzpicture}[mynodestyle/.style={rectangle,draw=mygreen,thick,inner sep=2mm,text justified,top color=white,bottom color=white,above}]\node[mynodestyle,at={(0.5*#1+2mm+0.4pt,0)}]{ % Box formatting
\begin{minipage}[t]{#1}
#2
\end{minipage}
};
\end{tikzpicture}
\par\vspace{-1.3em}}
%------------------------------------------------

%----------------------------------------------------------------------------------------
%	PRESENTATION INFORMATION
%----------------------------------------------------------------------------------------

\newcommand*{\mytitle}{Measuring Agility \\ {\footnotesize A Validity Study on Tools Measuring The \\ Agility Level of Software Development Teams}} % Title
\newcommand*{\runninghead}{Measuring Agility} % Running head displayed on almost all slides
\newcommand*{\myauthor}{Konstantinos Chronis} % Presenters name(s)
\newcommand*{\mydate}{\today} % Presentation date
\newcommand*{\myuni}{University of Gothenburg --- Software Engineering} % University or department

%----------------------------------------------------------------------------------------

\begin{document}

%----------------------------------------------------------------------------------------
%	TITLE SLIDE
%----------------------------------------------------------------------------------------

% Title slide - you may have to tweak a few of the numbers if you wish to make changes to the layout
\thispagestyle{empty} % No slide header and footer
\begin{tikzpicture}[remember picture,overlay] % Background box
\node [xshift=\paperwidth/2,yshift=\paperheight/2] at (current page.south west)[rectangle,fill,inner sep=0pt,minimum width=\paperwidth,minimum height=\paperheight/2,top color=mygreen,bottom color=mygreen]{}; % Change the height of the box, its colors and position on the page here
\end{tikzpicture}
% Text within the box
\begin{flushright}
\vspace{0.6cm}
\color{white}\sffamily
{\bfseries\large\mytitle\par} % Title
\vspace{0.5cm}
\normalsize
\myauthor\par % Author name
\mydate\par % Date
\vfill
\end{flushright}

\clearpage

%----------------------------------------------------------------------------------------
%	TABLE OF CONTENTS
%----------------------------------------------------------------------------------------

\thispagestyle{empty} % No slide header and footer

\small\tableofcontents % Change the font size and print the table of contents - it may be useful to shrink the font size further if the presentation is full of sections
% To exclude sections/subsections from the table of contents, put an asterisk after \(sub)section like so: \section*{Section Name}

\clearpage

%----------------------------------------------------------------------------------------
%	PRESENTATION SLIDES
%----------------------------------------------------------------------------------------

\chapter{Introduction}
\label{ch:introduction}

\lettrine[lines=4, loversize=-0.1, lraise=0.1]{A}{gile} and plan-driven methodologies are the two dominant approaches in the software development. Organisations and companies tend to leave the cumbersome area of Waterfall process and to embrace the Agile methodologies in the last years \cite{laurie_williams, Wang_Conboy}. Although it has been almost 20 years since the latter were introduced, the companies are quite reluctant in following them \cite{4599456}. Once they do, they start enjoying the benefits of the agile approach, but are these the only benefits they could leverage?

In order to answer to the previous question, one should first understand what ``\textit{agile}" means. According to the dictionary \cite{cambridge_dictionary}, it means ``\textit{to be able to move quickly and easily}", something which is almost impossible with a plan-driven approach. The term agility was first introduced as agile manufacturing in an industry book \cite{agile_manufacturing}.

In 2001, 17 developers formed the Agile Alliance and created the agile manifesto \cite{beck2001agile}, defining what is considered to be agile in order to avoid confusion: 
\begin{itemize}
	\item {\large \textbf{Individuals and interactions}} over processes and tools
	\item {\large \textbf{Working software}} over comprehensive documentation
	\item {\large \textbf{Customer collaboration}} over contract negotiation
	\item {\large \textbf{Responding to change}} over following a plan
\end{itemize}

Software development teams started adopting the most known agile methodologies, such as eXtreme Programming \cite{Beck:2004:EPE:1076267}, Feature Driven Development (FDD) \cite{Palmer:2001:PGF:600044}, Crystal \cite{Cockburn:2004:CCH:1406822}, Scrum \cite{scrum} and others. Most companies use a tailored methodology by following some of the aforementioned processes and practices which better suit their needs. \citet{williams2004toward} reports that rarely all XP practices are exercised in their pure form, something on which \citet{Reifer} and \citet{aveling} also agree based on the results of their surveys, which showed that it is common for organizations to partially adopt XP. \citet{sidky} mention that the organizations face four issues when transitioning to agile.
\begin{inparaenum} [a\upshape)]
\item their readiness for agility
\item the practices they should adopt
\item the potential difficulties in adopting them
\item the necessary organizational preparations for the adoption of agile practices. 
\end{inparaenum}
The most important issue though that tends to be neglected, is how well these methodologies are adopted?

According to \citet{6427226}, the agile methodologies are easier to misunderstand. Such a case could lead to problems later on in the software development process. The previous statement is also supported by \citet{cefam}, who argue that the agile software development methodologies are often applied to the wrong context. In addition, \citet{1629340} concludes that the organizations modify practices before implementing them, a fact also mentioned by \citet{1579312}. \citet{hossain} argue that improper use of agile practices creates problems. \citet{sahota} states that doing agile and being agile are two different things. For the first one a company should follow practices, while for the other one a company should think in an agile way. \citet{lappoA04} state that organizations following the practices of a methodology does not mean they gain much in terms of agility, while on the other hand, \citet{sidky_dissertation} defines the level of agility of a company as the amount of agile practices used. Considering this statement, a group that uses pair programming and collective code ownership at a very low level is more agile than a group which uses only pair programming but in a more efficient manner.

\citet{comparative_agility} pose the question ``\textit{How agile is agile enough}"? Practitioners think that declaring being agile is equally good as being agile. According to a survey conducted by \citet{ambysoft}, only 65\% of the agile companies that answered met the five agile criteria posed in the survey. In addition, 9\% of agile projects failed due to the lack of cultural transition, while 13\% of companies are at odds with core agile values based on the most recent survey by \citet{versionOne}. \citet{poonacha} mentioned that the different perception of agile practices when they are adopted is very worrying, since even people in the same team understand them differently, according to the result of a survey \cite{ambler}. It is evidently not only from literature but also from its application that agile is a way of thinking and working, it is a whole culture \cite{poonacha}. If we had to use one word we could state it is a way of \textit{being}. \citet{Nietzsche} said ``\textit{better know nothing than half-know many things}". In the same vein, maybe it is better not to transition to agile instead of thinking of being agile. 

Since agile methodologies become more and more popular, there is a great need for development of a tool that can measure the level of agility in the organizations that have adopted them. \citet{sidky} mentions the success stories of companies that have adopted agile methods, but without having a measurement tool that could tell if you are really agile. 

Measuring agility implies measuring the agile culture of a team. Alistair Cockburn \cite{cockburn2002agile, Cockburn-poetry} and Jim Highsmith \cite{Highsmith:2002} highlight the importance of culture. The culture though differentiates not only from team to team, but also from person to person within it, based on the values they follow. The only common basis for the agile values is the agile manifesto\cite{beck2001agile} as stated by \citet{Ingalls}. As a result, the ``agile culture tree" has the same root, but the branches grow independent, away from one another, making difficult to measure agility.
%maybe add the previous sentence in the discussion too.

For over a decade, researchers have been constantly coming up with models and frameworks in an effort to provide a solution. Unfortunately, the multiple tools have created a saturation in the field, resulting in being used only by the organizations that participated in the empirical studies for their creation \cite{samireh_jalali_dissertation, jalali_angelis}. As a result, the vicious circle of creating tools with no actual use holds back not only the software development companies, but the research community as well.

%Starts with M (or research process in the CM project case), describes how DC will be done, and ends with how DA will be done.
This Master's Thesis deals with three tools which claim to measure the agility of software development teams. These tools are Perceptive Agile Measurement \cite{pam}, Team Agility Assessment \cite{Leffingwell}, Objectives Principles Strategies \cite{sventha_dissertation}. The first one has been validated with a large sample of subjects, while the second one is used by companies and the third one covers many agile practices. In Chapter~\ref{ch:research_methodology} the completeness of the tools among them on measuring agility is checked by analysing if they overlap each other and how much. In addition, in Chapter~\ref{ch:research_methodology} the correlation of the agile practices of the tools was checked to see how much they correlate with each other. For the above, a case study was performed in the industry (company A). The teams of company A were asked to reply to the survey of each tool. As it was seen in Chapter~\ref{ch:research_methodology} the tools measure the agile practices in different ways. As a result in Chapter~\ref{ch:enhancing_ops} there is an effort of enhancing \ac{OPS} by combining the three tools.

%Contributions
%Probably rewrite it
This Master's Thesis gives the ability to see a comparative study of three tools used for measuring the agility of software development teams. Not only the weaknesses and strengths of the tools become apparent, but also how their results can actually provide a concrete view of how well the agility of a team can be measured. In addition, to the best of the author's knowledge there has not been another similar comparison which would be insightful and which can serve as a basis for future work. Moreover, the common areas are evident with the analysis of the tools completeness, while the unique ones are distinguished. Furthermore, by having a better view of these tools, an effort was made to fill in any existing gaps in order to create a more complete tool which will be able to better cover the needs of practitioners and researchers.

%Outline of the paper
In order to clarify the structure of the thesis,  Chapter~\ref{ch:related_work} presents the tools that measure the agility of agile methodologies (e.g. eXtreme Programming) and the tools which measure the agility of software development organisations/teams. After that, Chapter~\ref{ch:research_methodology} presents the research questions and research methodology followed for this Master's Thesis. Chapter~\ref{ch:results} presents the results of this case study and Chapter~\ref{ch:enhancing_ops} presents the enhancement of \ac{OPS} in measuring agility in combination with \ac{PAM} and \ac{TAA}. The results of the thesis are discussed in Chapter~\ref{ch:discussion} and the conclusions and future work are presented in Chapter~\ref{ch:conclusions_future_work}.  


%are companies agile?
%percent of companies claiming to be agile --- look again at the reports from VersionOne and Ambysoft
%find causes of agile failure
%silver bullet

%problems of the tools

%------------------------------------------------

\section{Research Purpose}

See if tools claiming to measure agility will yield similar results.

\clearpage

\section{Related Work}

Many tools for measuring agility have been created. We have separated them into two categories. 

\begin{itemize}
	\item How agile the agile methodologies are
	\item Agility level of software development teams
\end{itemize}

\subsection{Agility of Agile Methodologies}

\begin{itemize}
	\item 4-Dimensional Analytical Tool
	\item XP Evaluation Framework
	\item Comprehensive Evaluation Framework for Agile Methodologies (CEFAM)
\end{itemize}

\subsection{Agility Level of Teams}

\begin{itemize}
	\item Team Agility Assessment (TAA)
	\item Perceptive Agile Measurement (PAM)
	\item Objectives Principles Strategies Framework (OPS)
	\item Sidky's Agile Measurement Index (SAMI)
	\item Thoughtworks
	\item Comparative Agility
\end{itemize}

%------------------------------------------------
\subsection{Tools Selected}

\begin{itemize}
	\item Perceptive Agility Measurement (PAM) \cite{pam} ({\footnotesize validated with a large sample})
	\item Team Agility Assessment (TAA) \cite{Leffingwell} ({\footnotesize used in industry})
	\item Objectives Principles Strategies (OPS) \cite{sventha_dissertation} ({\footnotesize covers many agile practices})
\end{itemize}


\clearpage

%------------------------------------------------


\section{Research Questions}

\begin{enumerate}
	\item Will PAM, TAA and OPS yield similar results?
	%by changing into "Roman" you will get capital letters
	\begin{enumerate}[label*=\arabic*.]
  		\item Does convergent validity exist among the tools?
  		\item Will the questions that are exactly the same among the tools yield the same results?
		\item What is the coverage of agile practices among the tools?
	\end{enumerate}	
  	\item Can the tools be combined in a way that will provide a better approach to measuring agility?
\end{enumerate}

\clearpage

%------------------------------------------------


\section{Research Methodology}

%------------------------------------------------

\subsection*{Study in Company $A$}

United States company which activates in the Point Of Sales (POS) area. \\
Development and QA departments (4 mixed teams). \\

\subsection*{Methodology $A$}
The analysis made by Koch \cite{koch2005agile} was used for identifying the methodologies and practices Company $A$ uses.

\subsection*{Data Collection}

Company $A$ employees were asked to fill in the surveys created by PAM, TAA and OPS (Effectiveness part) during a 3-week period. The surveys were on a 1-7 Likert scale.%, except from the \textit{Collocation} practice which was on a Likert scale 1-5
Practices which were not used by Company $A$ were excluded from the surveys.

\clearpage

%------------------------------------------------

\subsection*{Data Preparation}

\subsubsection*{Problem}
All the tools have different amount of questions and cover different amount of practices

\subsubsection*{Solution}
\begin{enumerate}
	\item Mapping of practices between tools --- OPS covers more practices
	\item Mapping of questions under the OPS practices
\end{enumerate}

\clearpage

%------------------------------------------------

\subsection*{Data Analysis}

\begin{table} [H]
\centering
	\begin{tabular}{| c | c | c | c | c |} \hline
	\textbf{Practice} & \textbf{Participants} & \textbf{PAM} & \textbf{TAA} & \textbf{OPS} \\ \hline
	\multirow{3}{*}{Practice1} & Participant1 & Score1 & Score1 & Score1 \\ \hhline{~----}
	& \vdots & \vdots & \vdots  & \vdots \\ \hhline{~----}
	& ParticipantN & ScoreN & ScoreN & ScoreN \\ \hline
	\end{tabular}
	\caption{Collected Data Structure}
	\label{table:data_structure}
\end{table}

\clearpage

%\begin{itemize}
%	\item Correlations ({\footnotesize Convergent Validity Establishment})
%	\item Direct Match Questions Analysis
%	\item Tools' Agile Practices Coverage
%\end{itemize}
%
%\clearpage

\subsubsection*{Correlations}

\begin{itemize}
	\item Use the practices covered by each tool and see if they correlate with the same practices from the other two tools
	\item Use ``Spearman's rank correlation coefficient"
\end{itemize}

\clearpage

\subsubsection*{Same Questions Analysis}

\begin{itemize}
	\item Identify which questions were the same among the tools and form groups
	\item ``Mann-Whitney U test" and ``Kruskal-Wallis one-way analysis of variance" for statistical tests
\end{itemize}

The hypothesis is

\begin{itemize}[label={}]
	\item $H_0$: \textit{There is no difference between the groups of the same questions}
	\item $H_1$: \textit{There is a difference between the groups of the same questions}
\end{itemize}

\clearpage

\subsubsection*{Tools' Agile Practices Coverage}

Check which of the tools cover more agile practices and have more questions for a practice. There are two cases:

\begin{itemize}
	\item either the practices were the most popular according to the two case studies \cite{Williams_Microsoft, laurie_williams}.
	\item or they did not exist/were not popular in the case studies.
\end{itemize}

\clearpage
\chapter{Results}
\label{ch:results}

%Express that no correlation means no convergent validity

\lettrine[lines=4, loversize=-0.1, lraise=0.1]{T}{his} chapter presents the outcomes of the case study conducted in company A. We present the results of the monotrait-heteromethod correlations about establishing convergent validity along with the results of the direct match questions, checking if the respondents gave the same answers.

\section{Correlation Results}
As it was seen in the previous chapter only 8 out of 42 plots were monotonic. In the next pages are presented the monotrait-heteromethod correlations. In Table~\ref{table:correlations_frequency} one can see that half of the correlations are between \ac{PAM} and \ac{OPS}. In addition, from the 8 monotonic plots we can clearly see in Table~\ref{table:hbc_correlations} we have a negative correlation. This is surprising because it should not exist if convergent validity was established.  %This corroborates that \ac{OPS} has full coverage of \ac{PAM} as seen in Table ~\ref{table:questions_coverage}. %maybe add in the Analysis - chapter 4

\begin{table} [H]
 \RawFloats %allows to have captions in all of the tables
 \begin{minipage}{.45\textwidth}
  \caption{Continuous Feedback Correlations}
  \label{table:cf_correlations}
   \begin{tabular}{| c | c | c | c |} \hline
   \multicolumn{4}{|c|}{\textbf{Continuous Feedback}}  \\ \hline
   & \ac{PAM} & \ac{TAA} & \ac{OPS} \\ \hline
   \ac{PAM} & 1.000 & NA & 0.459 \\ \hline
   \ac{TAA} & NA & 1.000 & NA \\ \hline
   \ac{OPS} & 0.459 & NA & 1.000 \\ \hline
  \end{tabular}
 \end{minipage}%
%
 \begin{minipage}{.45\textwidth}
  \centering
  \caption{Client Driven Iterations Correlations}
  \label{table:cdi_correlations}
  \begin{tabular}{| c | c | c | c |} \hline
  \multicolumn{4}{|c|}{\textbf{Client Driven Iterations}}  \\ \hline
   & \ac{PAM} & \ac{TAA} & \ac{OPS} \\ \hline
  \ac{PAM} & 1.000 & NA & 0.161 \\ \hline
  \ac{TAA} & NA & 1.000 & NA \\ \hline
  \ac{OPS} & 0.161 & NA & 1.000 \\ \hline
 \end{tabular}
 \end{minipage}%
 %
\end{table}


\begin{table} [H]
 \RawFloats %allows to have captions in all of the tables
 \begin{minipage}{.45\textwidth}
  \caption{High Bandwidth Communication Correlations}
  \label{table:hbc_correlations}
  \begin{tabular}{| c | c | c | c |} \hline
  \multicolumn{4}{|c|}{\textbf{High Bandwidth Communication}}  \\ \hline
  & \ac{PAM} & \ac{TAA} & \ac{OPS} \\ \hline
  \ac{PAM} & 1.000 & 0.322 & -0.023 \\ \hline
  \ac{TAA} & 0.322 & 1.000 & 0.237 \\ \hline
  \ac{OPS} & -0.023 & 0.237 & 1.000 \\ \hline
 \end{tabular}
 \end{minipage}%
%
 \begin{minipage}{.45\textwidth}
  \centering
  \caption{Refactoring Correlations}
  \label{table:ref_correlations}
  \begin{tabular}{| c | c | c | c |} \hline
  \multicolumn{4}{|c|}{\textbf{Refactoring}}  \\ \hline
   & \ac{PAM} & \ac{TAA} & \ac{OPS} \\ \hline
   \ac{PAM} & 1.000 & 0.097 & -0.050 \\ \hline
   \ac{TAA} & 0.097 & 1.000 & 0.181 \\ \hline
   \ac{OPS} & -0.050 & 0.181 & 1.000 \\ \hline
  \end{tabular}  
 \end{minipage}%
 %
\end{table}

\begin{table} [H]
 \RawFloats %allows to have captions in all of the tables
 \begin{minipage}{.45\textwidth}
  \caption{Continuous Integration Correlations}
  \label{table:ci_correlations}
  \begin{tabular}{| c | c | c | c | } \hline
  \multicolumn{4}{|c|}{\textbf{Continuous Integration}}  \\ \hline
   & \ac{PAM} & \ac{TAA} & \ac{OPS} \\ \hline
  \ac{PAM} & 1.000 & 0.398 & 0.249 \\ \hline
  \ac{TAA} & 0.398 & 1.000 & 0.115 \\ \hline
  \ac{OPS} & 0.249 & 0.115 & 1.000 \\ \hline
 \end{tabular}
 \end{minipage}%
%
 \begin{minipage}{.45\textwidth}
  \centering
   \caption{Iterative and Incremental Development Correlations}
  \label{table:iid_correlations}
  \begin{tabular}{| c | c | c | c |} \hline
  \multicolumn{4}{|c|}{\textbf{Iterative and Incremental Development}}  \\ \hline
  & \ac{PAM} & \ac{TAA} & \ac{OPS} \\ \hline
  \ac{PAM} & 1.000 & 0.204 & 0.396 \\ \hline
  \ac{TAA} & 0.204 & 1.000 & -0.228 \\ \hline
  \ac{OPS} & 0.396 & -0.228 & 1.000 \\ \hline
 \end{tabular}
 \end{minipage}%
 %
\end{table}

\begin{table} [H]
	\caption{Frequency of correlation between tools}
	\label{table:correlations_frequency}
	\begin{tabular}{| c | c |} \hline
		\multicolumn{2}{|c|}{\textbf{Frequency}}  \\ \hline
		\ac{PAM}-\ac{OPS} & 4 \\ \hline
		\ac{PAM}-\ac{TAA} & 3 \\ \hline
		\ac{TAA}-\ac{OPS} & 1 \\ \hline
	\end{tabular}
\end{table}


In Table~\ref{table:descriptive_statistics} one can see the descriptive statistics of the data gathered.

	\begin{longtable}{| p{.13\textwidth} | p{.11\textwidth} | p{.06\textwidth} | p{.06\textwidth} | p{.06\textwidth} | p{.13\textwidth} | p{.11\textwidth} | p{.06\textwidth} | p{.06\textwidth} | p{.06\textwidth} |} \caption{Surveys Descriptive Statistics} \\ \hline 
		\label{table:descriptive_statistics}
		\textbf{Practice} & \textbf{Statistics} & \textbf{\ac{PAM}} & \textbf{\ac{TAA}} & \textbf{\ac{OPS}} &
		\textbf{Practice} & \textbf{Statistics} & \textbf{\ac{PAM}} & \textbf{\ac{TAA}} & \textbf{\ac{OPS}} \\ \hline
		\endhead %repeats the header to the next page
		Adherence to Standards & \begin{tabular}{c} Mean \\ Sd \\ Median \\ Min \\ Max \end{tabular} & 
		\begin{tabular}{c} 1.00 \\ 0.00 \\ 1 \\ 1 \\ 1 \end{tabular} & 
		\begin{tabular}{c} 11.67 \\ 2.17 \\ 12 \\ 7 \\ 14 \end{tabular} & 
		\begin{tabular}{c} 8.10 \\ 2.12 \\ 8 \\ 6 \\ 12 \end{tabular} &	
		Appropriate Distribution of Expertise & \begin{tabular}{c} Mean \\ Sd \\ Median \\ Min \\ Max \end{tabular} &
		\begin{tabular}{c} 1.00 \\ 0.00 \\ 1.0 \\ 1 \\ 1 \end{tabular} & 
		\begin{tabular}{c} 11.13 \\ 2.10 \\ 11.5 \\ 6 \\ 14 \end{tabular} & 
		\begin{tabular}{c} 27.20 \\ 3.51 \\ 27.0 \\ 21 \\ 35 \end{tabular} \\ \hline		
		Client-Driven Iterations & \begin{tabular}{c} Mean \\ Sd \\ Median \\ Min \\ Max \end{tabular} &
		\begin{tabular}{c} 8.63 \\ 3.20 \\ 8.5 \\ 3 \\ 14 \end{tabular} & 
		\begin{tabular}{c} 1.00 \\ 0.00 \\ 1.0 \\ 1 \\ 1 \end{tabular} & 
		\begin{tabular}{c} 13.87 \\ 2.78 \\ 14.0 \\ 9 \\ 21 \end{tabular} &	
		Continuous Feedback & \begin{tabular}{c} Mean \\ Sd \\ Median \\ Min \\ Max \end{tabular} &
		\begin{tabular}{c} 4.87 \\ 1.25 \\ 5.0 \\ 2 \\ 7 \end{tabular} & 
		\begin{tabular}{c} 1.00 \\ 0.00 \\ 1.0 \\ 1 \\ 1 \end{tabular} & 
		\begin{tabular}{c} 9.20 \\ 1.88 \\ 9.5 \\ 5 \\ 14 \end{tabular} \\ \hline		
		Continuous Integration & \begin{tabular}{c} Mean \\ Sd \\ Median \\ Min \\ Max \end{tabular} &
		\begin{tabular}{c} 21.97 \\ 4.40 \\ 21.0 \\ 11 \\ 31 \end{tabular} & 
		\begin{tabular}{c} 24.13 \\ 3.82 \\ 24.5 \\ 16 \\ 31 \end{tabular} & 
		\begin{tabular}{c} 48.10 \\ 4.23 \\ 48.5 \\ 40 \\ 56 \end{tabular} &	
		High-Bandwidth Communication & \begin{tabular}{c} Mean \\ Sd \\ Median \\ Min \\ Max \end{tabular} &
		\begin{tabular}{c} 36.73 \\ 4.11 \\ 38 \\ 29 \\ 42 \end{tabular} & 
		\begin{tabular}{c} 22.87 \\ 3.25 \\ 23 \\ 13 \\ 28 \end{tabular} & 
		\begin{tabular}{c} 60.30 \\ 5.69 \\ 60 \\ 51 \\ 75 \end{tabular} \\ \hline		
		Iteration Progress Tracking and Reporting & \begin{tabular}{c} Mean \\ Sd \\ Median \\ Min \\ Max \end{tabular} &
		\begin{tabular}{c} 21.67 \\ 6.42 \\ 22.5 \\ 8 \\ 35 \end{tabular} & 
		\begin{tabular}{c} 71.73 \\ 15.62 \\ 72.5 \\ 40 \\ 100 \end{tabular} & 
		\begin{tabular}{c} 31.73 \\ 1.55 \\ 32.0 \\ 27 \\ 35 \end{tabular} &	
		Iterative and Incremental Development & \begin{tabular}{c} Mean \\ Sd \\ Median \\ Min \\ Max \end{tabular} &
		\begin{tabular}{c} 27.10 \\ 2.71 \\ 27.0 \\ 22 \\ 34 \end{tabular} &
		\begin{tabular}{c} 8.43 \\ 2.11 \\ 8.5 \\ 4 \\ 13 \end{tabular} &
		\begin{tabular}{c} 14.47 \\ 2.13 \\ 15.0 \\ 11 \\ 18 \end{tabular} \\ \hline		
		Product Backlog & \begin{tabular}{c} Mean \\ Sd \\ Median \\ Min \\ Max \end{tabular} &
		\begin{tabular}{c} 1.00 \\ 0.00 \\ 1.0 \\ 1 \\ 1 \end{tabular} &
		\begin{tabular}{c} 4.97 \\ 0.85 \\ 5.0 \\ 3 \\ 6 \end{tabular} &
		\begin{tabular}{c} 15.80 \\ 2.14 \\ 15.5 \\ 12 \\ 19 \end{tabular}  &
		Refactoring & \begin{tabular}{c} Mean \\ Sd \\ Median \\ Min \\ Max \end{tabular} &
		\begin{tabular}{c} 2.03 \\ 0.85 \\ 2.0 \\ 1 \\ 4 \end{tabular} &
		\begin{tabular}{c} 10.80 \\ 2.27 \\ 11.0 \\ 6 \\ 14 \end{tabular} &
		\begin{tabular}{c} 20.67 \\ 3.66 \\ 20.5 \\ 14 \\ 28 \end{tabular}  \\ \hline
		Self-Organizing Teams & \begin{tabular}{c} Mean \\ Sd \\ Median \\ Min \\ Max \end{tabular} &
		\begin{tabular}{c} 3.6 \\ 1.19 \\ 3.5 \\ 2 \\ 6 \end{tabular} &
		\begin{tabular}{c} 62.9 \\ 6.57 \\ 63.0 \\ 48 \\ 75 \end{tabular} &
		\begin{tabular}{c} 36.5 \\ 5.20 \\ 37.0 \\ 26 \\ 45 \end{tabular}  &
		Smaller and Frequent Product Releases & \begin{tabular}{c} Mean \\ Sd \\ Median \\ Min \\ Max \end{tabular} &
		\begin{tabular}{c} 5.6 \\ 1.19 \\ 6 \\ 2 \\ 7 \end{tabular} &
		\begin{tabular}{c} 5.8 \\ 0.81 \\ 6 \\ 4 \\ 7 \end{tabular} &
		\begin{tabular}{c} 24.8 \\ 1.24 \\ 25 \\ 22 \\ 28 \end{tabular} \\ \hline		
		Software Configuration Management & \begin{tabular}{c} Mean \\ Sd \\ Median \\ Min \\ Max \end{tabular} &
		\begin{tabular}{c} 1 \\ 0 \\ 1 \\ 1 \\ 1 \end{tabular} &
		\begin{tabular}{c} 7 \\ 0 \\ 7 \\ 7 \\ 7 \end{tabular} &
		\begin{tabular}{c} 7 \\ 0 \\ 7 \\ 7 \\ 7 \end{tabular} &
		Test Driven Development & \begin{tabular}{c} Mean \\ Sd \\ Median \\ Min \\ Max \end{tabular} &
		\begin{tabular}{c} 10.90 \\ 2.90 \\ 10.5 \\ 6 \\ 17 \end{tabular} &
		\begin{tabular}{c} 6.57 \\ 3.28 \\ 6.0 \\ 3 \\ 15 \end{tabular} &
		\begin{tabular}{c} 9.10 \\ 1.97 \\ 9.0 \\ 6 \\ 13 \end{tabular} \\ \hline
		Minimal or Just Enough Documentation & \begin{tabular}{c} Mean \\ Sd \\ Median \\ Min \\ Max \end{tabular} &
		\begin{tabular}{c} 1.0 \\ 0.00 \\ 1 \\ 1 \\ 1 \end{tabular} &
		\begin{tabular}{c} 1.0 \\ 0.00 \\ 1 \\ 1 \\ 1 \end{tabular} &
		\begin{tabular}{c} 17.8 \\ 3.16 \\ 18 \\ 10 \\ 23 \end{tabular} &	
		Customer User Acceptance Testing & \begin{tabular}{c} Mean \\ Sd \\ Median \\ Min \\ Max \end{tabular} &
		\begin{tabular}{c} 17.37 \\ 7.04 \\ 17.5 \\ 5 \\ 33 \end{tabular} &
		\begin{tabular}{c} 1.00 \\ 0.00 \\ 1.0 \\ 1 \\ 1 \end{tabular} &
		\begin{tabular}{c} 1.00 \\ 0.00 \\ 1.0 \\ 1 \\ 1 \end{tabular} \\ \hline
		Evolutionary Requirements & \begin{tabular}{c} Mean \\ Sd \\ Median \\ Min \\ Max \end{tabular} &
		\begin{tabular}{c} 1.00 \\ 0.00 \\ 1 \\ 1 \\ 1 \end{tabular} &
		\begin{tabular}{c} 1.00 \\ 0.00 \\ 1 \\ 1 \\ 1 \end{tabular} &
		\begin{tabular}{c} 20.13 \\ 2.21 \\ 20 \\ 17 \\ 25 \end{tabular} &
		Constant Velocity & \begin{tabular}{c} Mean \\ Sd \\ Median \\ Min \\ Max \end{tabular} &
		\begin{tabular}{c} 1.00 \\ 0.00 \\ 1 \\ 1 \\ 1 \end{tabular} &
		\begin{tabular}{c} 5.93 \\ 1.01 \\ 6 \\ 4 \\ 7 \end{tabular} &
		\begin{tabular}{c} 1.00 \\ 0.00 \\ 1 \\ 1 \\ 1 \end{tabular} \\ \hline
\end{longtable}

In \textit{Continuous Feedback} \ac{PAM} and \ac{OPS} have a moderate positive correlation of $\rho$ = 0.459. Both tools focus on getting feedback from the customer, while \ac{OPS} also checks whether the product is developed according to the customer's needs and expectations.

In \textit{Client-Driven Iterations} \ac{PAM} and \ac{OPS} have a low positive correlation of $\rho$ = 0.161. Both tools check for the possibility of the requirements having been prioritized by the customer, while \ac{OPS} additionally focuses on the customers' requests and needs.

In \textit{Continuous Integration} \ac{PAM} and \ac{OPS} have a low positive correlation of $\rho$ = 0.249. The common areas are continuous builds, multiple submits and story acceptance. There is a small difference regarding whether the developers should sync to the latest available code that is supported by the \ac{PAM}.

In \textit{Iterative and Incremental Development} \ac{PAM} and \ac{OPS} have a low positive correlation of $\rho$ = 0.396. The \ac{OPS} focuses on the stories estimation and prioritization, while \ac{PAM} on the deadlines that have to be meet and on the software progress. 

In \textit{High Bandwidth Communication} \ac{PAM} and \ac{TAA} have a low positive correlation of $\rho$ = 0.322. Both of them check for the team collocation, while \ac{TAA} also checks for the communication with the customers. \ac{PAM} and \ac{OPS} surprisingly have a correlation of $\rho$ = -0.023 which means there is no correlation at all. They both focus on the communication, but \ac{OPS} does that to a huge extent, leading to this result. In addition, \ac{OPS} checks for effectively using the time for meetings. \ac{TAA} and \ac{OPS} have a positive correlation of $\rho$ = 0.237. It is worth mentioning that this is the only practice for which correlation can be calculated for all the tools.

In \textit{Refactoring} \ac{PAM} and \ac{TAA} have a correlation of $\rho$ = 0.097, which means there is almost no correlation at all. \ac{TAA} focuses on continuous refactoring, while on the other hand \ac{PAM} focuses on the unit testing of unit testing for refactoring.


\section{Direct Match Questions Results}
\label{sec:direct_match_results}

The groups of direct match questions showed some unexpectedly amazing results. One would expect that questions which are considered to be the same would yield the same results. On the contrary, this did not happen for any of the groups of questions, apart from group \hyperref[G13]{G13}. The heatmaps (see Appendix~\ref{ch:heatmaps}) which were generated by the answers made it crystal clear that the respondents gave different scores. 

\begin{table} [H]
	\begin{tabular}{| p{1cm} | p{2cm} | p{1cm} | p{2cm} | p{1cm} | p{2cm} |} \hline
		Group & Frequency & Group & Frequency & Group & Frequency \\ \hline
		\hyperref[G1]{G1} & 12 & \hyperref[G2]{G2} & 9 & \hyperref[G3]{G3} & 7 \\ \hline
		\hyperref[G4]{G4} & 8 & \hyperref[G5]{G5} & 12 & \hyperref[G6]{G6} & 16 \\ \hline
		\hyperref[G7]{G7} & 13 & \hyperref[G8]{G8} & 12 & \hyperref[G9]{G9} & 10 \\ \hline
		\hyperref[G10]{G10} & 13 & \hyperref[G11]{G11} & 19 & \hyperref[G12]{G12} & 16 \\ \hline
		\hyperref[G13]{G13} & 30 & \hyperref[G14]{G14} & 18 & \hyperref[G15]{G15} & 13 \\ \hline
		\hyperref[G16]{G16} & 12 & \hyperref[G17]{G17} & 6 & & \\ \hline
	\end{tabular}
	\caption{Frequency of Same Answers}
	\label{table:answers_frequency}
\end{table}

Table~\ref{table:answers_frequency} displays the group of questions and the frequency of the same answers given by the respondents. As it can be seen, \hyperref[G13]{G13} is the only group of questions in which all the respondents gave the exact same answer. \hyperref[G13]{G13} is about the existence of software control management and Company A uses version control management software for every single line of code written. On the other hand, \hyperref[G17]{G17} which is about backlog prioritization had the lowest score with only 6 respondents giving the same answer. The maximum difference in answers was up to 2 likert-scale points.

For a better view in the results, one can see the heatmaps in Appendix~\ref{ch:heatmaps}.

\section{Reasons behind results}
\label{subsec:reasons_for_correlations}

The plots showed an unexpected and very interesting result. Not only do not the tools have a correlation, but they do not have a monotonic relationship either between one another for the agile practices covered (see Table~\ref{table:monotonic_relationships}). This could indicate two things. The first one is that the results are random and the second one, is that all three of the tools measure agility differently. 

As far as the aspect of the correlation results being random is concerned, there is a possibility of being true. Maybe another approach on forming the data samples could provide different results, but the approach followed was evaluated as the most suitable at the beginning of the study.

On the other hand, the absence of monotonicity and the negative or extremely low correlations show that the questions used by the tools in order to cover an agile practice, do it differently and that \ac{PAM}, \ac{TAA} and \ac{OPS} measure the agility of software development teams in their own unique way. Each of the tools was constructed and statistically validated during its development by its creators having the agile concept in mind, but apparently following a different path in order to accomplish it. This is quite clear by looking on the different ways that each practice is covered (see Appendix~\ref{ch:mapping}) where many of the questions have a different perspective on measuring a practice although they focus on the same one. 

As it was explained in section~\ref{sec:direct_match_results} almost all groups had different responses for the same questions. This could be due to two reasons. The first one, is that the groups of direct match questions were not correctly formed and and the second one, is that people have the tendency to judge differently a question. As far as the aspect of the groups of direct match questions not being correctly formed in concerned, it is considered to have a low possibility since they were verified by employees whose opinion was asked as it was mentioned in subsection~\ref{subsubsec:direct_match_analysis}. On the other hand, according to \citet{Lacy} survey respondents tend to give different answers to the same questions even weeks apart, something which is a common issue in surveys.

The reasons for these unexpected phenomena are explained in the next paragraphs.

\subsubsection{Few or no questions for measuring a practice}
Another reason for not being able to calculate the correlation of the tools is that they cover slightly or even not at all some of the practices. An example of this is the \textit{Smaller and Frequent Product Releases} practice. \ac{OPS} has four questions for it, while on the other hand, \ac{PAM} and \ac{TAA} have a single one each. Furthermore, \textit{Appropriate Distribution of Expertise} is not covered at all by \ac{PAM} while it is by the rest of the tools. In case the single question gets a low score, this will affect how effectively the tool will measure an agile practice. On the contrary, multiple questions can better cover the practice by examining more factors that affect it. Apart from measuring a practice more precisely, this also has the benefit that even if one question gets a low score, the rest of them are candidates for getting a higher one.

\subsubsection{The same practice is measured differently}
Something very interesting that came up during the data analysis was that although the tools cover the same practices, they do it in different ways, leading to different results. An example of this is the practice of \textit{Refactoring} (check figure ~\ref{fig:ref_plot}). \ac{PAM} checks whether there are enough unit tests and automated system tests to allow the safe code refactoring. In case the course unit/system tests are not developed by a team, the respondents will give low scores to the question, as the team members in company A did. Nevertheless, this does not mean that the team never refactors the software or it does it with bad results. All teams in company A choose to refactor when it adds value to the system, but the level of unit tests is very low and they exist only for specific teams. On the other hand, \ac{TAA} and \ac{OPS} check how often the teams refactor among other factors.

\subsubsection{The same practice is measured in opposite questions}
\label{subsec:opposite_questions}
The \textit{Continuous Integration} practice has a unique paradox among \ac{TAA}, \ac{PAM} and \ac{OPS}. The first two tools have a question about the members of the team having synced to the latest code, while \ac{OPS} checks for the exact opposite. According to \citet{sventha_dissertation}, it is preferable for the teams not to share the same code in order to measure the practice. It is quite doubtful though how correct this question can be, since the \textit{Continuous Integration} requires frequent submits from the developers and thus the rest of the team will also have a local version of the code.

\subsubsection{Questions phrasing}
Although the tools might cover the same areas for each practice, the results could differ because of how a question is structured. An example of this is the \textit{Test Driven Development} practice. Both \ac{TAA} and \ac{PAM} ask about automated code coverage, while \ac{OPS} just asks about the existence of code coverage. Furthermore, \ac{TAA} focuses on 100\% automation while \ac{PAM} doesn’t. Thus, if a team has code coverage but it is not automated, then the score of the respective question should be low. In case of \ac{TAA}, if it is not fully automated, it should be even lower. It is evident that the abstraction level of a question has a great impact. The more specific it is, the more its answer will differ, resulting in possible low scores.
\todo{so a problem with measuring agility is the right abstraction level. So we don't know how, or at what level, agility should be measured. Interesting. This means we have issues even for simple aspects as if they use TDD.}

\subsubsection{Survey answering}
According to \citet{Wagner_Zeglovits} survey responses are affected mainly by two factors. One of them is the comprehension of a question and the other one is the judgement of a question. Although all respondents were free to ask questions for anything they did not understand, there is always the possibility that for their own reasons they preferred not to do it, maybe resulting in misunderstanding of a question's meaning. Moreover, the judgement of a person is extremely subjective which can lead to different approaches in giving an answer. Furthermore, \citet{Floyd_Fowler} argues that respondents can also answer to a question in a way that they will look good to the person reading the answers. 

\subsubsection{Better understanding of agile concepts}
In pre-post studies there is a possibility of the subjects becoming more aware of a problem in the second test due to the first test \cite{Campbell_Stanley}. Although the \textit{testing} threat as it is called does not directly apply here, the similar surveys on consecutive weeks could have enabled the respondents to take a deeper look in the agile concepts, resulting in better understanding of them and consequently providing different answers in the surveys' questions. 

\subsubsection{How people perceive agility --- Maybe add this at the final discussion of the document?}
Although the concept of agility is not new, people do not seem to fully understand it as \citet{Wang_Conboy} also mention. This is actually the reason for having so many tools in the field trying to measure how agile the teams are or the methodologies used. Teams implement agile methodologies differently and researchers create different measurement tools. There are numerous definitions about what agility is \cite{Kidd, Kara, Ramesh, agile_manufacturing}, and each of the tools creator adopt or adapt the tools to match their needs. Their only common basis is the agile manifesto \cite{beck2001agile} and its twelve principles \cite{agile_principles}, which are (and should be considered as) a compass for the agile practitioners. Nevertheless, they are not enough and this resulted in the saturation of the field. Moreover, \citet{conboy_fitzgerald} state that the Agile Manifesto principles do not provide practical understanding of the concept of Agility. Consequently, all the reasons behind the current survey results are driven by the way in which tool creators and tool users perceive agility.

The questions in the surveys were all based on how their creators perceived the agile concept which is quite vague as \citet{tsourveloudis} have pointed out. As the reader has seen in previous chapters, \ac{PAM}, \ac{TAA} and \ac{OPS} focus on some common areas/practices, such as  \textit{Smaller and Frequent Product Releases} and \textit{High-Bandwidth Communication}, while many are different. None of the \citet{sventha_dissertation}, \citet{pam}, \citet{Leffingwell} claimed of course to have created the most complete measurement tool, but still this leads to the oxymoron that the tools created by specialists to measure the agility of software development tools, actually do it differently without providing substantial solution to the problem. On the contrary, this leads to more confusion for the agile practitioners who are at a loose ends.

Considering that the researchers and specialists in the agile field perceive the concept of agility differently, it would be naive to say that the teams do not do the same. The answers in surveys are subjective and people answer them depending on how they understand them. This is also corroborated by the fact that although a team works at the same room and follows the same processes for weeks, it is rather unlikely if its members will have the same understanding of what a retrospection or a releasing planning meeting means for them something which is also stated by \citet{Williams_Microsoft}.

%Validity threats?
%http://fluidsurveys.com/university/tips-for-avoiding-respondent-bias/
%http://fluidsurveys.com/university/tips-for-overcoming-researcher-bias/


%Reasons
%different perception on things -- write a new one small
%mix the reasons for results together and explain which applies where.

%http://en.wikipedia.org/wiki/Social_desirability_bias

\section{Research Question recap}
Regarding RQ \#2, \textit{``In what ways do the tools correlate"}, it was seen that the agile practices correlations almost do not exist at all between \ac{PAM}, \ac{TAA} and \ac{OPS}. Only 8 our 42 relationships have a correlation and they are mostly positive but low.

\section{Chapter Summary}
In this chapter was presented how complete were the tools among them. \ac{OPS} covers both \ac{PAM} and \ac{TAA} to a large extent. In the second part the case study took place which showed that only 8 out of 42 relationships between the agile practices do correlate. The rest lack monotonicity due to
\begin{inparaenum} [a\upshape)]
	\item measuring a practice in the same way,
	\item having few or no questions for measuring a practice,
	\item measuring the same practice differently,
	\item measuring the same practice in opposite questions,
	\item questions phrasing.
\end{inparaenum}

\section{OPS Enhancement}

Combine the questions of the three tools to create one which covers more practices and questions.

\textbf{Excluded questions which concerned}

\begin{itemize}
	\item product ownership (Scrum oriented)
	\item iteration defects (software should be delivered to client)
\end{itemize}

\begin{table} [H]
	\begin{tabular}{| c | c | c |} \hline
		 & \textbf{Indicators Introduced} & \textbf{Questions Introduced} \\ \hline
		 \textbf{Capability} & 3 & 7 \\ \hline
		 \textbf{Effectiveness} & 9 & 46 \\ \hline
	\end{tabular}
	\caption{Summary of Indicators and Questions Added}
\end{table}

\clearpage
\chapter{Discussion}
\label{ch:discussion}
\lettrine[lines=4, loversize=-0.1, lraise=0.1]{T}{his} chapter analyses the reasons behind the results presented in Chapter~\ref{ch:results} and provides answers to the research questions (RQ) tabled in Chapter~\ref{ch:research_methodology}. Finally, it concludes with the presentation of validity threats and how these were mitigated while conducting this case study.

\section{Answers to Research Questions}

\subsection{RQ\#1 - Will \ac{PAM}, \ac{TAA} and \ac{OPS} yield similar results?}

The plots in Appendix~\ref{ch:correlation_plots} showed an unexpected and very interesting result. Not only do the tools lack a correlation, but they do not even have a monotonic relationship when compared to each other for the agile practices covered (see Table~\ref{table:monotonic_relationships}), resulting in absence of convergent validity. This could indicate two things. The first one is that the results are random and the second one is that all three of the tools measure agility differently. 

As far as the aspect of the correlation results being random is concerned, there is a possibility that this is true. Maybe another approach on forming the data samples could provide different results, but the approach followed was evaluated as the most suitable at the beginning of the study.

On the other hand, the absence of monotonicity and the negative or extremely low correlations show that the questions used by the tools in order to cover an agile practice do it differently as well as that \ac{PAM}, \ac{TAA} and \ac{OPS} measure the agility of software development teams in their own unique way. Each of the tools was constructed and statistically validated during the development by its creators who had the agile concept in mind, but apparently, they followed a different path in order to accomplish creating a tool for measuring agility. This is quite clear by looking into the different ways in which each practice is covered (see Appendix~\ref{ch:mapping}), while many of the questions have a different perspective on measuring a practice, although they focus on the same one. 

As it was explained in section~\ref{sec:direct_match_results}, almost all groups had different responses to the same questions. This could be due to two reasons. The first one is that the groups of direct match questions were not correctly formed and the second one is that the people have the tendency to judge a question differently. With regards to the aspect of the groups of direct match questions not being correctly formed, it seems that this has a low probability due to the fact that the questions were verified by the employees at the onset of the survey, as mentioned in section~\ref{subsubsec:direct_match_analysis}. On the other hand, according to \citet{Lacy}, survey respondents tend to give different answers to the same questions even weeks apart, something which is a common issue in surveys.

With regards to the question \textbf{\textit{``Does convergent validity exist among the tools?"}}, we showed that convergent validity could not be established due to the low (if existing) correlations among the tools.

Concerning the question \textbf{\textit{``Will the questions that are exactly the same among the tools yield the same results?"}}, we saw that a considerable amount of respondents' answers were different.

As far as the question \textbf{\textit{``What is the coverage of agile practices among tools?"}} is concerned, we saw that the tools did not cover the agile practices at the same level, neither in terms of questions plethora nor concerning the number of agile practices.

\ac{PAM} was validated with hundreds of subjects, while \ac{TAA} has been used by many companies and \ac{OPS} has been used in case studies as well. Nevertheless, we conclude that \ac{PAM}, \ac{TAA} and \ac{OPS} do not yield similar results, although they should. The reasons for this unexpected phenomenon are explained in the following paragraphs.

\subsubsection{Few or no questions for measuring a practice}
A reason for not being able to calculate the correlation of the tools is that they cover slightly or even not at all some of the practices. An example of this is the \textit{Smaller and Frequent Product Releases} practice. \ac{OPS} includes four questions, while on the other hand, \ac{PAM} and \ac{TAA} have a single question each. Furthermore, \textit{Appropriate Distribution of Expertise} is not covered at all by \ac{PAM}, while it is covered by the rest of the tools. In case the single question gets a low score, this will affect how effectively the tool will measure an agile practice. On the contrary, multiple questions can better cover the practice by examining more factors that affect it. Apart from measuring a practice more precisely, this also has the benefit that even if one question gets a low score, the rest of them are candidates for getting a higher one.

\subsubsection{The same practice is measured differently}
Something very interesting that came up during the data analysis was that although the tools cover the same practices, they do it in different ways, leading to different results. An example of this is the practice of \textit{Refactoring} (check figure ~\ref{fig:ref_plot}). \ac{PAM} checks whether there are enough unit tests and automated system tests to allow the safe code refactoring. In case the course unit/system tests are not developed by a team, the respondents will give low scores to the question, as the team members in Company $A$ did. Nevertheless, this does not mean that the team never refactors the software or that it does it with bad results. All teams in Company $A$ choose to refactor when it adds value to the system, but the level of unit tests is very low and they exist only for specific teams. On the other hand, \ac{TAA} and \ac{OPS} check how often the teams refactor, among other factors.

\subsubsection{The same practice is measured in opposite questions}
\label{subsec:opposite_questions}
The \textit{Continuous Integration} practice has a unique paradox among \ac{TAA}, \ac{PAM} and \ac{OPS}. The first two tools include a question about the members of the team having synced to the latest code, while \ac{OPS} checks for the exact opposite. According to \citet{sventha_dissertation}, it is preferable for the teams not to share the same code in order to measure the practice. It is quite doubtful though how correct this question can be, since the \textit{Continuous Integration} requires frequent submits from the developers and thus the rest of the team will also have a local version of the code.

\subsubsection{Questions phrasing}
Although the tools might cover the same areas for each practice, the results could differ because of how a question is structured. An example of this is the \textit{Test Driven Development} practice. Both \ac{TAA} and \ac{PAM} ask about automated code coverage, while \ac{OPS} just asks about the existence of code coverage. Furthermore, \ac{TAA} focuses on 100\% automation, while \ac{PAM} does not. Thus, if a team has code coverage but it is not automated, then the score of the respective question should be low. In case of \ac{TAA}, if the code coverage is not fully automated, its score should be even lower. It is evident that the abstraction level of a question has a great impact. The more specific it is, the more a reply to it will differ, resulting in possible low scores.
%\todo{so a problem with measuring agility is the right abstraction level. So we don't know how, or at what level, agility should be measured. Interesting. This means we have issues even for simple aspects as if they use TDD.}

\subsubsection{Survey answering}
According to \citet{Wagner_Zeglovits}, survey responses are affected mainly by two factors. One of them is the comprehension of a question and the other one is the judgement of a question. Although all respondents were free to ask about any question they did not understand, there is always the possibility that, for their own reasons, they preferred not to do so, a fact that might result in misunderstanding of a question's meaning. Moreover, the judgement of a person is extremely subjective, which can lead to different approaches in giving an answer. Furthermore, \citet{Floyd_Fowler} argues that respondents can also answer to a question in a way that they will look good to the person reading the answers, a statement in which \citet{feldt_angelis_torkar_samuelsson} also agree.

\subsubsection{Better understanding of agile concepts}
In pre-post studies there is a possibility of the subjects becoming more aware of a problem in the second test due to the first test \cite{Campbell_Stanley}. Although the \textit{testing} threat, as it is called, does not directly apply here, the similar surveys on consecutive weeks could have enabled the respondents to take a deeper look into the agile concepts, resulting in better understanding of them, and consequently, providing different answers to the surveys' questions. 

\subsubsection{How people perceive agility}
Although the concept of agility is not new, people do not seem to fully understand it, as \citet{Wang_Conboy} also mention. This is actually the reason behind the existence of so many tools in the field which are trying to measure how agile the teams are or the methodologies used. The teams implement agile methodologies differently and researchers create different measurement tools. There are numerous definitions of what agility is \cite{Kidd, Kara, Ramesh, agile_manufacturing}, and each of the tool creators adopt or adapt the tools to match their needs. Their only common basis is the agile manifesto \cite{beck2001agile} and its twelve principles \cite{agile_principles}, which are (and should be considered as) a compass for the agile practitioners. Nevertheless, they are not enough and this resulted in the saturation of the field. Moreover, \citet{conboy_fitzgerald} state that the agile manifesto principles do not provide practical understanding of the concept of agility. Consequently, all the reasons behind the current survey results are driven by the way in which tool creators and tool users perceive agility.

The questions in the surveys were all based on how their creators perceived the agile concept which is quite vague, as \citet{tsourveloudis} have pointed out. As the reader has seen in previous chapters, \ac{PAM}, \ac{TAA} and \ac{OPS} focus on some common areas/practices, such as  \textit{Smaller and Frequent Product Releases} and \textit{High-Bandwidth Communication}, while many are different. None of the \citet{sventha_dissertation}, \citet{pam}, \citet{Leffingwell} claimed, of course, to have created the most complete measurement tool, but still, this leads to the oxymoron that the tools created by specialists to measure the agility of software development teams actually do it differently and without providing substantial solution to the problem. On the contrary, this leads to more confusion for the agile practitioners who are at their wits' end.

Considering that the researchers and specialists in the agile field perceive the concept of agility differently, it would be naive to say that the teams do not do the same. The answers to surveys are subjective and people reply to them depending on how they understand them. \citet{ambler} had commented the following ``I suspect that developers and management have different criteria for what it means to be agile", which shows that people do not see eye to eye. This is also corroborated by the fact that, although a team works at the same room and follows the same processes for weeks, it is rather unlikely that its members will have the same understanding of what a retrospection or a releasing planning meeting means to them, a statement which is also supported by \citet{Williams_Microsoft}. This fact is not only mentioned by \citet{Dave_Thomas}, but also supported by the statement that ``no two teams should be doing agile the same way".

\subsection{RQ\#2 - Can the tools be combined in a way that will provide a better approach in measuring agility?}

We have created an enhanced version of \ac{OPS} with the questions which existed only in \ac{PAM} and \ac{TAA}. Although the new tool has not been used and validated, we believe that the results should only be better considering the larger agile area covered by the additional questions. On the other hand though, more questions imply more effort and time from the respondents to spend, which could lead to the opposites results than the desired ones. In Appendix~\ref{ch:ops_pam_taa}, one can see the enhanced version of \ac{OPS}.

\section{Threats to Validity}

\subsection{Construct Validity}
Construct validity mainly deals with obtaining the right method for the concept under study \cite{wohlin2012expse}. We consider that the construct validity concerning the surveys given to the subjects was already handled by the creators of the tools which were used in this Master's Thesis. Our own construct validity lies in establishing the convergent validity in the chapters throughout this document. The small sample of subjects was the biggest threat in establishing convergent validity, making the results very specific to Company $A$ itself. A future work on this topic should be performed in another company to mitigate this threat. In order to avoid mono-method bias, some employees were asked to fill in the surveys first so as to detect any possible issues with them. All the subjects were promised to remain anonymous, resulting in mitigating the evaluation apprehension \cite{wohlin2012expse}. 

\subsection{Internal Validity}
Internal validity deals with the issues that may affect the casual relationship between treatment and results \cite{wohlin2012expse}. The creators of \ac{PAM}, \ac{TAA} and \ac{OPS} have already tried to mitigate this when creating their tools. Yet, there are still some aspects of internal validity, such as selection bias maturation and testing effect. With regard to maturation, this concerns the fatigue and boredom of the respondents. Although the surveys were small in size and did not require more than 15-20 minutes each, still the similar and possibly repetitive questions on the topic could cause fatigue and boredom to the subjects. This could result in the participants giving random answers to the survey questions. The mitigation for this threat was to separate the surveys and conduct them during three different periods. In addition, the respondents could stop the survey at any point and continue whenever they wanted. As far as the testing effect is concerned, this threat could not be mitigated. The testing effect threat applies to pre-post design studies only, but due to the same topic of the surveys, the subjects were to some extent more aware of what questions to expect in the second and third survey. Finally, selection could not be mitigated as well, since the case study focused on a specific company only.

\subsection{Conclusion Validity}
Conclusion validity concerns the possibility of reaching a wrong conclusion \cite{wohlin2012expse}. Although the questions of the surveys have been carefully phrased by their creators, still there may be uncertainty about them. In order to mitigate this, for each survey a pilot one was conducted to spot any questions which would be difficult to understand. In addition, the participants could ask the author of this Master's Thesis about any issue they had concerning the survey questions. Finally, the statistical tests were run only for the data that satisfied the prerequisites, with the aim to mitigate the possibility of incorrect results. %Violated assumptions of the test statistics

\subsection{External Validity}
External validity deals with the ability to generalize the outcomes of the case study \cite{wohlin2012expse}. This Master's Thesis was conducted in collaboration with one company and 30 subjects only. Consequently, it is hard to generalize the outcomes. Nevertheless, we believe that any researcher replicating the case study in another organization with teams which follow the same agile practices as those used in Company $A$ and identified in Table~\ref{table:methodologyA_practices}, should have similar results.

\subsection{Reliability}
Reliability validity concerns the dependence of the data and the analysis on the specific researchers \cite{wohlin2012expse}. To enable other researchers to conduct a similar study, the steps followed have been described and the reasons for the decisions made have been explained. Furthermore, all the data exist in digital format which can be provided to anyone who wants to review them. The presentation of the findings could be probably threatened by the author's experience. In order to mitigate this, the findings were discussed with a Company $A$ employee who did not participate in the case study.

%Maybe mention concurrent validity threat for the enhanced tool?
%http://dissertation.laerd.com/criterion-validity-concurrent-and-predictive-validity-p2.php

%Cronbach's alpha
%https://statistics.laerd.com/spss-tutorials/cronbachs-alpha-using-spss-statistics.php
%http://www.statisticshell.com/docs/reliability.pdf
%https://statistics.laerd.com/minitab-tutorials/cronbachs-alpha-using-minitab.php


%http://www.indiana.edu/~educy520/sec5982/week_9/520in_ex_validity.pdf
%http://en.wikipedia.org/wiki/Validity_%28statistics%29

%===
%Criterion?
%Concurrent? - http://en.wikipedia.org/wiki/Concurrent_validity

\chapter{Conclusions and Future Work}
\label{ch:conclusions_future_work}

\section{Conclusions}

This Master's Thesis contributed to the area of measuring the agility of software developments teams. This contribution can be useful for the research community, but mostly for the practitioners. We provided evidence that tools claiming to measure agility do not yield the same results. The expertise of the tool creators is unquestionable, but nevertheless, their perception of agility and their personal experience has led them to creating a tool in the way they consider more appropriate. A measurement tool which satisfies the needs of one team may not be suitable for other teams. This derives not only from the team's needs but also from the way it transitioned to agile. We should not forget that a team can be considered agile up to a point, under the condition that it uses even only one agile practice. There is still work to be done in order to find a universal tool for measuring agility. Finally, we believe that it does not really matter how much agile a team is, as long as the company is viable and the customers and employees are satisfied.

\section{Future Work}

It would be very interesting to see the results of a study that would take place in more companies, in order to compare them to the results of the present study. In addition, another way of forming the data samples could indicate different results, which is worth seeing. Moreover, future work in the field could check for establishing convergent validity among other agility measurement tools which were presented in Chapter~\ref{ch:related_work}. Finally, the enhancements made in \ac{OPS} (Chapter~\ref{ch:enhancing_ops}) should be validated to check whether they provide a better approach to measuring agility or not.


%------------------------------------------------

\thispagestyle{empty} % No slide header and footer

\begin{tikzpicture}[remember picture,overlay] % Background box
\node [xshift=\paperwidth/2,yshift=\paperheight/2] at (current page.south west)[rectangle,fill,inner sep=0pt,minimum width=\paperwidth,minimum height=\paperheight/3,top color=mygreen,bottom color=mygreen]{}; % Change the height of the box, its colors and position on the page here
\end{tikzpicture}
% Text within the box
\begin{flushright}
\vspace{0.6cm}
\color{white}\sffamily
{\bfseries\LARGE Questions?\par} % Request for questions text
\vfill
\end{flushright}

%------------------------------------------------

\thispagestyle{empty} % No slide header and footer

\bibliographystyle{unsrt}
\bibliography{../references}

\clearpage


%----------------------------------------------------------------------------------------

\end{document}