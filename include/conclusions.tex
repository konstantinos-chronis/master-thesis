\chapter{Conclusions and Future Work}
\label{ch:conclusions_future_work}

\section{Conclusions}

This Master's Thesis contributed to the area of measuring the agility of software developments teams. This contribution can be useful for the research community, but mostly for the practitioners. We provided evidence that tools claiming to measure agility do not yield the same results. The expertise of the tool creators is unquestionable, but nevertheless, their perception of agility and their personal experience has led them to creating a tool in the way they consider more appropriate. A measurement tool which satisfies the needs of one team may not be suitable for other teams. This derives not only from the team's needs but also from the way it transitioned to agile. We should not forget that a team can be considered agile up to a point, under the condition that it uses even only one agile practice. There is still work to be done in order to find a universal tool for measuring agility. Finally, we believe that it does not really matter how much agile a team is, as long as the company is viable and the customers and employees are satisfied.

\section{Future Work}

It would be very interesting to see the results of a study that would take place in more companies, in order to compare them to the results of the present study. In addition, another way of forming the data samples could indicate different results, which is worth seeing. Moreover, future work in the field could check for establishing convergent validity among other agility measurement tools which were presented in Chapter~\ref{ch:related_work}. Finally, the enhancements made in \ac{OPS} (Chapter~\ref{ch:enhancing_ops}) should be validated to check whether they provide a better approach to measuring agility or not.