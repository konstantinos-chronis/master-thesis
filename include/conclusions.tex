\chapter{Conclusions and Future Work}
\label{ch:conclusions_future_work}

\section{Conclusions}

This Master's Thesis contributed in the area of measuring the agility of software developments teams. This contribution come hand for the research community but mostly for the practitioners. We provided evidence that tools claiming to measure agility don't do it in the same way. The expertise of the tool creators is unquestionable but nevertheless their perception of agility and their personal experience has led them in creating a tool in the way they consider more appropriate. One measurement tool which fits for a team does not mean it suits to another one. This derives not only from the team's needs but also from the way it transitioned to agile. Finally, we strongly believe that it doesn't really matter how much agile you are, as long as the company is viable and the customers and employees are satisfied.

\section{Future Work}

It would be very interesting to see the results of a study that would take place in more companies in order to compare the results of this one. In addition, another way of forming the data samples could indicate different results which is worth seeing. Moreover, future work on the field could check for establishing convergent validity among other agility measurement tools which were presented in Chapter~\ref{ch:related_work}. Finally, the enhancements made in \ac{OPS} (Chapter~\ref{ch:enhancing_ops})  should be validated to check whether it measures agility better or not.