\begin{appendices}

\chapter{The Capability and Effectiveness Hierarchy for Company F}

\section{Capability Hierarchy} \label{sec:capability_hierarchy}
%\renewcommand{\labelitemi}{$\blacksquare$}
%\renewcommand{\labelitemii}{$\textbullet$}
%\renewcommand{\labelitemiii}{$\circ$}
\begin{itemize}
	\item Refactoring
		\begin{itemize}
			\item Support for Refactoring 
				\begin{itemize}
					\item Is refactoring an expected activity?
					\item Is it feasible to implement code refactoring?
					\item Is it feasible to implement architecture refactoring?
				\end{itemize}
			\item Buy-in for Refactoring
				\begin{itemize}
					\item Are the teams receptive to implementing refactoring?
					\item Is the management receptive to supporting refactoring efforts?
				\end{itemize}
			\item Minimizing Technical Debt
				\begin{itemize}
					\item Is it expected that a well-defined process be adopted to minimize technical debt?
					\item Is it expected that a well-defined process be adopted to manage technical debt?
					\item Is minimizing technical debt a high priority activity?
				\end{itemize}
		\end{itemize}	
%---------------------------------------------------------------------------			
	\item Distribution of expertise
		\begin{itemize}
			\item Appropriate team composition
				\begin{itemize}
					\item Is a scheme for appropriate team composition defined?
					\item Are the requisite skillsets for particular projects identified upfront?
					\item Is it expected that the right people be chosen to accomplish the tasks?
				\end{itemize}
		\end{itemize}
%---------------------------------------------------------------------------		
	\item Configuration Management
		\begin{itemize}
			\item Tool Support for Configuration Management
				\begin{itemize}
					\item Do tools for version control and management exist?
				\end{itemize}
			\item Support for Configuration Management
				\begin{itemize}
					\item Is it expected that the code be kept up to date?
					\item Is it expected that the tests be kept up to date?
					\item Is it expected that the builds be kept up to date?
					\item Is it expected that the release infrastructure be kept up to date?
					\item Is it expected that the documentation be kept up to date?
				\end{itemize}
		\end{itemize}
%---------------------------------------------------------------------------
	\item Adherence to standards
		\begin{itemize}
			\item Identifying features
				\begin{itemize}
					\item Is it expected that well-defined techniques be used to identify the features?
					\end{itemize}
			\item Estimation
				\begin{itemize}
					\item Is it expected that a well-defined approach to estimating the amount of work to be done during each release cycle and iteration be used?
				\end{itemize}
			\item Requirements Prioritization
				\begin{itemize}
					\item Is it expected that a well-defined approach to prioritizing bugs/enhancements, and tasks be used?
				\end{itemize}	
			\item Feature Decomposition
				\begin{itemize}
					\item Is it expected that a mechanism for decomposing the selected features to be developed during the current release cycle into bugs/enhancements be defined?
				\end{itemize}
			\item Coding standards
				\begin{itemize}
					\item Is it expected that each team creates and adopts a set of coding standards?
					\item Is it expected that practices such as pair-programming, collective code ownership be adopted or automated tools be used to ensure adherence to the set standards?
				\end{itemize}
		\end{itemize}
%---------------------------------------------------------------------------
	\item Continuous Integration
		\begin{itemize}
			\item Tool Support for Continuous Integration
				\begin{itemize}
					\item Do automated test suites exist?
					\item Does the requisite test environment exist?
					\item Do appropriate configuration management systems exist?
				\end{itemize}
			\item Process Support for Continuous Integration
				\begin{itemize}
					\item Is continuous integration an expected activity?
					\item Are the team members expected to integrate their code every few hours?
					\item Is it expected that the builds, tests, and other release infrastructure be kept up to date?
					\item Is it expected that automated test suites be developed?
					\item Is it expected that the build process be automated?
				\end{itemize}
			\item Buy-in for Continuous Integration
				\begin{itemize}
					\item Are the teams receptive to implementing continuous integration?
				\end{itemize}
			\item Story Completeness
				\begin{itemize}
					\item Is it expected that the criteria for Done/Done be specified upfront?
				\end{itemize}
		\end{itemize}
%---------------------------------------------------------------------------
	\item Self-managing teams
		\begin{itemize}
			\item Team Empowerment
				\begin{itemize}
					\item Are the team members expected to be involved in determining, planning, and managing their day-to-day activities?
				\end{itemize}
			\item Ownership
				\begin{itemize}
					\item Are the team members expected to demonstrate individual or collective code ownership? 
				\end{itemize}
			\item Performance Expectations
				\begin{itemize}
					\item Is there a set of performance expectations that are agreed upon by the team and the management?
				\end{itemize}
		\end{itemize}
%---------------------------------------------------------------------------
	\item High-bandwidth communication
		\begin{itemize}
			\item On-site Customer
				\begin{itemize}
					\item Are the customers available onsite to answer questions and provide continuous feedback? 
					\item In the absence of an onsite customer, do the customers provide feedback via other means? 
				\end{itemize}	
			\item Scheduling
				\begin{itemize}
					\item Is it expected that time be allocated for Release Planning?
					\item Is it expected that time be allocated for Iteration Planning?
					\item Is it expected that time be allocated for Retrospection? 
					\item Is it expected that time be allocated for Daily Progress Tracking meetings?
				\end{itemize}
			\item Inter- and intra-team communication
				\begin{itemize}
					\item Is it expected that team members communicate and collaborate with their colleagues?
					\item Do the teams have access to requisite tools to support inter- and intra-team communication?
				\end{itemize}
			\item Physical environment
				\begin{itemize}
					\item Is the physical environment conducive to supporting high bandwidth communication?
				\end{itemize}
		\end{itemize}
%---------------------------------------------------------------------------
	\item Client-driven Iterations
		\begin{itemize}
			\item Identifying and prioritizing features
				\begin{itemize}
					\item Are the customers expected to be involved in identifying the features?
					\item Are the customers expected to establish the priorities of the features?
				\end{itemize}
		\end{itemize}
%---------------------------------------------------------------------------
	\item Short delivery cycles
		\begin{itemize}
			\item Development time-frames
				\begin{itemize}
					\item Is it expected that the product be developed over short delivery cycles? For example, a product increment should be released every 6 - 12 months and iterations last for four weeks or less.
				\end{itemize}
		\end{itemize}
%---------------------------------------------------------------------------
	\item Iterative Progression
		\begin{itemize}
			\item Planning
				\begin{itemize}
					\item Is the team expected to plan for each iteration?
				\end{itemize}
			\item Estimation Authority
				\begin{itemize}
					\item Are the developers expected to estimate the time required to complete each bug/enhancement?
				\end{itemize}
			\item Estimation
				\begin{itemize}
					\item Is it expected that a well-defined approach to estimating the amount of work to be done during each release cycle and iteration be used?
				\end{itemize}
		\end{itemize}
%---------------------------------------------------------------------------
	\item Incremental Development
		\begin{itemize}
			\item Estimation Authority
				\begin{itemize}
					\item Are the developers expected to estimate the time required to complete each bug/enhancement?
				\end{itemize}
			\item Requirements Management
				\begin{itemize}
					\item Are tools available for managing the bugs/enhancements?
				\end{itemize}
			\item Identifying and prioritizing features
				\begin{itemize}
					\item Are the customers expected to be involved in identifying the features?
					\item Are the customers expected to establish the priorities of the features?
				\end{itemize}
		\end{itemize}
%---------------------------------------------------------------------------
	\item Evolutionary Requirements
		\begin{itemize}
			\item Minimal Big Requirements Up Front and Big Design Up Front
				\begin{itemize}
					\item Is it expected that only high level features be identified upfront?
					\item Is it expected that an evolutionary approach to architecting the system be followed as opposed to creating the architecture upfront?
				\end{itemize}
			\item Just-In-Time Refinement
				\begin{itemize}
					\item Is it expected that the requirements be determined and refined just-in-time?
				\end{itemize}
			\item Feature Decomposition
				\begin{itemize}
					\item Is it expected that a mechanism for decomposing the selected features to be developed during the current release cycle into stories be defined?
				\end{itemize}
		\end{itemize}
%---------------------------------------------------------------------------
	\item Minimal Documentation
		\begin{itemize}
			\item Tool Support for Minimal Documentation
				\begin{itemize}
					\item Do tools for maintaining documentation exist?
				\end{itemize}
			\item Process support for Minimal Documentation
				\begin{itemize}
					\item Is it expected that minimal documentation be maintained?
				\end{itemize}
			\item Buy-in for Minimal Documentation
				\begin{itemize}
					\item Are the teams receptive to maintaining minimal or just-enough documentation?
				\end{itemize}
		\end{itemize}
\end{itemize}






\section{Effectiveness Hierarchy} \label{sec:effectiveness_hierarchy}

\begin{itemize}
	\item Refactoring 
		\begin{itemize}
			\item Minimizing Technical Debt 
				\begin{itemize}
					\item To what extent do the teams manage technical debt? 
					\item To what extent do the teams minimize technical debt when developing new systems? 
					\item To what extent does the system and the development environment allow Technical Debt to be minimized? 
				\end{itemize}
			\item Buy-in for Refactoring 
				\begin{itemize}
					\item To what extent does the management support the implementation of refactoring? 
					\item To what extent do the teams implement refactoring? 
				\end{itemize}
		\end{itemize}
%---------------------------------------------------------------------------
	\item Distribution of expertise 
		\begin{itemize}
			\item Process Outcomes for Distribution of Expertise
				\begin{itemize}
					\item To what extent do the team members have the requisite expertise to complete the tasks assigned to them? 
					\item To what extent is the work assigned to the team members commensurate with their expertise? 
					\item To what extent does the team effectively complete the work that they have committed to? 
					\item To what extent do the teams have members in leadership positions that can guide the others? 
					\item To what extent do the teams not rely on knowledge external to their teams? 
				\end{itemize}
		\end{itemize}
%---------------------------------------------------------------------------
	\item Configuration Management
		\begin{itemize}
			\item Project Environment for Configuration Management
				\begin{itemize}
					\item To what extent do teams use appropriate tools for version control and management?
				\end{itemize}
		\end{itemize}
%---------------------------------------------------------------------------		
	\item Adherence to standards
		\begin{itemize}
			\item Estimation
				\begin{itemize}
					\item To what extent are the estimates for the amount of work to be done during each iteration accurate?
				\end{itemize}
			\item Coding Standards
				\begin{itemize}
					\item To what extent do the team members agree with the set coding standards? 
					\item To what extent do the team members adhere to the set coding standards?
				\end{itemize}
		\end{itemize}
%---------------------------------------------------------------------------
	\item Continuous Integration
		\begin{itemize}
			\item Project Environment for Continuous Integration 
				\begin{itemize}
					\item To what extent are automated test suites developed?
					\item To what extent are the code bases not shared?
				\end{itemize}
			\item Bug/Enhancement Completeness
				\begin{itemize}
					\item To what extent has each bug/enhancement been coded? 
					\item To what extent has each bug/enhancement been unit tested? 
					\item To what extent has each bug/enhancement been refactored? 
					\item To what extent has each bug/enhancement been checked into the code base? 
					\item To what extent has each bug/enhancement been integrated with the existing code base? 
					\item To what extent has each bug/enhancement been reviewed? 
					\item To what extent has each bug/enhancement been accepted by the customer? 
				\end{itemize}
			\item Daily/Frequent builds
				\begin{itemize}
					\item To what extent do automated builds run one or more times everyday?
				\end{itemize}
		\end{itemize}

%---------------------------------------------------------------------------		
	\item Self-managing teams
		\begin{itemize}
			\item Team Empowerment
				\begin{itemize}
					\item To what extent do the team members determine the amount of work to be done? 
					\item To what extent do the team members take ownership of work items? 
					\item To what extent do the team members hold each other accountable for the work to be completed? 
					\item To what extent do the team members ensure that they complete the work that they are accountable for?
				\end{itemize}
			\item Autonomy
				\begin{itemize}
					\item To what extent do the team members determine, plan, and manage their day-to-day activities under reduced or no supervision from the management? 
					\item To what extent do the developers form ad-hoc groups to determine and refine requirements just-in-time? 
				\end{itemize}
			\item Management support
				\begin{itemize}
					\item To what extent does the management support the self-managing nature of the teams?
				\end{itemize}
		\end{itemize}
%---------------------------------------------------------------------------		
	\item High-bandwidth communication
		\begin{itemize}
			\item Customer Satisfaction
				\begin{itemize}
					\item To what extent is the product developed so far in-sync with the customers' needs and expectations?
				\end{itemize}
			\item Scheduling
				\begin{itemize}
					\item To what extent is the time allocated for the release planning meetings utilized effectively? 
					\item To what extent is the time allocated for the iteration planning meetings utilized effectively?
					\item To what extent is the time allocated for the retrospective meetings utilized effectively? 
					\item To what extent is the time allocated for the daily progress tracking meetings utilized effectively? 
					\item To what extent do the scheduled meetings (except the daily progress tracking meetings) begin and end on time? 
					\item To what extent do the meetings (except the daily progress tracking meetings) take place as scheduled? 
				\end{itemize}
			\item Inter- and intra-team communication
				\begin{itemize}
					\item To what extent does open communication prevail between the business and the development team? 
					\item To what extent does open communication prevail between the manager and the developers and testers? 
					\item To what extent does open communication prevail between the developers and the testers? 
					\item To what extent does open communication prevail among the developers? 
					\item To what extent does open communication prevail between the external customer/user and the business? 
					\item To what extent does open communication prevail between the external customer/user and the development team? 
					\item To what extent does open communication prevail between members of different teams?
				\end{itemize}
		\end{itemize}
%---------------------------------------------------------------------------		
	\item Client-driven Iterations
		\begin{itemize}
			\item Requirements Prioritization
				\begin{itemize}
					\item To what extent do the customers establish the priorities of the bug/enhancement?
				\end{itemize}
			\item Customer Satisfaction
				\begin{itemize}
					\item To what extent is the product developed so far in-sync with the customers' needs and expectations?
				\end{itemize}
			\item Customer Requests
				\begin{itemize}
					\item To what extent are the changes requested by the customers accommodated?
				\end{itemize}			
		\end{itemize}
%---------------------------------------------------------------------------	
	\item Short delivery cycles
		\begin{itemize}
			\item Development time-frames
				\begin{itemize}
					\item To what extent is software released frequently? (length of a release cycle is one year or less)
					\item To what extent is software released frequently? (length of an iteration is four weeks or less)
				\end{itemize}
			\item Customer Satisfaction
				\begin{itemize}
					\item To what extent is the product developed so far in-sync with the customers' needs and expectations?
				\end{itemize}
			\item Roll-backs
				\begin{itemize}
					\item To what extent are the deployments not rolled back?
				\end{itemize}
		\end{itemize}
%---------------------------------------------------------------------------		
	\item Iterative Progression
		\begin{itemize}
			\item Estimation
				\begin{itemize}
					\item To what extent are the estimates for the amount of work to be done during each iteration accurate?
				\end{itemize}
			\item Iteration length
				\begin{itemize}
					\item To what extent are the iterations timeboxed?
					\item To what extent is the length of an iteration 4 weeks or less?
				\end{itemize}
			\item Requirements Management for Iterations
				\begin{itemize}
					\item To what extent is an iteration list maintained?
					\item To what extent are the bugs/enhancements fully estimated when added to the list?
					\item To what extent are the bug/enhancement prioritized when added to the list?
				\end{itemize}
		\end{itemize}
%---------------------------------------------------------------------------		
	\item Incremental Development
		\begin{itemize}
			\item Requirements Management for Releases
				\begin{itemize}
					\item To what extent is a product backlog maintained?
					\item To what extent are the features priorotized when they are added to the backlog?
					\item To what extent are the features fully estimated before they are added to the backlog?
				\end{itemize}
			\item Timeboxing Releases
				\begin{itemize}
					\item To what extent are the release cycles timeboxed?
					\item To what extent are only a subset of the identified features developed during a release cycle?
				\end{itemize}
		\end{itemize}
%---------------------------------------------------------------------------
	\item Evolutionary Requirements
		\begin{itemize}
			\item Requirements Reprioritization 
				\begin{itemize}
					\item To what extent are the features reprioritized as and when new features are identified?
				\end{itemize}
			\item Customer Requests
				\begin{itemize}
					\item To what extent are the changes requested by the customers accommodated?
				\end{itemize}
			\item Minimal Big Requirements Up Front and Big Design Up Front
				\begin{itemize}
					\item To what extent are only the high level features identified upfront?
					\item To what extent are the architecture requirements allowed to evolve over time?
				\end{itemize}
		\end{itemize}
%---------------------------------------------------------------------------
	\item Minimal Documentation
		\begin{itemize}
			\item Maintaining documentation 
				\begin{itemize}
					\item To what extent is minimal documentation supported by teams?
					\item To what extent is minimal documentation created/developed?
					\item To what extent is minimal documentation recorded/archived?
					\item To what extent is minimal documentation maintained?
				\end{itemize}
		\end{itemize}
\end{itemize}


\chapter{Effectiveness Questions for Company F's Teams}
\label{sec:effectiveness_survey}

{\large \textbf{To what rate are the following implemented?}}

\begin{enumerate}
	\item Iterative Progression {\footnotesize (Develop the product over several iterations/cycles in sequence)}
	\item Incremental Development {\footnotesize (Create the product incrementally. Develop only a selected/prioritized set of bugs/enhancements during a release cycle)}
	\item Short Delivery Cycles {\footnotesize (Deliver valuable software frequently)}
	\item Evolutionary Requirements {\footnotesize (Allow the features/requirements to evolve over the development lifecycle)}
	\item Refactoring {\footnotesize (Refine the architecture, design, code, and/or other process artifacts regularly to improve the quality of that artifact)}
	\item Self-Managing Teams {\footnotesize(Allow the team members to determine, plan, and manage their day-to-day activities and duties under reduced or no supervision)}
	\item Continuous Integration {\footnotesize(Team members integrate their work frequently; usually each person integrates at least daily - leading to multiple integrations per day. Each integration is verified by an automated build (including test) to detect integration errors as quickly as possible)}
	\item Minimal Documentation {\footnotesize (Maintain just-enough documentation to satisfy the needs of the development team and the customer)}
	\item High-bandwidth communication {(Facilitate continuous communication among the (developers, testers, customers) (in-person, face-to-face interactions))}
	\item Client-driven iterations {\footnotesize (The customers and users prioritize the bugs/enhancements. Build only what is of value to the customers and users)} 
	\item Appropriate distribution of expertise {\footnotesize (Select the right people to complete the tasks. Ensure that the team is composed of people with the appropriate skill sets to complete the assigned tasks)}
	\item Configuration Management {\footnotesize (Manage the evolution of the product and other artifacts, both during the initial stages of development and during all stages of maintenance)}
	\item Adherence to Standards {\footnotesize (Conform to a set of standards that the team or organization has agreed to comply with,  e.g. Coding standards.)}	
\end{enumerate}



\chapter{Perceptive Agile Measurement}  %48 in total
\label{sec:pam}

\begin{itemize}
	\item Iteration Planning
		\begin{itemize}
			\item All members of the technical team actively participated during iteration planning meetings
			\item All technical team members took part in defining the effort estimates for requirements of the current iteration
   			\item When effort estimates differed, the technical team members discussed their underlying assumption
   			\item All concerns from team members about reaching the iteration goals were considered
   			\item The effort estimates for the iteration scope items were modified only by the technical team members
   			\item Each developer signed up for tasks on a completely voluntary basis
   			\item The customer picked the priority of the requirements in the iteration plan
		\end{itemize}
	\item Iterative Development
		\begin{itemize}
			\item We implemented our code in short iterations
			\item The team rather reduced the scope than delayed the deadline
			\item When the scope could not be implemented due to constraints, the team held active discussions on re-prioritization with the customer on what to finish within the iteration
			\item We kept the iteration deadlines
			\item At the end of an iteration, we delivered a potentially shippable product
			\item The software delivered at iteration end always met quality requirements of production code
			\item Working software was the primary measure for project progress
		\end{itemize}
	\item Continuous Integration And Testing
		\begin{itemize}
			\item The team integrated continuously
			\item Developers had the most recent version of code available
			\item Code was checked in quickly to avoid code synchronization/integration hassles
			\item The implemented code was written to pass the test case
			\item New code was written with unit tests covering its main functionality
			\item Automated unit tests sufficiently covered all critical parts of the production code
			\item For detecting bugs, test reports from automated unit tests were systematically used to capture the bugs
			\item All unit tests were run and passed when a task was finished and before checking in and integrating
			\item There were enough unit tests and automated system tests to allow developers to safely change any code
		\end{itemize}
	\item Stand-Up Meetings
		\begin{itemize}
			\item Stand up meetings were extremely short (max. 15 minutes)
			\item Stand up meetings were to the point, focusing only on what had been done and needed to be done on that day
			\item All relevant technical issues or organizational impediments came up in the stand up meetings
			\item Stand up meetings provided the quickest way to notify other team members about problems
			\item When people reported problems in the stand up meetigs, team members offered to help instantly
		\end{itemize}
	\item Customer Access
		\begin{itemize}
			\item The customer was reachable
			\item The developers could contact the customer directly or through a customer contact person without any bureaucratical hurdles
			\item The developers had responses from the customer in a timely manner
			\item The feedback from the customer was clear and clarified his requirements or open issues to the developers
		\end{itemize}
	\item Customer Acceptance Tests
		\begin{itemize}
			\item How often did you apply customer acceptance tests?
			\item A requirement was not regarded as finished until its acceptance tests (with the customer) had passed
			\item Customer acceptance tests were used as the ultimate way to verify system functionality and customer requirements
			\item The customer provided a comprehensive set of test criteria for customer acceptance
			\item The customer focused primarily on customer acceptance tests to determine what had been accomplished at the end of an iteration
		\end{itemize}
	\item Retrospectives
		\begin{itemize}
			\item How often did you apply retrospectives?
			\item All team members actively participated in gathering lessons learned in the retrospectives
			\item The retrospectives helped us become aware of what we did well in the past iteration(s)
			\item The retrospectives helped us become aware of what we should improve in the upcoming iteration(s)
			\item In the retrospectives (or shortly afterwards), we systematically assigned all important points for improvement to responsible individuals
			\item Our team followed up intensively on the progress of each improvement point elaborated in a retrospective
		\end{itemize}
	\item Co-Location %Have to continue this
		\begin{itemize}
			\item Developers were located majorly in
			\item All members of the technical team (including QA engineers, db admins) were located in
			\item Requirements engineers were located with developers in
			\item The project/release manager worked with the developers in
			\item The customer was located with the developers in
		\end{itemize}
	
\end{itemize}




\chapter{Team Agility Assessment}
\label{sec:team_agility_assessment} % 57 in total

\begin{itemize}
	\item Product Ownership
		\begin{itemize}
			\item Backlog prioritized and ranked by business value 
			\item Backlog estimated at gross level 
			\item Product owner defines acceptance criteria for stories 
			\item Product owner and stakeholders participate at iteration and release planning 
			\item Product owner and stakeholders participate at iteration and release review 
			\item Product owner collaboration with team is continuous 
			\item Stories sufficiently elaborated prior to planning meetings
		\end{itemize}
	\item Release Planning and Tracking
		\begin{itemize}
			\item Release theme established and communicated 
			\item Release planning meeting attended and effective 
			\item Release backlog defined 
			\item Release backlog ranked by priority 
			\item Release backlog estimated at plan level 
			\item The team has small and frequent releases
			\item The team has a common language and metaphor to describe the release
			\item Release progress tracked by feature acceptance
			\item Team completes and product owner accepts the release by the release date
			\item Release review meeting attended and effective
			\item Team inspects and adapts (continuous improvement) the release plan
			\item Team meets its commitments to release
		\end{itemize}
	\item Iteration Planning and Tracking
		\begin{itemize}
			\item Iteration theme established and communicated
			\item Iteration planning meeting attended and effective
			\item Team velocity measured and used for planning
			\item Iteration backlog defined
			\item Iteration backlog ranked by priority
			\item Team develops and manages iteration backlog
			\item Team defines, estimates, and selects their own work (stories and tasks)
			\item Team discusses acceptance criteria during iteration planning
			\item Team manages interdependencies and constraints
			\item Iteration progress tracked by task to do (burn-down chart) and card acceptance (velocity)
			\item Work is not added by the product owner during the iteration
			\item Team completes and product owner accepts the iteration 
			\item Iterations are of a consistent fixed length
			\item Iterations are no more than four weeks in length
			\item Iteration review meeting attended and effective
			\item Team inspects and adapts (continuous improvement) the Iteration Plan
		\end{itemize}
	\item Team
		\begin{itemize}
			\item The whole team is present at release planning meetings
			\item Team is cross-functional with integrated product owner, development, documentation and QA
			\item Team is colocated
			\item Team is 100\% dedicated to the release (no time-slicing)
			\item Team is smaller than 15 people
			\item Team works in a physical environment that fosters collaboration
			\item Team works at a sustainable pace
			\item Team members complete commitments
			\item Daily standup on time, fully attended and effectively communicates
			\item Team leads communication; communication not managed
			\item Team self-polices and reinforces use of agile practices and rules
			\item Team inspects and adapts (continuous improvement) the overall process
			\item Team Coach/Scrum Master exists, is full-time, and is effective
			\item The team has an effective channel for obstacle escalation
		\end{itemize}
	\item Testing Practices
		\begin{itemize}
			\item All testing is done within the iteration and does not lag behind
			\item Iteration defects are fixed within that iteration
			\item Unit tests written before development
			\item Acceptance tests written before development
			\item 100\% automated unit test coverage
			\item Automated acceptance tests
		\end{itemize}
	\item Development Practices/Infrastructure
		\begin{itemize}
			\item Source control system exists
			\item Continuous build with 100\% successful builds
			\item Developers integrate code multiple times per day
			\item Team has administrative access to their own workstations
			\item Team has administrative control over their development environment
			\item Team is permitted to refactor anywhere in the code base
			\item Adequate and effective code review practices
			\item Coding standards exist and applied
			\item Stories accepted and demonstrated on integrated build
			\item Refactoring is continuous
			\item Pair programming is practiced
			\item Identical builds for developers' workstations
		\end{itemize}
\end{itemize}









\end{appendices}
