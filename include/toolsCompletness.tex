\chapter{Tools Completness}

\textbf{RQ \#2 - How much complete are the tools in measuring agility?}

\section{OPP Practices}
OPS Framework is the successor of the Objectives, Principles, Practices (OPP) Framework \cite{opp}. OPP identified 27 practices as implementations of the principles which later on were transformed into 17 strategies. In Table~\ref{table:opp_practices} one can see OPP's practices.

\begin{tabular}{| p{7.5cm}  p{7cm} |}
	\hline
	\multicolumn{2}{|c|}{\textbf{OPP Practices}}  \\ \hline
     	\begin{itemize}
     		\item Iterative and Incremental Development 
     		\item Continuous Feedback 
     		\item Evolutionary Requirements 
     		\item Smaller and Frequent Product Releases 
     		\item Customer/User Acceptance Testing 
     		\item Frequent Face-to-Face Communication 
     		\item Refactoring 
     		\item Automated Test Builds 
     		\item Software Configuration Management 
     		\item Test First Development 
     		\item Iteration Progress Tracking and Reporting 
     		\item Code Ownership 
     		\item Retrospectives Meetings 
     		\item Just-in-Time Refinement of Features /Stories/Tasks 
     		\end{itemize} 
     	& \begin{itemize}
     		\item Appropriate Distribution of Expertise 
     		\item Self-Managing Teams 
     		\item Client-Driven Iterations 
     		\item Product Backlog 
     		\item Agile Project Estimation 
     		\item Adherence to Coding Standards 
     		\item Physical Setup Reflecting Agile Philosophy 
     		\item Daily Progress Tracking Meetings 
     		\item Minimal or Just Enough Documentation 
     		\item Minimal Big Requirements Up Front and Big Design Up Front 
     		\item Collocated Customers 
     		\item Constant Velocity 
     		\item Pair Programming  
 		\end{itemize} 
     \\ \hline
\end{tabular}
\captionof{table}{Agile practices covered by OPP}
\label{table:opp_practices}

\section{PAM Practices}
\section{Leffingwell Practices}

%\newenvironment{multi_enumerate}{%
%	\begin{enumerate} 
%	\begin{multicols}{2}
%}{%
%	\end{multicols}
%	\end{enumerate}
%}

\begin{minipage}[b]{0.45\textwidth}
  \centering
  \begin{tabular}{| p{7cm} |}
    \hline
     \textbf{PAM Practices}\\ \hline
     \begin{itemize} \item Iteration Planning \item Iterative Development \item Continuous Integration and Testing \item Co-Location \item Stand-up Meetings \item Customer Access \item Customer Acceptance Tests \item Retrospectives \end{itemize}  \\ \hline
  \end{tabular}
  \captionof{table}{Agile practices coverd by PAM}
  \label{table:pam_practices}
\end{minipage}\qquad
\begin{minipage}[b]{0.5\textwidth}
  \centering
  \begin{tabular}{| p{7cm} |}
    \hline
     \textbf{Leffingwell Areas}\\ \hline
     \begin{itemize} \item Product Ownership \item Release Planning and Tracking \item Iteration Planning and Tracking \item Team \item Testing Practices \item Development Practices / Infrastructure \end{itemize}  \\ \hline
  \end{tabular}
  \captionof{table}{Agile practices coverd by Leffingwell}
  \label{table:leffingwell_practices}
\end{minipage}

\section{Practices Covered Between The Tools}

As it was clearly seen between Tables~\ref{table:opp_practices}, \ref{table:pam_practices} \ref{table:leffingwell_practices} OPP and as a consequence OPS covers more agile practices than the other tools. \\

In the next pages follows a mapping between OPP and PAM (see Table~\ref{table:opp_pam_practices}) and OPP and Leffingwell (see Table~\ref{table:opp_leffingwell_practices}). The connection between the practices is done based on the questions of each tool. The aforementioned connections are depicted with colours. When a practice has more than one colour, it is because it covers more practices from the other tool {\footnotesize (The colours among  Tables~\ref{table:opp_pam_practices}, ~\ref{table:opp_leffingwell_practices} are randomly selected and do not imply any connetion between the practices)}.

\begin{tabular}{| p{7.5cm} | p{7cm} |}
	\hline
	\textbf{OPS} & \textbf{PAM}  \\ \hline
     	\begin{itemize}
     		\item {\color{RoyalBlue1}Iterative} {\color{DarkMagenta}and Incremental Development} 
     		\item {\color{DarkBlue}Continuous Feedback} 
     		\item {\color{red2}Customer/User Acceptance Testing} 
     		\item {\color{DarkBlue}Frequent Face-to-Face} {\color{MediumAquamarine}Communication} 
     		\item {\color{DarkOrange1}Automated Test Builds} 
     		\item {\color{DarkOrange1}Software Configuration Management} 
     		\item {\color{DarkOrange1}Test First Development} 
     		\item {\color{RoyalBlue1}Iteration Progress Tracking and Reporting}
     		\item {\color{DarkRed}Retrospectives Meetings} 
     		\item {\color{DarkBlue}Client-Driven} {\color{RoyalBlue1}Iterations}
     		\item {\color{MediumAquamarine}Physical Setup Reflecting Agile Philosophy} 
     		\item {\color{green4}Daily Progress Tracking Meetings} 
     		\item {\color{DarkBlue}Collocated} {\color{MediumAquamarine}Customers}
 		\end{itemize} 
 		& \begin{itemize}
 			\item {\color{RoyalBlue1}Iteration Planning} 
 			\item {\color{DarkMagenta}Iterative Development} 
 			\item {\color{DarkOrange1}Continuous Integration and Testing} 
 			\item {\color{MediumAquamarine}Co-Location} 
 			\item {\color{green4}Stand-up Meetings} 
 			\item {\color{DarkBlue}Customer Access} 
 			\item {\color{red2}Customer Acceptance Tests} 
 			\item {\color{DarkRed}Retrospectives} 
 		\end{itemize}
     \\ \hline
\end{tabular}
\captionof{table}{Relation of OPP and PAM practices}
\label{table:opp_pam_practices}


\begin{tabular}{| p{7.5cm} | p{7cm} |}
	\hline
	\textbf{OPS} & \textbf{Leffingwell}  \\ \hline
     	\begin{itemize}
     		\item {\color{green4}Smaller and Frequent Product Releases} 
     		\item {\color{RoyalBlue1}Customer/User Acceptance Testing}
     		\item {\color{DarkRed}Frequent Face-to-Face Communication} 
     		\item {\color{DarkMagenta}Refactoring}
     		\item {\color{DarkOrange1}Automated Test Builds} 
     		\item {\color{DarkMagenta}Software Configuration Management} 
     		\item {\color{DarkOrange1}Test First Development}
     		\item {\color{RoyalBlue1}Iteration Progress Tracking and Reporting} 
     		\item {\color{DarkRed}Appropriate Distribution of Expertise}
     		\item {\color{DarkRed}Self-Managing} {\color{RoyalBlue1}Teams}
     		\item Product Backlog ????
     		\item {\color{RoyalBlue1}Agile Project Estimation}
     		\item {\color{DarkMagenta}Adherence to Coding Standards} 
     		\item {\color{DarkRed}Physical Setup Reflecting Agile Philosophy}
     		\item {\color{DarkRed}Daily Progress Tracking Meetings}
     		\item {\color{RoyalBlue1}Constant Velocity}
     		\item {\color{DarkMagenta}Pair Programming}
     	\end{itemize} 
     	& \begin{itemize} 
     		\item Product Ownership 
     		\item {\color{green4}Release Planning and Tracking}
     		\item {\color{RoyalBlue1}Iteration Planning and Tracking} 
     		\item {\color{DarkRed}Team}
     		\item {\color{DarkOrange1}Testing Practices}
     		\item {\color{DarkMagenta}Development Practices / Infrastructure}  	
 		\end{itemize} 
     \\ \hline
\end{tabular}
\captionof{table}{Relation of OPP and Leffingwell practices}
\label{table:opp_leffingwell_practices}

\section{Result}
By viewing tables Tables~\ref{table:opp_pam_practices}, ~\ref{table:opp_leffingwell_practices} one can clearly see that OPS is more complete than the others in measuring agility.