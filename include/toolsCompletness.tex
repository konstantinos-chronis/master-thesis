\chapter{Enhancing OPS}
\label{ch:enhancing_ops}


%To create a shorter version of a well-established measurement procedure. You want to create a shorter version of an existing measurement procedure, which is unlikely to be achieved through simply removing one or two measures within the measurement procedure (e.g., one or two questions in a survey), possibly because this would affect the content validity of the measurement procedure [see the article: Content validity]. Therefore, you have to create new measures for the new measurement procedure. However, to ensure that you have built a valid new measurement procedure, you need to compare it against one that is already well-established; that is, one that already has demonstrated construct validity and reliability [see the articles: Construct validity and Reliability in research]. This well-established measurement procedure is the criterion against which you are comparing the new measurement procedure (i.e., why we call it criterion validity).
%
%Indeed, sometimes a well-established measurement procedure (e.g., a survey), which has strong construct validity and reliability, is either too long or longer than would be preferable. A measurement procedure can be too long because it consists of too many measures (e.g., a 100 question survey measuring depression). Whilst the measurement procedure may be content valid (i.e., consist of measures that are appropriate/relevant and representative of the construct being measured), it is of limited practical use if response rates are particularly low because participants are simply unwilling to take the time to complete such a long measurement procedure. We also stated that a measurement procedure may be longer than would be preferable, which mirrors that argument above; that is, that it's easier to get respondents to complete a measurement procedure when it's shorter. However, the one difference is that an existing measurement procedure may not be too long (e.g., having only 40 questions in a survey), but would encourage much greater response rates if shorter (e.g., having just 18 questions). This may be a time consideration, but it is also an issue when you are combining multiple measurement procedures, each of which has a large number of measures (e.g., combining two surveys, each with around 40 questions).
%http://dissertation.laerd.com/criterion-validity-concurrent-and-predictive-validity.php


\lettrine[lines=2, loversize=-0.1, lraise=0.1]{T}{his} chapter makes an effort to combine \ac{PAM}, \ac{TAA} and \ac{OPS} in a more enhanced tool, based on \ac{OPS}'s structure. As it was discussed in Section~\ref{subsubsec:practices_among_tools}, \ac{OPP}, and subsequently \ac{OPS}, covers all the practices and areas of both \ac{PAM} and \ac{TAA}. As a result, we consider that by combining these three tools we will end up with a tool that covers the agile practices to a wider degree/in a wider manner.

\section{\ac{OPS} Enhancement}
The combination of the tools took place based on the analysis performed in section~\ref{subsubsec:practices_among_tools} and section~\ref{sec:coverage_results}. Since \ac{OPS} covers more agile practices, it was selected to remain as it is and be enhanced with the questions from \ac{PAM} and \ac{TAA}, which have been transformed to match the style of \ac{OPS}.

Although this case study used only the ``Effectiveness" survey from \ac{OPS}, it was selected to enhance the ``Capability" part as well, since some of the \ac{PAM} and \ac{TAA} questions fit there more.

\subsection{Questions Excluded}
The questions excluded belong only to \ac{TAA}. Most of them were referring to product ownership and were needless, since \ac{OPS} focuses on measuring teams practising agile development methods like eXtreme Programming \cite{Beck:2004:EPE:1076267}, Crystal \cite{Cockburn:2004:CCH:1406822}, First Driven Development \cite{Palmer:2001:PGF:600044} and not agile project management methods like Scrum \cite{scrum}. The other questions excluded were about iteration defects being fixed within the iteration. Although a software without defects is always welcome, trying to mitigate all of them throws away some of the team's agility, since the release could be delayed. Defects should be fixed with a delay in the software's release only if they are a stopper for the business of the customer.

\subsection{Questions Added}
The questions which already existed in \ac{OPS} remained as they were, apart from the question ``To what extent are the code bases not shared", which was changed since we consider that for achieving \textit{Continuous Integration} the developers should be synced to the latest code available. The questions which were added were formulated and positioned under the appropriate \ac{OPS} indicator. When an indicator did not cover the questions added, a new one was created. The \ac{OPS} questions follow the pattern of starting a question with the phrase ``To what extent \dots ". The same pattern was used in the additional questions. In Table~\ref{table:summary_questions_added}, one can see the number of indicators and questions added. 

\begin{table} [H]
	\caption{Summary of Indicators and Questions Added}
	\label{table:summary_questions_added}
	\begin{tabular}{| c | c | c |} \hline
		 & \textbf{Indicators Introduced} & \textbf{Questions Introduced} \\ \hline
		 \textbf{Capability} & 3 & 7 \\ \hline
		 \textbf{Effectiveness} & 9 & 46 \\ \hline
	\end{tabular}
\end{table}

In total, 11 indicators and 53 questions were introduced. The questions did not exist in \ac{OPS} and they cover a variety of aspects which were considered important by the creators of \ac{PAM} and \ac{TAA}. The \ac{OPS} now has in total 46 indicators and 77 questions for ``Capability" and 45 indicators and 126 questions for ``Effectiveness". Although \ac{OPS} supports 17 strategies, and ``Velocity" is one of them, it was not used in the surveys. On the contrary, questions in \ac{PAM} and \ac{TAA} supported this strategy, thus it was added to both ``Capability" and ``Effectiveness". In Table~\ref{table:numbers_enhanced_ops}, one can see in more detail the numbers of indicators and questions for each strategy. In Appendix~\ref{ch:ops_pam_taa}, one can see the enhanced version of \ac{OPS}. Below is the list of symbols that describe the type of addition.

\begin{itemize}
	\item strategy addition - \TwelweStar
	\item indicator addition - \FiveStarOutline
	\item question addition - \FiveStar
\end{itemize}

\begin{table}
	\caption{Numbers of indicators and questions in the enhanced \ac{OPS}}
	\label{table:numbers_enhanced_ops}
	\begin{tabular}{| p{.33\textwidth} | p{.13\textwidth} | p{.13\textwidth} | p{.13\textwidth} | p{.13\textwidth} |} \hline
	\multirow{2}{*}{\textbf{Strategy}} & \multicolumn{2}{c|}{\textbf{Capability}} & \multicolumn{2}{c|}{\textbf{Effectiveness}} \\ \hhline{~----}
	& \textbf{No. of Indicators}  & \textbf{No. of Questions} & \textbf{No. of Indicators} & \textbf{No. of Questions} \\ \hline
	Iterative Progression & 3 & 4 & 4 & 18 \\ \hline
	Incremental Development & 3 & 4 & 2 & 8 \\ \hline
	Short Delivery Cycles & 1 & 1 & 3 & 4 \\ \hline
	Evolutionary Requirements & 3 & 4 & 3 & 5 \\ \hline
	Continuous Feedback & 2 & 2 & 3 & 5 \\ \hline
	Distribution of Expertise & 1 & 3 & 1 & 6 \\ \hline
	Test-first development & 3 & 5 & 3 & 5 \\ \hline
	Refactoring & 3 & 9 & 2 & 7 \\ \hline
	Adherence to standards & 5 & 6 & 2 & 3 \\ \hline
	Continuous Integration & 4 & 10 & 4 & 15 \\ \hline
	Configuration Management & 2 & 6 & 1 & 1 \\ \hline
	Minimal Documentation & 3 & 3 & 1 & 4 \\ \hline
	High bandwidth Communication & 4 & 9 & 5 & 21 \\ \hline
	Self-Managing Teams & 5 & 6 & 4 & 9 \\ \hline
	Client-driven Iterations & 1 & 2 & 3 & 4 \\ \hline
	Retrospection & 2 & 2 & 3 & 10 \\ \hline
	Velocity & 1 & 1 & 1 & 1 \\ \hline
	\textbf{Total} & \textbf{46} & \textbf{77} & \textbf{45} & \textbf{126}  \\ \hline
	\end{tabular}
\end{table}


%Mention in which practices you added questions or what type of questions were added. Mention about velocity addition

%Answer to the research question


\section{Chapter Summary}
In this chapter there was an effort to enhance \ac{OPS} with questions from \ac{PAM} and \ac{TAA}. The questions added cover \ac{OPS}'s weaknesses. After the changes, \ac{OPS} has in total 46 indicators and 77 questions for ``Capability" and 45 indicators and 126 questions for ``Effectiveness".






%Team size
%http://www.jamesshore.com/Agile-Book/the_xp_team.html
%http://ieeexplore.ieee.org/xpl/articleDetails.jsp?tp=&arnumber=6681108&matchBoolean%3Dtrue%26searchField%3DSearch_All%26queryText%3D%28%28Agile%29+AND+team+size%29





%\begin{itemize}
%	\item Product owner defines acceptance criteria for stories 
%	\item Product owner and stakeholders participate at iteration and release planning 
%	\item Product owner and stakeholders participate at iteration and release review 
%	\item Product owner collaboration with team is continuous
%	\item Team completes and product owner accepts the release by the release date
%	\item Work is not added by the product owner during the iteration
%	\item Team completes and product owner accepts the iteration
%\end{itemize}