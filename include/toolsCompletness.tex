\chapter{Tools Completness}

\textbf{RQ \#2 - How much complete are the tools in measuring agility?}

\section{OPP Practices} %change numbers
OPS Framework is the successor of the Objectives, Principles, Practices (OPP) Framework \cite{opp}. OPP identified 27 practices as implementations of the principles which later on were transformed into 17 strategies. In Table~\ref{table:opp_practices} one can see OPP's practices.

\begin{tabular}{| p{7.5cm}  p{7.5cm} |}
	\hline
	\multicolumn{2}{|c|}{\textbf{OPP Practices}}  \\ \hline
     	\begin{itemize}
     		\item Iterative and Incremental Development 
     		\item Continuous Feedback 
     		\item Evolutionary Requirements 
     		\item Smaller and Frequent Product Releases 
     		\item Customer/User Acceptance Testing 
     		\item Frequent Face-to-Face Communication 
     		\item Refactoring 
			\item Continuous Integration
     		\item Software Configuration Management 
     		\item Test First Development 
     		\item Iteration Progress Tracking and Reporting 
     		\item Code Ownership 
     		\item Retrospectives Meetings 
     		\item Just-in-Time Refinement of Features /Stories/Tasks 
     	\end{itemize} 
     	& \begin{itemize}
     	 	\item Appropriate Distribution of Expertise
  			\item Self-Managing Teams 
     		\item Client-Driven Iterations 
     		\item Product Backlog 
     		\item Agile Project Estimation 
     		\item Adherence to Coding Standards 
     		\item Physical Setup Reflecting Agile Philosophy 
     		\item Daily Progress Tracking Meetings 
     		\item Minimal or Just Enough Documentation 
     		\item Minimal Big Requirements Up Front and Big Design Up Front 
     		\item Collocated Customers 
     		\item Constant Velocity 
     		\item Pair Programming  
 		\end{itemize} 
     \\ \hline
\end{tabular}
\captionof{table}{Agile practices covered by OPP}
\label{table:opp_practices}

\section{PAM Practices}
\section{Leffingwell Practices}

%\newenvironment{multi_enumerate}{%
%	\begin{enumerate} 
%	\begin{multicols}{2}
%}{%
%	\end{multicols}
%	\end{enumerate}
%}

\begin{minipage}[b]{0.45\textwidth}
  \centering
  \begin{tabular}{| p{7cm} |}
    \hline
     \textbf{PAM Practices}\\ \hline
     \begin{itemize} \item Iteration Planning \item Iterative Development \item Continuous Integration and Testing \item Co-Location \item Stand-up Meetings \item Customer Access \item Customer Acceptance Tests \item Retrospectives \end{itemize}  \\ \hline
  \end{tabular}
  \captionof{table}{Agile practices coverd by PAM}
  \label{table:pam_practices}
\end{minipage}\qquad
\begin{minipage}[b]{0.5\textwidth}
  \centering
  \begin{tabular}{| p{7cm} |}
    \hline
     \textbf{Leffingwell Areas}\\ \hline
     \begin{itemize} \item Product Ownership \item Release Planning and Tracking \item Iteration Planning and Tracking \item Team \item Testing Practices \item Development Practices / Infrastructure \end{itemize}  \\ \hline
  \end{tabular}
  \captionof{table}{Agile practices coverd by Leffingwell}
  \label{table:leffingwell_practices}
\end{minipage}

\section{Practices Covered Between The Tools}

As it was clearly seen between Tables~\ref{table:opp_practices}, \ref{table:pam_practices} \ref{table:leffingwell_practices} OPP and as a consequence OPS covers more agile practices than the other tools. \\

In the next pages follows a mapping between OPP and PAM (see Table~\ref{table:opp_pam_practices}) and OPP and Leffingwell (see Table~\ref{table:opp_leffingwell_practices}). The connection between the practices is done based on the questions of each tool. The aforementioned connections are depicted with colours. When a practice has more than one colour, it is because it covers more practices from the other tool {\footnotesize (The colours among  Tables~\ref{table:opp_pam_practices}, ~\ref{table:opp_leffingwell_practices} are randomly selected and do not imply any connetion between the practices)}.


%Analyze each the practices between them

\begin{tabular}{| p{7.5cm} | p{7cm} |}
	\hline
	\textbf{OPS} & \textbf{PAM}  \\ \hline
     	\begin{itemize}
     		\item {\color{RoyalBlue1}Iteration Progress Tracking and Reporting}
     		\item {\color{RoyalBlue1}Iterative} {\color{DarkMagenta}and Incremental Development} 
     		\item {\color{DarkOrange1}Continuous Integration} 
     		\item {\color{DarkOrange1}Software Configuration Management} 
     		\item {\color{DarkOrange1}Test First Development} 
     		\item {\color{DeepPink1}Physical Setup Reflecting Agile Philosophy} 
     		\item {\color{DarkBlue}Collocated} {\color{DeepPink1}Customers}
     		\item {\color{DarkBlue}Frequent Face-to-Face} {\color{DeepPink1}Communication} 
     		\item {\color{green4}Daily Progress Tracking Meetings} 
     		\item {\color{DarkBlue}Client-Driven} {\color{RoyalBlue1}Iterations}
     		\item {\color{DarkBlue}Continuous Feedback} 
     		\item {\color{red2}Customer/User Acceptance Testing}
     		\item {\color{DarkRed}Retrospectives Meetings}
 		\end{itemize} 
 		& \begin{itemize}
 			\item {\color{RoyalBlue1}Iteration Planning} 
 			\item {\color{DarkMagenta}Iterative Development} 
 			\item {\color{DarkOrange1}Continuous Integration and Testing} 
 			\item {\color{DeepPink1}Co-Location} 
 			\item {\color{green4}Stand-up Meetings} 
 			\item {\color{DarkBlue}Customer Access} 
 			\item {\color{red2}Customer Acceptance Tests} 
 			\item {\color{DarkRed}Retrospectives} 
 		\end{itemize}
     \\ \hline
\end{tabular}
\captionof{table}{Relation of OPP and PAM practices}
\label{table:opp_pam_practices}

%Maybe separate to capability and effectiveness?

\begin{tabular}{| p{7.5cm} | p{7cm} |}
	\hline
	\textbf{OPS} & \textbf{Leffingwell}  \\ \hline
     	\begin{itemize}
     		\item {\color{DeepPink1} Iterative and Incremental Development}
     	    \item {\color{DeepPink1}Product Backlog}
     		\item {\color{green4}Smaller and Frequent Product Releases} 
     		\item {\color{RoyalBlue1}Customer/User {\color{green4}Acceptance Testing}}  
     		\item {\color{RoyalBlue1}Constant Velocity}		
     		\item {\color{RoyalBlue1}Iteration Progress Tracking and Reporting}
     		\item {\color{DarkRed}Self-} {\color{green4}Mana}{\color{RoyalBlue1}ging} {\color{DarkMagenta}Teams} 
     		\item {\color{DarkRed}Appropriate Distribution of Expertise}
     		\item {\color{DarkRed}Frequent Face-to-Face Communication} 
     		\item {\color{DarkRed}Physical Setup Reflecting Agile Philosophy}
     		\item {\color{DarkRed}Daily Progress Tracking Meetings}
     		\item {\color{DarkRed}Retro}{\color{RoyalBlue1}spectives} {\color{green4}Meetings} 
     		\item {\color{DarkOrange1}Test First Development}
     		\item {\color{DarkMagenta}Refactoring}
     		\item {\color{DarkMagenta}Software Configuration Management} 
     		\item {\color{DarkMagenta}Adherence to Coding Standards} 
     		\item {\color{DarkMagenta}Pair Programming}
     		\item {\color{DarkMagenta}Continuous} {\color{DarkOrange1}Integration}
     	\end{itemize} 
     	& \begin{itemize} 
     		\item {\color{DeepPink1}Product Ownership}
     		\item {\color{green4}Release Planning and Tracking}
     		\item {\color{RoyalBlue1}Iteration Planning and Tracking} 
     		\item {\color{DarkRed}Team}
     		\item {\color{DarkOrange1}Testing Practices}
     		\item {\color{DarkMagenta}Development Practices / Infrastructure}  	
 		\end{itemize} 
     \\ \hline
\end{tabular}
\captionof{table}{Relation of OPP and Leffingwell practices}
\label{table:opp_leffingwell_practices}


\textbf{Iterative and Incremental Development}
\begin{itemize}
	\item Stories sufficiently elaborated prior to planning meetings
\end{itemize}

\textbf{Product Backlog}
\begin{itemize}
	\item Backlog prioritized and ranked by business value
	\item Backlog estimated at gross level
\end{itemize}

\textbf{Smaller and Frequent Product Releases}
\begin{itemize}
	\item The team has small and frequent releases
\end{itemize}

\textbf{Customer/User Acceptance Testing}
\begin{itemize}
	\item Team discusses acceptance criteria during iteration planning
	\item Release progress tracked by feature acceptance
\end{itemize}

\textbf{Constant Velocity}
\begin{itemize}
	\item Team velocity measured and used for planning
	\item Iteration progress tracked by task to do (burn-down chart) and card acceptance (velocity)
\end{itemize}

\textbf{Iteration Progress Tracking and Reporting}
\begin{itemize}
	\item Iteration backlog defined
	\item Iteration backlog ranked by priority
	\item Team develops and manages iteration backlog
	\item Iteration planning meeting attended and effective
	\item Iterations are no more than four weeks in length
	\item Iterations are of a consistent fixed length
	\item Iteration theme established and communicated 
\end{itemize}

\textbf{Self-Managing Teams}
\begin{itemize}
	\item Team self-polices and reinforces use of agile practices and rules
	\item Team leads communication; communication not managed
	\item Team defines, estimates, and selects their own work (stories and tasks)
	\item Team develops and manages iteration backlog
	\item Team manages interdependencies and constraints
	\item Team has administrative access to their own workstations 
	\item Team has administrative control over their development environment
\end{itemize}

\textbf{Appropriate Distribution of Expertise}
\begin{itemize}
	\item Team is cross-functional with integrated product owner, development, documentation and QA
	\item Team manages interdependencies and constraints
	\item Team Coach/Scrum Master exists, is full-time, and is effective
\end{itemize}

\textbf{Frequent Face-to-Face Communication}
\begin{itemize}
	\item Team is collocated
	\item Team works in a physical environment that fosters collaboration
\end{itemize}

\textbf{Physical Setup Reflecting Agile Philosophy}
\begin{itemize}
	\item Team is co-located
	\item Team works in a physical environment that fosters collaboration
\end{itemize}

\textbf{Daily Progress Tracking Meetings}
\begin{itemize}
	\item Daily standup on time, fully attended and effectively communicates
\end{itemize}

\textbf{Retrospective Meetings}
\begin{itemize}
	\item Team inspects and adapts (continuous improvement) the overall process
	\item Team inspects and adapts (continuous improvement) the Iteration Plan 
	\item Team inspects and adapts (continuous improvement) the release plan
	\item Iteration review meeting attended and effective
	\item Release review meeting attended and effective
\end{itemize}

\textbf{Test First Development}
\begin{itemize}
	\item Unit tests written before development
	\item Acceptance tests written before development
\end{itemize}

\textbf{Refactoring}
\begin{itemize}
	\item Refactoring is continuous
	\item Team is permitted to refactor anywhere in the code base
\end{itemize}

\textbf{Software Configuration Management}
\begin{itemize}
	\item Source control system exists
\end{itemize}

\textbf{Adherence to Coding Standards}
\begin{itemize}
	\item Coding standards exist and applied
\end{itemize}

\textbf{Pair Programming}
\begin{itemize}
	\item Pair programming is practiced
\end{itemize}

\textbf{Continuous Integration}
\begin{itemize}
	\item Developers integrate code multiple times per day
	\item Stories accepted and demonstrated on integrated build
	\item Continuous build with 100\% successful builds
	\item Automated acceptance tests
	\item 100\% automated unit test coverage
\end{itemize}

\section{Result}
By viewing tables Tables~\ref{table:opp_pam_practices}, ~\ref{table:opp_leffingwell_practices} one can clearly see that OPS is more complete than the others in measuring agility.