\chapter{Tools Completeness}

\textbf{RQ \#2 - How much complete are the tools in measuring agility?}

\section{Team Agility Assessment Areas}
Team Agility Assessment (TAA) does not state covering specific agile practices, but rather areas important for a team. It focuses on product ownership for Scrum teams but also on the release and iteration planning and tracking. The team factor plays a great role but also the development practices and the working environment. Automated testing is important here as well. Finally it is worth mentioning that it is the only tool focusing so much on the release planning. In Table~\ref{table:taa_practices} one can see TAA's areas.

\begin{table} [H]
  \begin{tabular}{| p{6.8cm} |}
    \hline
     \textbf{TAA Areas}\\ \hline
     \begin{itemize} \item Product Ownership \item Release Planning and Tracking \item Iteration Planning and Tracking \item Team \item Testing Practices \item Development Practices / Infrastructure \end{itemize}  \\ \hline
  \end{tabular}
  \captionof{table}{Areas covered by TAA}
  \label{table:taa_practices}
\end{table}

\section[PAM Practices]{Perceptive Agile Measurement Practices}
The Perceptive Agile Measurement (PAM) tool focuses on the iterations during software development but also on the stand-up meetings among the team members, their collocation and the retrospectives they have. The customers access and their acceptance criteria have a high significance as well. Finally the continuous integration and the automated unit testing are considered crucial in order to be agile. In Table~\ref{table:pam_practices} one can see PAM's practices.

\begin{table} [H]
  \begin{tabular}{| p{6.8cm} |}
    \hline
     \textbf{PAM Practices}\\ \hline
     \begin{itemize} \item Iteration Planning \item Iterative Development \item Continuous Integration and Testing \item Co-Location \item Stand-up Meetings \item Customer Access \item Customer Acceptance Tests \item Retrospectives \end{itemize}  \\ \hline
  \end{tabular}
  \captionof{table}{Agile practices covered by PAM}
  \label{table:pam_practices}
\end{table}

\section{Objectives, Principles, Strategies Practices}
Objectives, Principles, Strategies (OPS) Framework is the successor of the Objectives, Principles, Practices (OPP) Framework \cite{opp}. OPP identified 27 practices as implementations of the principles which later on were transformed into 17 strategies. In Table~\ref{table:opp_practices} one can see OPP's practices. 

\begin{table} [H]
\begin{tabular}{| p{7.5cm}  p{7.5cm} |}
	\hline
	\multicolumn{2}{|c|}{\textbf{OPP Practices}}  \\ \hline
     	\begin{itemize}
     		\item Iterative and Incremental Development 
     		\item Continuous Feedback 
     		\item Evolutionary Requirements 
     		\item Smaller and Frequent Product Releases 
     		\item Customer/User Acceptance Testing 
     		\item Frequent Face-to-Face Communication
     		\item Refactoring 
     		\item Automated Test Builds
     		\item Software Configuration Management 
     		\item Test Driven Development
     		\item Iteration Progress Tracking and Reporting 
     		\item Code Ownership 
     		\item Retrospectives Meetings 
     		\item Just-in-Time Refinement of Features /Stories/Tasks 
     	\end{itemize} 
     	& \begin{itemize}
     	 	\item Appropriate Distribution of Expertise
  			\item Self-Organizing Teams
     		\item Client-Driven Iterations 
     		\item Product Backlog 
     		\item Agile Project Estimation 
     		\item Adherence to Coding Standards 
     		\item Physical Setup Reflecting Agile Philosophy
     		\item Daily Progress Tracking Meetings 
     		\item Minimal or Just Enough Documentation 
     		\item Minimal Big Requirements Up Front and Big Design Up Front 
     		\item Collocated Customers
     		\item Constant Velocity 
     		\item Pair Programming  
 		\end{itemize} 
     \\ \hline
\end{tabular}
\captionof{table}{Agile practices covered by OPP}
\label{table:opp_practices}
\end{table}

\section[Tool Practices]{Practices Covered Between The Tools}

As it can be clearly seen between Tables~\ref{table:opp_practices}, \ref{table:pam_practices} \ref{table:taa_practices} OPP and as a consequence OPS covers more agile practices than the other tools. \\

In the next pages follows a mapping between OPP and PAM (see Table~\ref{table:opp_pam_practices}) and OPP and TAA (see Table~\ref{table:opp_taa_practices}). \\

Some of the OPP practices though have abstracted to OPS strategies in order to avoid repetition of the questions' mapping and in order to better reflect the OPS Framework. The OPP practices \begin{inparaenum} [a\upshape)]
	\item \textit{Frequent Face-to-Face Communication}
	\item \textit{Physical Setup Reflecting Agile Philosophy}
	\item \textit{Collocated Customers}
\end{inparaenum} have been abstracted to the OPS strategy \textit{High-Bandwidth Communication}  \cite[p. 57]{sventha_dissertation}. In the same way the OPP \textit{Automated test builds} practice has been abstracted to the OPS strategy \textit{Continuous Integration} \cite[p. 57]{sventha_dissertation}. \\

The connection between the practices and strategies is done based on the questions of each tool. The aforementioned connections are depicted with colours. When a practice has more than one colour, it is because it covers more practices from the other tool (The colours and symbols among Tables~\ref{table:opp_pam_practices}, ~\ref{table:opp_taa_practices} are randomly selected and do not imply any connection between the two tables). \\

%Analyze each the practices between them

\begin{table}
\begin{tabular}{| p{8.0cm} | p{6.8cm} |}
	\hline
	\textbf{OPP/OPS} & \textbf{PAM}  \\ \hline
     	\begin{itemize}[leftmargin=*, label=]
     		\item {\color{RoyalBlue1}Iteration Progress Tracking and Reporting} \FourStar
     		\item {\color{RoyalBlue1}Iterative} {\color{DarkMagenta}and Incremental  Development} \FourStar ~\JackStarBold
     		\item {\color{DarkOrange1}Continuous Integration} \AsteriskRoundedEnds
     		\item {\color{DarkOrange1}Software Configuration Management} \AsteriskRoundedEnds
     		\item {\color{DarkOrange1}Test Driven} {\color{red2}Development} \AsteriskRoundedEnds ~\AsteriskThin 
     		\item {\color{DarkBlue}High-Bandwidth} {\color{DeepPink1}Communication} \JackStar ~\Asterisk 
     		\item {\color{green4}Daily Progress Tracking Meetings} \EightStar
     		\item {\color{DarkBlue}Client-Driven} {\color{RoyalBlue1}Iterations} \JackStar ~\FourStar
     		\item {\color{red2}Evolutionary Requirements} \AsteriskThin
     		\item {\color{red2}Customer/User Acceptance Testing} \AsteriskThin
     		\item {\color{DarkRed}Retrospectives Meetings} \CrossMaltese
     		\item {\color{RoyalBlue1}Self-Organizing Teams} \FourStar
 		\end{itemize} 
 		& \begin{itemize}[leftmargin=*, label=]
 			\item {\color{RoyalBlue1}Iteration Planning} \FourStar
 			\item {\color{DarkMagenta}Iterative Development} \JackStarBold
 			\item {\color{DarkOrange1}Continuous Integration and Testing} \AsteriskRoundedEnds 
 			\item {\color{DeepPink1}Co-Location} \Asterisk 
 			\item {\color{green4}Stand-up Meetings} \EightStar
 			\item {\color{DarkBlue}Customer Access} \JackStar
 			\item {\color{red2}Customer Acceptance Tests} \AsteriskThin
 			\item {\color{DarkRed}Retrospectives} \CrossMaltese
 		\end{itemize}
     \\ \hline
\end{tabular}
\caption{Relation of OPP/OPS and PAM practices}
\label{table:opp_pam_practices}
\end{table}

\begin{table}
\begin{tabular}{| p{7.8cm} | p{7.0cm} |}
	\hline
	\textbf{OPP/OPS} & \textbf{TAA}  \\ \hline
     	\begin{itemize}[leftmargin=*, label=]
     		\item {\color{DeepPink1} Iterative and Incremental Development} \Asterisk 
     	    \item {\color{DeepPink1}Product Backlog} \Asterisk 
     		\item {\color{green4}Smaller and Frequent Product Releases} \EightStar
     		\item {\color{RoyalBlue1}Customer/User {\color{green4}Acceptance Testing}} \FourStar ~\EightStar
     		\item {\color{RoyalBlue1}Constant Velocity} \FourStar	
     		\item {\color{RoyalBlue1}Iteration Progress Tracking and Reporting} \FourStar
     		\item {\color{DarkRed}Self-} {\color{green4}Orga}{\color{RoyalBlue1}nizing} {\color{DarkMagenta}Teams} \CrossMaltese ~\EightStar ~\FourStar ~\JackStar 
     		\item {\color{DarkRed}Appropriate Distribution of Expertise} \CrossMaltese
     		\item {\color{DarkRed}High-Bandwidth Communication} \CrossMaltese 
     		\item {\color{DarkRed}Daily Progress Tracking Meetings} \CrossMaltese
     		\item {\color{DarkRed}Retro}{\color{RoyalBlue1}spectives} {\color{green4}Meetings}  \CrossMaltese ~\FourStar ~\EightStar 
     		\item {\color{DarkOrange1}Test Driven Development} \AsteriskRoundedEnds
     		\item {\color{DarkMagenta}Refactoring} \JackStar
     		\item {\color{DarkMagenta}Software Configuration Management} \JackStar
     		\item {\color{DarkMagenta}Adherence to Coding Standards} \JackStar
     		\item {\color{DarkMagenta}Pair Programming} \JackStar
     		\item {\color{DarkMagenta}Continuous} {\color{DarkOrange1}Integration} \JackStar ~\AsteriskRoundedEnds
     	\end{itemize} 
     	& \begin{itemize}[leftmargin=*, label=] 
     		\item {\color{DeepPink1}Product Ownership} \Asterisk 
     		\item {\color{green4}Release Planning and Tracking} \EightStar
     		\item {\color{RoyalBlue1}Iteration Planning and Tracking} \FourStar
     		\item {\color{DarkRed}Team} \CrossMaltese
     		\item {\color{DarkOrange1}Testing Practices} \AsteriskRoundedEnds
     		\item {\color{DarkMagenta}Development Practices/Infrastructure} \JackStar	
 		\end{itemize} 
     \\ \hline
\end{tabular}
\caption{Relation of OPP/OPS and TAA practices/areas}
\label{table:opp_taa_practices}
\end{table}


\section[PAM and TAA Mapping]{Mapping of questions from the PAM and TAA tools}
\label{mapping}

PAM has divided its questions based on agile practices, while on the other hand TAA has divided them based on areas considered important. In the next pages there is a mapping of the questions used from the PAM and TAA tools, with the practices from OPP and strategies from OPS. As one can see from the tables above, while all practices/areas from PAM and TAA are mapped to OPP and OPS, not all of their questions are under OPP practices or OPS strategies. This can be explained due to the different perception/angle, of the creators of the tools, of what is important for an organization to be agile.

%Discuss that although the questions may not fit with OPS, they still match the specific category.

%Fill in the missing questions for the 3rd RQ

%write about comparative agility

%Explain direct and relevant mapping 

The questions among the tools will be match based on whether they are covered directly, relevantly, or not at all. Direct match will be considered the one where a question from a tool is almost similar with one from OPS. Relevant match will be considered the one where a question of a tool does not exist in OPS, but its practice does exist in OPS. Non relevant match will be considered the one where a question cannot be matched at all in OPS. 


\section{Analysis}
By viewing Tables~\ref{table:opp_pam_practices}, ~\ref{table:opp_taa_practices} and Section~\ref{mapping} one can clearly distinguish that OPP and consequently OPS is more complete than the others in measuring agility, covering all the areas of the PAM and TAA tools. Furthermore as it can be seen in Table~\ref{table:questions_coverage} OPS covers a high percent of questions from both tools directly and relevantly. TAA has a high percent of non relevant matches mostly due to \textit{Product Ownership} and \textit{Release Planning and Tracking} perspectives which are not covered in such an extent from OPS. The first one can be explained by the fact that OPS covers basis methodologies for developing software such as XP, FDD, Crystal, Lean \cite[p. 44]{sventha_dissertation} whereas \textit{Product Ownership} refers explicitly to Scrum which is a method for managing product development \cite{koch2005agile}. On the other hand OPS considers the release cycle as smaller iteration cycles \cite[p. 13]{sventha_dissertation} which as a consequence makes the framework to set the focus on the iterations, rather than the releases. As a result the \textit{Release Planning and Tracking} is only covered to a small extent.

\begin{table} [H]
	\begin{tabular}{{| p{4cm} | p{4cm} | p{4cm} |}}
		\hline
		\multicolumn{3}{|c|}{\textbf{Questions Coverage}}  \\ \hline
		\textbf{Match}  & \textbf{PAM} & \textbf{TAA}  \\ \hline		
		Direct Match & 17/48 (35.4\%) & 25/68 (36.7\%) \\ \hline
		Relevant Match & 31/48 (64.5\%) & 24/68 (35.2\%) \\ \hline
		Non Relevant & 0/48 (0\%) & 19/68 (27.9\%) \\ \hline
	\end{tabular}
\caption{Questions Coverage from OPS}
\label{table:questions_coverage}
\end{table}


%Team size
%http://www.jamesshore.com/Agile-Book/the_xp_team.html
%http://ieeexplore.ieee.org/xpl/articleDetails.jsp?tp=&arnumber=6681108&matchBoolean%3Dtrue%26searchField%3DSearch_All%26queryText%3D%28%28Agile%29+AND+team+size%29
%http://www.academia.edu/4893042/Team_size_in_agile_development_-_A_limiting_factor
