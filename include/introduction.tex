                                                                                     \chapter{Introduction}
\label{ch:introduction}

\lettrine[lines=4, loversize=-0.1, lraise=0.1]{A}{gile} and plan-driven methodologies are the two dominant approaches in the software development. Organisations and companies tend to leave the cumbersome area of Waterfall process and to embrace the Agile methodologies in the last years \cite{laurie_williams, Wang_Conboy, Salo_Abrahamsson}. Although it has been almost 20 years since the latter were introduced, the companies are quite reluctant in following them \cite{4599456}. Once they do, they start enjoying the benefits of the agile approach, but are these the only benefits they could leverage?

In order to answer to the previous question, one should first understand what ``\textit{agile}" means. According to the dictionary \cite{cambridge_dictionary}, it means ``\textit{to be able to move quickly and easily}", something which is almost impossible with a plan-driven approach. The term agility was first introduced as agile manufacturing in an industry book \cite{agile_manufacturing} as stated by \citet{conboy_fitzgerald}.

In 2001, 17 developers formed the Agile Alliance and created the agile manifesto \cite{beck2001agile}, defining what is considered to be agile in order to avoid confusion: 
\begin{itemize}
	\item {\large \textbf{Individuals and interactions}} over processes and tools
	\item {\large \textbf{Working software}} over comprehensive documentation
	\item {\large \textbf{Customer collaboration}} over contract negotiation
	\item {\large \textbf{Responding to change}} over following a plan
\end{itemize}

Software development teams started adopting the most known agile methodologies, such as eXtreme Programming \cite{Beck:2004:EPE:1076267}, Feature Driven Development (FDD) \cite{Palmer:2001:PGF:600044}, Crystal \cite{Cockburn:2004:CCH:1406822}, Scrum \cite{scrum} and others. Most companies use a tailored methodology by following some of the aforementioned processes and practices which better suit their needs. \citet{williams2004toward} report that rarely all XP practices are exercised in their pure form, something on which \citet{Reifer} and \citet{aveling} also agree based on the results of their surveys, which showed that it is common for organizations to partially adopt XP. \citet{sidky} mention that the organizations face four issues when transitioning to agile.
\begin{inparaenum} [a\upshape)]
\item their readiness for agility
\item the practices they should adopt
\item the potential difficulties in adopting them
\item the necessary organizational preparations for the adoption of agile practices. 
\end{inparaenum}
The most important issue that tends to be neglected, though, is how well these methodologies are adopted.

According to \citet{6427226}, the agile methodologies are easier to misunderstand. Such a case could lead to problems later on in the software development process. The previous statement is also supported by \citet{cefam}, who argue that the agile software development methodologies are often applied to the wrong context. In addition, \citet{1629340} concludes that the organizations modify practices before implementing them, a fact also mentioned by \citet{1579312}. \citet{hossain} argue that improper use of agile practices creates problems. \citet{sahota} states that doing agile and being agile are two different things. For the first one, a company should follow practices, while for the other one, a company should think in an agile way. \citet{lappoA04} state that the organizations which follow the practices of a methodology may not gain much in terms of agility, while on the other hand, \citet{sidky_dissertation} defines the level of agility of a company as the amount of agile practices used. Considering this statement, a group that uses pair programming and collective code ownership at a very low level is more agile than a group which uses only pair programming but in a more efficient manner.

\citet{comparative_agility} pose the question ``\textit{How agile is agile enough}"? Practitioners think that declaring being agile is equally good as being agile. According to a survey conducted by \citet{ambysoft}, only 65\% of the agile companies that answered met the five agile criteria posed in the survey. In addition, 9\% of agile projects failed due to the lack of cultural transition, while 13\% of companies are at odds with core agile values based on the most recent survey by \citet{versionOne}. \citet{poonacha} mentioned that the different perception of agile practices when they are adopted is very worrying, since even people in the same team understand them differently, according to the result of a survey \cite{ambler}. It is evidently not only from literature but also from its application that agile is a way of thinking and working, it is a whole culture \cite{poonacha}. If we had to use one word we could state it is a way of \textit{being}. \citet{Nietzsche} said ``\textit{better know nothing than half-know many things}". In the same vein, maybe it is better not to transition to agile instead of thinking of being agile. 

Since agile methodologies become more and more popular, there is a great need for developing a tool that can measure the level of agility in the organizations that have adopted them. \citet{sidky} mentions (the success stories of companies that have adopted agile methods. However, these companies did not have a measurement tool that could tell them if they are really agile. )

Measuring agility implies measuring the agile culture of a team. Alistair Cockburn \cite{cockburn2002agile, Cockburn-poetry} and Jim Highsmith \cite{Highsmith:2002} highlight the importance of culture. However, the culture  differentiates not only from team to team, but also from person to person within it, based on the values they follow. The only common basis for the agile values is the agile manifesto\cite{beck2001agile} as stated by \citet{Ingalls}. As a result, the ``agile culture tree" has the same root, but the branches grow independent, away from one another, making it difficult to measure agility.
%maybe add the previous sentence in the discussion too.

For over a decade, researchers have been constantly coming up with models and frameworks in an effort to provide a solution. Unfortunately, the multiple tools have created a saturation in the field, resulting in being used only by the organizations that participated in the empirical studies for their creation \cite{samireh_jalali_dissertation, jalali_angelis}. As a result, the vicious circle of creating tools with no actual use holds back not only the software development companies, but the research community as well.

%Starts with M (or research process in the CM project case), describes how DC will be done, and ends with how DA will be done.
This Master's Thesis deals with three tools which claim to measure the agility of software development teams. These tools are Perceptive Agile Measurement (PAM) \cite{pam}, Team Agility Assessment (TAA) \cite{Leffingwell}, Objectives Principles Strategies (OPS) \cite{sventha_dissertation}. The first one has been validated with a large sample of subjects, the second one is used by companies and the third one covers many agile practices. Since all three tools measure agility, convergent validity should be established among them to corroborate this. The surveys from the three tools will be given to company A employees to answer. The analysis of the data will be performed by grouping the survey questions to agile practices. The correlation of these practices will be the proof for establishing the convergent validity. Moreover, questions identified as the same among the tools should have the same answers from the respondents.

%Contributions
%Probably rewrite it
This Master's Thesis is a validity study of three tools used for measuring the agility of software development teams. To the best of the author's knowledge there has not been another similar study which would be insightful and which can serve as a basis for future work. Furthermore, by having a better view of these tools, an effort was made to fill in any existing gaps in order to create an enhanced tool which will be able to better cover the needs of practitioners and researchers. 

%Outline of the paper
In order to clarify the structure of the thesis,  Chapter~\ref{ch:related_work} presents the tools that measure the agility of agile methodologies (e.g. eXtreme Programming) and the tools which measure the agility of software development organisations/teams. After that, Chapter~\ref{ch:research_methodology} presents the research questions and research methodology followed for this Master's Thesis. Chapter~\ref{ch:results} presents the results of this case study and Chapter~\ref{ch:enhancing_ops} presents the enhancement of \ac{OPS} in measuring agility in combination with \ac{PAM} and \ac{TAA}. The results of the thesis are discussed in Chapter~\ref{ch:discussion} and the conclusions and future work are presented in Chapter~\ref{ch:conclusions_future_work}.  