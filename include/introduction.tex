\chapter{Introduction}

Agile and plan-driven methodologies are the two dominant species in the software development. The eternal battle between them seems will go on for a long time before  one of them prevails. Organizations and companies tend to leave the cumbersome area of Waterfall processes and tend to embrace the Agile methodologies. Although it has been almost 20 years since the latter were introduced companies are quite reluctanct in following them. Once they do, they start enjoying the agile benefits, but are these the only benefits they could enjoy? 

In order to answer to the previous question one should first understand what does ``agile" mean? According to a dictionary, it means to be able to move quickly and easily, something which is almost impossible with a plan-driven approach. The term agility was first introduced as agile manufacturing in an industry book \cite{agile_manufacturing}.

In 2001, 17 developers were gathered to create the agile manifesto \cite{beck2001agile} defining is considered to agile in order to avoid confusion:  
\begin{itemize}
	\item {\large \textbf{Individuals and interactions}} over processes and tools
	\item {\large \textbf{Working software}} over comprehensive documentation
	\item {\large \textbf{Customer collaboration}} over contract negotiation
	\item {\large \textbf{Responding to change}} over following a plan
\end{itemize}

Software development teams started to adopt the most known agile methodologies, such as Extreme Programming, Feature Driven Development, Crystal, Scrum and others. Most companies use a tailored methodology by following some processes and practices of the aforementioned ones in order to better suit the companies. \citet{williams2004toward} reports that rarely all XP practices are exercised in their pure form. \citet{sidky} mention four issues organisations face when transitioning to agile.
\begin{inparaenum} [a\upshape)]
\item the organization's readiness for agility
\item the practices it should adopt
\item the potential difficulties in adopting them
\item the necessary organizational preparations for the adoption of agile practices. 
\end{inparaenum}
The most important issue though that tends to be neglected, is how well are these methodologies adopted? 

According to \citet{6427226} the agile methodologies are easier to misunderstand. Such a case could lead to problems later on in the software development process. \citet{sahota} says that doing agile and being agile are two different things. For the first one a company should follow practices while for the other one a company should think in an agile way. \citet{lappoA04} state that organizations following the practices of a methodology does not mean they gain much in terms of agility, while on the other hand \citet{sidky_dissertation} defines the agility of a company as the amount of agile practices used. Considering the aforementioned statement, a group that uses pair programming and collective code ownership at a very low level is better than a group which uses only pair programming but in a more efficient manner.

It is evident from literature that agile is a way of thinking and working. It is a way of \textit{being}. People think that declaring being agile is equally good as being agile. \citet{Nietzsche} said ``better know nothing than half-know many things". In the same context maybe it is better not to transition to agile instead of thinking being agile. 

Since agile methodologies become more and more popular there is a great need for development of a tool that can measure the level of agility in the organizations that have adopted them. Reasearchers for more than a decade have been constantly coming up with models and frameworks in an effort to provide a solution. Unfortunately the multiple tools have created a saturation in the field since none of them has been validated \cite{samireh_jalali_dissertation}. As a result, the vicious circle of creating tools with no actual use does not stop, holding back not only the software development companies, but the research community as well.


\section{Thesis Structure}


%what is agile
%are companies agile?
%percent of companies claiming to be agile
%find causes of agile failure
%silver bullet

%problems of the tools
%

