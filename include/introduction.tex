\chapter{Introduction}
\label{ch:introduction}

\lettrine[lines=4, loversize=-0.1, lraise=0.1]{A}{gile} and plan-driven methodologies are the two dominant approaches in the software development. The eternal battle between them will go on for a long time before one of them prevails. Organizations and companies tend to leave the cumbersome area of Waterfall process and to embrace the Agile methodologies. Although it has been almost 20 years since the latter were introduced, the companies are quite reluctant in following them. Once they do, they start enjoying the Agile's benefits, but are these the only benefits they could enjoy?

In order to answer to the previous question, one should first understand what does ``\textit{agile}" mean. According to the dictionary, it means ``\textit{to be able to move quickly and easily}", something which is almost impossible with a plan-driven approach. The term agility was first introduced as agile manufacturing in an industry book \cite{agile_manufacturing}.

In 2001, 17 developers formed the Agile Alliance and created the agile manifesto \cite{beck2001agile}, defining what is considered to be agile in order to avoid confusion: 
\begin{itemize}
	\item {\large \textbf{Individuals and interactions}} over processes and tools
	\item {\large \textbf{Working software}} over comprehensive documentation
	\item {\large \textbf{Customer collaboration}} over contract negotiation
	\item {\large \textbf{Responding to change}} over following a plan
\end{itemize}

Software development teams started adopting the most known agile methodologies, such as eXtreme Programming \cite{Beck:2004:EPE:1076267}, Feature Driven Development (FDD) \cite{Palmer:2001:PGF:600044}, Crystal \cite{Cockburn:2004:CCH:1406822}, Scrum \cite{scrum} and others. Most companies use a tailored methodology by following some of the aforementioned processes and practices which better suit their needs. \citet{williams2004toward} reports that rarely all XP practices are exercised in their pure form, something on which \citet{Reifer} and \citet{aveling} also agree based on the results of their surveys, which showed that it is common for organizations to partially adopt XP. \citet{sidky} mention that the organizations face four issues when transitioning to agile.
\begin{inparaenum} [a\upshape)]
\item the organization's readiness for agility
\item the practices it should adopt
\item the potential difficulties in adopting them
\item the necessary organizational preparations for the adoption of agile practices. 
\end{inparaenum}
The most important issue though that tends to be neglected, is how well are these methodologies adopted? 

According to \citet{6427226}, the agile methodologies are easier to misunderstand. Such a case could lead to problems later on in the software development process. The previous statement is also supported by \citet{cefam} as well, who argue that the agile software development methodologies are often applied to the wrong context. In addition, \citet{1629340} concludes that the organizations modify practices before implementing them, a fact also mentioned by \citet{1579312}. \citet{hossain} argue that improper use of agile practices creates problems. \citet{sahota} states that doing agile and being agile are two different things. For the first one a company should follow practices, while for the other one a company should think in an agile way. \citet{lappoA04} state that organizations following the practices of a methodology does not mean they gain much in terms of agility, while on the other hand, \citet{sidky_dissertation} defines the level of agility of a company as the amount of agile practices used. Considering the this statement, a group that uses pair programming and collective code ownership at a very low level is more agile than a group which uses only pair programming but in a more efficient manner.

\citet{comparative_agility} pose the question ``\textit{How agile is agile enough}"? Practitioners think that declaring being agile is equally good as being agile. According to a survey conducted by \citet{ambysoft}, only 65\% of the agile companies that answered met the five agile criteria posed in the survey. In addition, 9\% of agile projects failed due to the lack of cultural transition, while 13\% of companies are at odds with core agile values based on the most recent survey by \citet{versionOne}. \citet{poonacha} mentioned that the different perception of agile practices when they are adopted is very worrying, since even people in the same team understand them differently, according to the result of a survey \cite{ambler}. It is evident not only from literature but also from its application that agile is a way of thinking and working, it is a whole culture \cite{poonacha}. If we had to use one word we could state it is a way of \textit{being}. \citet{Nietzsche} said ``\textit{better know nothing than half-know many things}". In the same vein, maybe it is better not to transition to agile instead of thinking of being agile. 

Since agile methodologies become more and more popular, there is a great need for development of a tool that can measure the level of agility in the organizations that have adopted them. \citet{sidky} mentions the success stories of companies that have adopted agile methods, but without having a measurement tool that could tell if you are really agile. For over a decade, the researchers have been constantly coming up with models and frameworks in an effort to provide a solution. Unfortunately, the multiple tools have created a saturation in the field, resulting in being used only by the organizations that participated in the empirical studies for their creation \cite{samireh_jalali_dissertation}\cite{Jalali2014}. As a result, the vicious circle of creating tools with no actual use holds back not only the software development companies, but the research community as well.

%Starts with M (or research process in the CM project case), describes how DC will be done, and ends with how DA will be done.
This master thesis deals with three tools which claim to measure the agility of software development teams. These tools are Perceptive Agile Measurement \cite{pam}, Team Agility Assessment \cite{Leffingwell}, Objectives Principles Strategies \cite{sventha_dissertation} in Chapter~\ref{ch:case_study}. In Chapter~\ref{add chapter number} the completeness of the tools among them on measuring agility is checked by analysing if they overlap each other and how much. In addition, in Chapter~\ref{add chapter number} the correlation of the agile practices of the tools was checked to see how much they correlate with each other. For the above, a case study was performed in the industry (company F \footnote{F is the first letter of the company's name}). The team's of company F were asked to reply to the survey of each tool. As it was seen in Chapter~\ref{add chapter number} the tools measure the agile practices in different ways. As a result in Chapter~\ref{add chapter number} there is an effort of enhancing OPS by combining the three tools.

%Contributions
%Probably rewrite it
This master thesis gives to the reader the ability to see a comparative study of three tools used for measuring the agility of software development teams. Not only the weaknesses and strengths of the tools become apparent, but also how their results can actually provide a concrete view of how well the agility of a team can be measured. In addition, to the best of the author's knowledge there has not been another similar comparison which would be insightful and which can serve as a basis for future work. Moreover, the common areas are evident with the analysis of the tools completeness, while the unique ones are distinguished. Furthermore, by having a better view of these tools, an effort was made to fill in any existing gaps in order to create a more complete tool which will be able to better cover the needs of practitioners and researchers.

%Outline of the paper
In order to clarify the structure of the thesis, the Chapter~\ref{ch:related_work} presents the tools that measure the agility of agile methodologies (e.g. XP) and the tools which measure the agility of software development organisations/teams. After that, in Chapter~\ref{add chapter number} presents the case study and an analysis on how much complete are the tools in measuring agility among them. Chapter~\ref{add chapter number} presents the enhancement of OPS in measuring agility. The results of the thesis are discussed in Chapter~\ref{add chapter number} and in Chapter~\ref{} the future work is presented. 


%are companies agile?
%percent of companies claiming to be agile --- look again at the reports from VersionOne and Ambysoft
%find causes of agile failure
%silver bullet

%problems of the tools