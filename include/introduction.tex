\chapter{Introduction}
\label{ch:introduction}

\lettrine[lines=4, loversize=-0.1, lraise=0.1]{A}{gile} and plan-driven methodologies are the two dominant approaches in the software development. The eternal battle between them seems will go on for a long time before  one of them prevails. Organizations and companies have the tendency to leave the cumbersome area of Waterfall process and to embrace the Agile methodologies. Although it has been almost 20 years since the latter were introduced, companies are quite reluctanct in following them. Once they do, they start enjoying the agile benefits, but are these the only benefits they could enjoy? 

In order to answer to the previous question one should first understand what does ``\textit{agile}" mean? According to a dictionary, it means ``\textit{to be able to move quickly and easily}", something which is almost impossible with a plan-driven approach. The term agility was first introduced as agile manufacturing in an industry book \cite{agile_manufacturing}.

In 2001, 17 developers formed the Agile Alliance and created the agile manifesto \cite{beck2001agile} defining what is considered to be agile in order to avoid confusion:  
\begin{itemize}
	\item {\large \textbf{Individuals and interactions}} over processes and tools
	\item {\large \textbf{Working software}} over comprehensive documentation
	\item {\large \textbf{Customer collaboration}} over contract negotiation
	\item {\large \textbf{Responding to change}} over following a plan
\end{itemize}

Software development teams started to adopt the most known agile methodologies, such as eXtreme Programming \cite{Beck:2004:EPE:1076267}, Feature Driven Development (FDD) \cite{Palmer:2001:PGF:600044}, Crystal \cite{Cockburn:2004:CCH:1406822}, Scrum \cite{scrum} and others. Most companies use a tailored methodology by following some processes and practices of the aforementioned ones in order to better suit their needs. \citet{williams2004toward} reports that rarely all XP practices are exercised in their pure form something to which \citet{Reifer} and \citet{aveling} also agree based on the results of their surveys showing that it is common for organizations to partially adopt XP. \citet{sidky} mention four issues organisations face when transitioning to agile.
\begin{inparaenum} [a\upshape)]
\item the organization's readiness for agility
\item the practices it should adopt
\item the potential difficulties in adopting them
\item the necessary organizational preparations for the adoption of agile practices. 
\end{inparaenum}
The most important issue though that tends to be neglected, is how well are these methodologies adopted? 

According to \citet{6427226} the agile methodologies are easier to misunderstand. Such a case could lead to problems later on in the software development process. The aforementioned statement is also supported by \citet{cefam} as well who argue that the agile software development methdologies are often applied to the wrong context. \citet{1629340} does not only see eye to eye with the previously mentioned researchers but also comes to the conclusion that organizations modify practices before implementing them which is also mentioned by \citet{1579312}. \citet{hossain} argues that improper use of agile practices creates problems. \citet{sahota} states that doing agile and being agile are two different things. For the first one a company should follow practices while for the other one a company should think in an agile way. \citet{lappoA04} state that organizations following the practices of a methodology does not mean they gain much in terms of agility, while on the other hand \citet{sidky_dissertation} defines the level of agility of a company as the amount of agile practices used. Considering the aforementioned statement, a group that uses pair programming and collective code ownership at a very low level is more agile than a group which uses only pair programming but in a more efficient manner.

\citet{comparative_agility} pose the question ``\textit{How agile is agile enough}"? Practictioners think that declaring being agile is equally good as being agile. According to a survey from \citet{ambysoft} only 65\% of the agile companies that answered met the five agile criteria posed in the survey. In addition, 9\% of agile projects failed due to lack of cultural transition while 13\% of companies are at odds with core agile values based on the most recent survey by \citet{versionOne}. \citet{poonacha} mentioned that the different perception of agile practices when they are adopted is very worrying, since even people in the same team understand them differently according to the result of a survey \cite{ambler}. It is evident not only from literature but also from its application that agile is a way of thinking and working, it is a whole culture \cite{poonacha}. If we had to use one word we could state it is a way of \textit{being}. \citet{Nietzsche} said ``\textit{better know nothing than half-know many things}". In the same context maybe it is better not to transition to agile instead of thinking of being agile. 

Since agile methodologies become more and more popular there is a great need for development of a tool that can measure the level of agility in the organizations that have adopted them. \citet{sidky} mentions of success stories about companies that have adopted agile methods, but without having a measurement tool that could tell if you are really agile. Reasearchers for more than a decade have been constantly coming up with models and frameworks in an effort to provide a solution. Unfortunately the multiple tools have created a saturation in the field resulting in being used only from the organizations that participated in the empirical studies for their creation \cite{samireh_jalali_dissertation}\cite{Jalali2014}. As a result, the vicious circle of creating tools with no actual use holds back not only the software development companies, but the research community as well.


\section{Thesis Structure} %add pages
\begin{itemize}
	\item Introduction - page~\pageref{ch:introduction}
	\item Related Work - page~\pageref{ch:related_work}
	\item Case Study - page~\pageref{ch:case_study}
	\item Results
	\item Discussion
	\item Appendix - page~\pageref{ch:appendices}
\end{itemize}


%are companies agile?
%percent of companies claiming to be agile --- look again at the reports from VersionOne and Ambysoft
%find causes of agile failure
%silver bullet

%problems of the tools

