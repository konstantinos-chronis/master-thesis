\chapter{Perceptive Agile Measurement}  %48 in total
\label{ch:pam}

The items marked with \XSolidBrush  are the ones not included in the surveys to the teams.

\begin{itemize}
	\item Iteration Planning
		\begin{itemize}
			\item All members of the technical team actively participated during iteration planning meetings
			\item All technical team members took part in defining the effort estimates for requirements of the current iteration
   			\item When effort estimates differed, the technical team members discussed their underlying assumption
   			\item All concerns from team members about reaching the iteration goals were considered
   			\item The effort estimates for the iteration scope items were modified only by the technical team members
   			\item Each developer signed up for tasks on a completely voluntary basis
   			\item The customer picked the priority of the requirements in the iteration plan
		\end{itemize}
	\item Iterative Development
		\begin{itemize}
			\item We implemented our code in short iterations
			\item The team rather reduced the scope than delayed the deadline
			\item When the scope could not be implemented due to constraints, the team held active discussions on re-prioritization with the customer on what to finish within the iteration
			\item We kept the iteration deadlines
			\item At the end of an iteration, we delivered a potentially shippable product
			\item The software delivered at iteration end always met quality requirements of production code
			\item Working software was the primary measure for project progress
		\end{itemize}
	\item Continuous Integration And Testing
		\begin{itemize}
			\item The team integrated continuously
			\item Developers had the most recent version of code available
			\item Code was checked in quickly to avoid code synchronization/integration hassles
			\item The implemented code was written to pass the test case
			\item New code was written with unit tests covering its main functionality
			\item Automated unit tests sufficiently covered all critical parts of the production code
			\item For detecting bugs, test reports from automated unit tests were systematically used to capture the bugs
			\item All unit tests were run and passed when a task was finished and before checking in and integrating
			\item There were enough unit tests and automated system tests to allow developers to safely change any code
		\end{itemize}
	\removed Stand-Up Meetings
		\begin{itemize}
			\removed Stand up meetings were extremely short (max. 15 minutes)
			\removed Stand up meetings were to the point, focusing only on what had been done and needed to be done on that day
			\removed All relevant technical issues or organizational impediments came up in the stand up meetings
			\removed Stand up meetings provided the quickest way to notify other team members about problems
			\removed When people reported problems in the stand up meetings, team members offered to help instantly
		\end{itemize}
	\item Customer Access
		\begin{itemize}
			\item The customer was reachable
			\item The developers could contact the customer directly or through a customer contact person without any bureaucratical hurdles
			\item The developers had responses from the customer in a timely manner
			\item The feedback from the customer was clear and clarified his requirements or open issues to the developers
		\end{itemize}
	\item Customer Acceptance Tests
		\begin{itemize}
			\item How often did you apply customer acceptance tests?
			\item A requirement was not regarded as finished until its acceptance tests (with the customer) had passed
			\item Customer acceptance tests were used as the ultimate way to verify system functionality and customer requirements
			\item The customer provided a comprehensive set of test criteria for customer acceptance
			\item The customer focused primarily on customer acceptance tests to determine what had been accomplished at the end of an iteration
		\end{itemize}
	\removed Retrospectives
		\begin{itemize}
			\removed How often did you apply retrospectives?
			\removed All team members actively participated in gathering lessons learned in the retrospectives
			\removed The retrospectives helped us become aware of what we did well in the past iteration(s)
			\removed The retrospectives helped us become aware of what we should improve in the upcoming iteration(s)
			\removed In the retrospectives (or shortly afterwards), we systematically assigned all important points for improvement to responsible individuals
			\removed Our team followed up intensively on the progress of each improvement point elaborated in a retrospective
		\end{itemize}
	\item Co-Location
		\begin{itemize}
			\item Developers were located majorly in Greece
			\item All members of the technical team (including QA engineers, db admins) were located in Greece
			\item Requirements engineers were located with developers in Greece
			\item The project/release manager worked with the developers in Greece
			\item The customer was located with the developers in Greece
		\end{itemize}	
\end{itemize}
