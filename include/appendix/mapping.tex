\chapter{Mapping}
\label{ch:mapping}

%elaborate why some belong to different practices than strategies
%count again, there have been some changes

\section{Questions to Practices/Strategies}

\newcommand*\taa{\item[\FiveStar]}
\newcommand*\taar{\item[\FiveStarShadow]}
\newcommand*\taanr{\item[\SnowflakeChevron]}
\newcommand*\pam{\item[\AsteriskBold]}
\newcommand*\pamr{\item[\AsteriskThinCenterOpen]}
\newcommand*\ops{\item[\FiveStarOutlineHeavy]}

In order to distinguish the questions among the tools the following annotation has been used. 
\begin{itemize}
  \taa Team Agility Assessment - Direct match
  \taar Team Agility Assessment - Relevant match
  \taanr Team Agility Assessment - Non Relevant
  \pam Perceptive Agile Management - Direct match
  \pamr Perceptive Agile Management - Relevant match
\end{itemize}

\vspace{0.5cm}

\textbf{Iterative and Incremental Development}
\begin{itemize}
	\taa Stories sufficiently elaborated prior to planning meetings
	\taar Team discusses acceptance criteria during iteration planning
	\pamr We kept the iteration deadlines
	\pamr The software delivered at iteration end always met quality requirements of production code
	\pamr The team rather reduced the scope than delayed the deadline
	\pamr At the end of an iteration, we delivered a potentially shippable product
	\pamr Working software was the primary measure for project progress
	\ops To what extent are the estimates for the amount of work to be done during each iteration accurate?
	\ops To what extent are the stories fully estimated when added to the iteration backlog?
	\ops To what extent are the stories prioritized when added to the iteration backlog?	
\end{itemize}

\textbf{Product Backlog}
\begin{itemize}
	\taa Backlog prioritized and ranked by business value
	\taar Backlog estimated at gross level %removed
	\ops To what extent is a product backlog maintained?
	\ops To what extent are the features prioritized when they are added to the product backlog?
	\ops To what extent are the features fully estimated before they are added to the product backlog?
\end{itemize}

\textbf{Smaller and Frequent Product Releases}
\begin{itemize}
	\taa The team has small and frequent releases
	\pam We implemented our code in short iterations
	\ops To what extent is software released frequently? (length of a release cycle is one year or less)
	\ops To what extent is software released frequently? (length of an iteration is four weeks or less)
	\ops To what extent is the product developed so far in-sync with the customers' needs and expectations? 
	\ops To what extent are the deployments not rolled back? 
\end{itemize}

\textbf{Customer/User Acceptance Testing}
\begin{itemize}
	\pamr The customer provided a comprehensive set of test criteria for customer acceptance %(Test First Development)
	\pamr How often did you apply customer acceptance tests?
	\pamr The customer focused primarily on customer acceptance tests to determine what had been accomplished at the end of an iteration
	\pamr Customer acceptance tests were used as the ultimate way to verify system functionality and customer requirements
	\pamr A requirement was not regarded as finished until its acceptance tests (with the customer) had passed
\end{itemize}

\textbf{Constant Velocity}
\begin{itemize}
	\taar Team works at a sustainable pace
\end{itemize}

\textbf{Iteration Progress Tracking and Reporting}
\begin{itemize}
	\taa Iteration backlog ranked by priority
	\taa Iteration planning meeting attended and effective
	\taa Iterations are no more than four weeks in length
	\taa Iterations are of a consistent fixed length
	\taar Iteration backlog defined
	%\taar Team develops and manages iteration backlog
	\taar Iteration theme established and communicated %removed
	\taar Team velocity measured and used for planning
	\taar Release theme established and communicated 
	\taar Release planning meeting attended and effective 
	\taar Release backlog defined 
	\taar Release backlog ranked by priority 
	\taar Release backlog estimated at plan level 
	\taar Release progress tracked by feature acceptance
	\taar The whole team is present at release planning meetings
	\taar Iteration progress tracked by task to do (burn-down chart) and card acceptance (velocity)
	\pamr All members of the technical team actively participated during iteration planning meetings
	\pamr All technical team members took part in defining the effort estimates for 
requirements of the current iteration
	\pamr All concerns from team members about reaching the iteration goals were considered
	\pamr When effort estimates differed, the technical team members discussed their underlying assumption %removed
	\pamr The effort estimates for the iteration scope items were modified only by the  technical team members %removed
	\ops To what extent is an iteration backlog maintained?
	\ops To what extent is the length of an iteration 4 weeks or less?
	\ops To what extent are the iterations timeboxed?
	\ops To what extent are the release cycles timeboxed? 
	\ops To what extent are only a subset of the identified features developed during a release cycle? 
\end{itemize}

\textbf{Self-Organizing Teams}
\begin{itemize}
	\taa Team defines, estimates, and selects their own work (stories and tasks)
	\taar Team leads communication; communication not managed
	\taar Team develops and manages iteration backlog
	\taar Team manages interdependencies and constraints %removed
	\taar Team has administrative access to their own workstations
	\taar Team has administrative control over their development environment
	\taar Team is 100\% dedicated to the release (no time-slicing)
	\taar Team meets its commitments to release
	\taar The team has a common language and metaphor to describe the release
	\taar Team self-polices and reinforces use of agile practices and rules
	\taar Team is smaller than 15 people
	\pam Each developer signed up for tasks on a completely voluntary basis
	\ops To what extent do the team members determine the amount of work to be done? 
	\ops To what extent do the team members take ownership of work items?
	\ops To what extent do the team members hold each other accountable for the work to be completed?
	\ops To what extent do the team members ensure that they complete the work that they are accountable for?
	\ops To what extent do the team members determine, plan, and manage their day-to-day activities under reduced or no supervision from the management?
	\ops To what extent do the developers form ad-hoc groups to determine and refine requirements just-in-time?
	\ops To what extent does the management support the self-managing nature of the teams?
\end{itemize}

\textbf{Appropriate Distribution of Expertise}
\begin{itemize}
	\taa Team members complete commitments
	\taar Team is cross-functional with integrated product owner, development, documentation and QA
	\ops To what extent do the team members have the requisite expertise to complete the tasks assigned to them?
	\ops To what extent is the work assigned to the team members commensurate with their expertise?
	\ops To what extent does the team effectively complete the work that they have committed to? 
	\ops To what extent do the teams have members in leadership positions that can guide the others?
	\ops To what extent do the teams not rely on knowledge external to their teams? 
\end{itemize}

\textbf{High-Bandwidth Communication}
\begin{itemize}
	\taa Release planning meeting attended and effective
	\taa Team works in a physical environment that fosters collaboration
	\taar Team is collocated
	\taar The team has an effective channel for obstacle escalation
	\pamr The customer was reachable
	\pamr The developers could contact the customer directly or through a customer contact person without any bureaucratical hurdles

	\pamr Developers were located majorly in \ldots
	\pamr All members of the technical team (including QA engineers, db admins) were located in \ldots
	\pamr Requirements engineers were located with developers in \ldots
	\pamr The project/release manager worked with the developers in \ldots
	\pamr The customer was located with the developers in \ldots
	\pamr The developers had responses from the customer in a timely manner
	\ops To what extent is the product developed so far in-sync with the customers' needs and expectations?
	\ops To what extent is the time allocated for the release planning meetings utilized effectively?
	\ops To what extent is the time allocated for the iteration planning meetings utilized effectively?
	\ops To what extent is the time allocated for the retrospective meetings utilized effectively?
	\ops To what extent do the scheduled meetings (except the daily progress tracking meetings) begin and end on time?
	\ops To what extent do the meetings (except the daily progress tracking meetings) take place as scheduled? 
	\ops To what extent does open communication prevail between the business and the development team? 
	\ops To what extent does open communication prevail between the manager and the developers and testers? 
	\ops To what extent does open communication prevail between the developers and the testers? 
	\ops To what extent does open communication prevail among the developers? 
	\ops To what extent does open communication prevail between the external customer/user and the business? 
	\ops To what extent does open communication prevail between the external customer/user and the development team?
	\ops To what extent does open communication prevail between members of different teams? 
\end{itemize}

\textbf{Daily Progress Tracking Meetings}
\begin{itemize}
	\taar Daily stand up on time, fully attended and effectively communicates
	\pamr Stand up meetings were to the point, focusing only on what had been done and needed to be done on that day
	\pamr Stand up meetings were extremely short (max. 15 minutes)
	\pamr All relevant technical issues or organizational impediments came up in the stand up meetings
	\pamr Stand up meetings provided the quickest way to notify other team members about problems
	\pamr When people reported problems in the stand up meetings, team members offered to help instantly
	\ops To what extent is the time allocated for the daily progress tracking meetings utilized effectively?
\end{itemize}

\textbf{Retrospective Meetings}
\begin{itemize}
	\taar Iteration review meeting attended and effective
	\taar Release review meeting attended and effective
	\taar Team inspects and adapts (continuous improvement) the overall process
	\taar Team inspects and adapts (continuous improvement) the Iteration Plan
	\taar Team inspects and adapts (continuous improvement) the release plan
	\pamr How often did you apply retrospectives? %removed
	\pamr The retrospectives helped us become aware of what we did well in the past iteration(s)
	\pamr The retrospectives helped us become aware of what we should improve in the upcoming iteration(s)
	\pamr All team members actively participated in gathering lessons learned in the retrospectives
	\pamr In the retrospectives (or shortly afterwards), we systematically assigned all important points for improvement to responsible individuals
	\pamr Our team followed up intensively on the progress of each improvement point elaborated in a retrospective
	\ops To what extent were practices that worked well during the iteration or the release cycle and hence should be used in the future identified?
	\ops To what extent were practices that did not yield the expected results and hence should be discontinued identified?
	\ops To what extent were new practices that may better suit the team's needs identified?
	\ops To what extent were the retrospective goals set during the previous iteration met? 
\end{itemize}

\textbf{Test Driven Development}
\begin{itemize}
	\taa Unit tests written before development
	\taa 100\% automated unit test coverage
	\taar Acceptance tests written before development
	\pamr New code was written with unit tests covering its main functionality
	\pamr The implemented code was written to pass the test case
	\pamr Automated unit tests sufficiently covered all critical parts of the production code
	\ops To what extent did the developers provide adequate code coverage from the tests?
	\ops To what extent is the product developed so far in-sync with the customers' needs and expectations?
	\ops To what extent do developers write tests first before writing code?
	\ops To what extent are the test plans created before the developers start coding? 
\end{itemize}

\textbf{Refactoring}
\begin{itemize}
	\taa Refactoring is continuous
	\taar Team is permitted to refactor anywhere in the code base
	\pamr There were enough unit tests and automated system tests to allow developers to safely change any code
	\ops To what extent do the teams manage technical debt? 
	\ops To what extent do the teams minimize technical debt when developing new systems? 
	\ops To what extent does the system and the development environment allow Technical Debt to be minimized? 
	\ops To what extent does the management support the implementation of refactoring? 
\end{itemize}

\textbf{Software Configuration Management}
\begin{itemize}
	\taa Source control system exists
	\ops To what extent do teams use appropriate tools for version control and management?
\end{itemize}

\textbf{Adherence to Standards} %Change in tables and text
\begin{itemize}
	\taar Coding standards exist and applied
	\taar Pair programming is practised %this was moved from Pair Programming
	\taar Adequate and effective code review practices
	\ops To what extent do the team members agree with the set coding standards? 
	\ops To what extent do the team members adhere to the set coding standards?
\end{itemize}

\textbf{Continuous Integration}
\begin{itemize}
	\taa Automated acceptance tests
	\taar Developers integrate code multiple times per day
	\taar Stories accepted and demonstrated on integrated build
	\taar Continuous build with 100\% successful builds
	\taar Identical builds for developers' workstations %removed
	\pam Developers had the most recent version of code available
	\pamr Code was checked in quickly to avoid code synchronization/integration hassles
	\pamr The team integrated continuously
	\pamr All unit tests were run and passed when a task was finished and before checking in and integrating
	\pamr For detecting bugs, test reports from automated unit tests were systematically used to capture the bugs
	\ops To what extent has each story been coded? 
	\ops To what extent has each story been unit tested? 
	\ops To what extent has each story been refactored? 
	\ops To what extent has each story been checked into the code base? 
	\ops To what extent has each story been integrated with the existing code base? 
	\ops To what extent has each story been reviewed?
	\ops To what extent are automated test suites developed? 
	\ops To what extent are the code bases not shared? 
	\ops To what extent do automated builds run one or more times everyday?
	\ops To what extent has each story been accepted by the customer? %not sure
\end{itemize}

\textbf{Client-Driven Iterations}
\begin{itemize}
	\pam The customer picked the priority of the requirements in the iteration plan
	\pamr When the scope could not be implemented due to constraints, the team held active discussions on re-prioritization with the customer on what to finish within the iteration 
	\ops To what extent do the customers establish the priorities of the story?
	\ops To what extent is the product developed so far in-sync with the customers' needs and expectations?
	\ops To what extent are the changes requested by the customers accommodated?
\end{itemize}

\textbf{Minimal or Just Enough Documentation}
\begin{itemize}
	\ops To what extent is minimal documentation supported by teams? 
	\ops To what extent is minimal documentation created/developed? 
	\ops To what extent is minimal documentation recorded/archived? 
	\ops To what extent is minimal documentation maintained? 
\end{itemize}

\textbf{Continuous Feedback}
\begin{itemize}
	\pamr The feedback from the customer was clear and clarified his requirements or open issues to the developers % change it in more completed tool
	\ops To what extent do the customers provide feedback to the business and the development team? 
	\ops To what extent is the product developed so far in-sync with the customers' needs and expectations?
\end{itemize}

\textbf{Evolutionary Requirements}
\begin{itemize}
	\ops To what extent are the features reprioritized as and when new features are identified?
	\ops To what extent are the changes requested by the customers accommodated? 
	%taken from Minimal BRUF and BDUF
	\ops To what extent are only the high level features identified upfront?
	\ops To what extent are the architecture requirements allowed to evolve over time?
\end{itemize}

\section{Non Relevant Questions}

\begin{itemize}
	\taanr Product owner defines acceptance criteria for stories 
	\taanr Product owner and stakeholders participate at iteration and release planning 
	\taanr Product owner and stakeholders participate at iteration and release review 
	\taanr Product owner collaboration with team is continuous  
	\taanr Team completes and product owner accepts the release by the release date	 
	\taanr Work is not added by the product owner during the iteration 
	\taanr Team completes and product owner accepts the iteration  	 
	\taanr All testing is done within the iteration and does not lag behind 
	\taanr Iteration defects are fixed within that iteration 
	\taanr Team Coach/Scrum Master exists, is full-time, and is effective
\end{itemize}
