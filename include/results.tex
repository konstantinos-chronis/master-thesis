\chapter{Results}
\label{ch:results}

%Express that no correlation means no convergent validity

\lettrine[lines=4, loversize=-0.1, lraise=0.1]{T}{his} chapter presents the outcomes of the case study conducted in Company $A$. We present the results of the monotrait-heteromethod correlations which deal with establishing convergent validity, along with the results of the direct match questions whose aim is to check if the respondents gave the same answers.

As it was seen in the previous chapter, only eight out of 42 plots were monotonic. The monotrait-heteromethod correlations are presented in the following pages. In Table~\ref{table:correlations_frequency}, one can see that half of the correlations are between \ac{PAM} and \ac{OPS}. In addition, we can clearly see in Table~\ref{table:hbc_correlations} that we have a negative correlation for eight monotonic plots. Moreover, the tables ~\ref{table:cf_correlations}, ~\ref{table:cdi_correlations}, ~\ref{table:hbc_correlations}, ~\ref{table:ref_correlations}, ~\ref{table:ci_correlations} and ~\ref{table:iid_correlations} allow us to see a more interesting result than the correlations. This is the non-existence of monotonicity in the other 34 relationships, which leads us to the conclusion that there is little convergence among the tools. This is surprising because tools claiming to measure the same thing should converge.  

\section{Correlation Results}

\begin{table} [H]
 \RawFloats %allows to have captions in all of the tables
 \begin{minipage}{.45\textwidth}
  \caption{Continuous Feedback Correlations}
  \label{table:cf_correlations}
   \begin{tabular}{| c | c | c | c |} \hline
   \multicolumn{4}{|c|}{\textbf{Continuous Feedback}}  \\ \hline
   & \ac{PAM} & \ac{TAA} & \ac{OPS} \\ \hline
   \ac{PAM} & 1.000 & NA & 0.459 \\ \hline
   \ac{TAA} & NA & 1.000 & NA \\ \hline
   \ac{OPS} & 0.459 & NA & 1.000 \\ \hline
  \end{tabular}
 \end{minipage}%
%
 \begin{minipage}{.45\textwidth}
  \centering
  \caption{Client Driven Iterations Correlations}
  \label{table:cdi_correlations}
  \begin{tabular}{| c | c | c | c |} \hline
  \multicolumn{4}{|c|}{\textbf{Client Driven Iterations}}  \\ \hline
   & \ac{PAM} & \ac{TAA} & \ac{OPS} \\ \hline
  \ac{PAM} & 1.000 & NA & 0.161 \\ \hline
  \ac{TAA} & NA & 1.000 & NA \\ \hline
  \ac{OPS} & 0.161 & NA & 1.000 \\ \hline
 \end{tabular}
 \end{minipage}%
 %
\end{table}


\begin{table} [H]
 \RawFloats %allows to have captions in all of the tables
 \begin{minipage}{.45\textwidth}
  \caption{High Bandwidth Communication Correlations}
  \label{table:hbc_correlations}
  \begin{tabular}{| c | c | c | c |} \hline
  \multicolumn{4}{|c|}{\textbf{High Bandwidth Communication}}  \\ \hline
  & \ac{PAM} & \ac{TAA} & \ac{OPS} \\ \hline
  \ac{PAM} & 1.000 & 0.322 & -0.023 \\ \hline
  \ac{TAA} & 0.322 & 1.000 & 0.237 \\ \hline
  \ac{OPS} & -0.023 & 0.237 & 1.000 \\ \hline
 \end{tabular}
 \end{minipage}%
%
 \begin{minipage}{.45\textwidth}
  \centering
  \caption{Refactoring Correlations}
  \label{table:ref_correlations}
  \begin{tabular}{| c | c | c | c |} \hline
  \multicolumn{4}{|c|}{\textbf{Refactoring}}  \\ \hline
   & \ac{PAM} & \ac{TAA} & \ac{OPS} \\ \hline
   \ac{PAM} & 1.000 & 0.097 & -0.050 \\ \hline
   \ac{TAA} & 0.097 & 1.000 & 0.181 \\ \hline
   \ac{OPS} & -0.050 & 0.181 & 1.000 \\ \hline
  \end{tabular}  
 \end{minipage}%
 %
\end{table}

\begin{table} [H]
 \RawFloats %allows to have captions in all of the tables
 \begin{minipage}{.45\textwidth}
  \caption{Continuous Integration Correlations}
  \label{table:ci_correlations}
  \begin{tabular}{| c | c | c | c | } \hline
  \multicolumn{4}{|c|}{\textbf{Continuous Integration}}  \\ \hline
   & \ac{PAM} & \ac{TAA} & \ac{OPS} \\ \hline
  \ac{PAM} & 1.000 & 0.398 & 0.249 \\ \hline
  \ac{TAA} & 0.398 & 1.000 & 0.115 \\ \hline
  \ac{OPS} & 0.249 & 0.115 & 1.000 \\ \hline
 \end{tabular}
 \end{minipage}%
%
 \begin{minipage}{.45\textwidth}
  \centering
   \caption{Iterative and Incremental Development Correlations}
  \label{table:iid_correlations}
  \begin{tabular}{| c | c | c | c |} \hline
  \multicolumn{4}{|c|}{\textbf{Iterative and Incremental Development}}  \\ \hline
  & \ac{PAM} & \ac{TAA} & \ac{OPS} \\ \hline
  \ac{PAM} & 1.000 & 0.204 & 0.396 \\ \hline
  \ac{TAA} & 0.204 & 1.000 & -0.228 \\ \hline
  \ac{OPS} & 0.396 & -0.228 & 1.000 \\ \hline
 \end{tabular}
 \end{minipage}%
 %
\end{table}

\begin{table} [H]
	\caption{Frequency of correlation between tools}
	\label{table:correlations_frequency}
	\begin{tabular}{| c | c |} \hline
		\multicolumn{2}{|c|}{\textbf{Frequency}}  \\ \hline
		\ac{PAM}-\ac{OPS} & 4 \\ \hline
		\ac{PAM}-\ac{TAA} & 3 \\ \hline
		\ac{TAA}-\ac{OPS} & 1 \\ \hline
	\end{tabular}
\end{table}


In Table~\ref{table:descriptive_statistics}, one can see the descriptive statistics of the data gathered.

	\begin{longtable}{| p{.13\textwidth} | p{.11\textwidth} | p{.06\textwidth} | p{.06\textwidth} | p{.06\textwidth} | p{.13\textwidth} | p{.11\textwidth} | p{.06\textwidth} | p{.06\textwidth} | p{.06\textwidth} |} \caption{Surveys Descriptive Statistics} \\ \hline 
		\label{table:descriptive_statistics}
		\textbf{Practice} & \textbf{Statistics} & \textbf{\ac{PAM}} & \textbf{\ac{TAA}} & \textbf{\ac{OPS}} &
		\textbf{Practice} & \textbf{Statistics} & \textbf{\ac{PAM}} & \textbf{\ac{TAA}} & \textbf{\ac{OPS}} \\ \hline
		\endhead %repeats the header to the next page
		Adherence to Standards & \begin{tabular}{c} Mean \\ Sd \\ Median \\ Min \\ Max \end{tabular} & 
		\begin{tabular}{c} 1.00 \\ 0.00 \\ 1 \\ 1 \\ 1 \end{tabular} & 
		\begin{tabular}{c} 11.67 \\ 2.17 \\ 12 \\ 7 \\ 14 \end{tabular} & 
		\begin{tabular}{c} 8.10 \\ 2.12 \\ 8 \\ 6 \\ 12 \end{tabular} &	
		Appropriate Distribution of Expertise & \begin{tabular}{c} Mean \\ Sd \\ Median \\ Min \\ Max \end{tabular} &
		\begin{tabular}{c} 1.00 \\ 0.00 \\ 1.0 \\ 1 \\ 1 \end{tabular} & 
		\begin{tabular}{c} 11.13 \\ 2.10 \\ 11.5 \\ 6 \\ 14 \end{tabular} & 
		\begin{tabular}{c} 27.20 \\ 3.51 \\ 27.0 \\ 21 \\ 35 \end{tabular} \\ \hline		
		Client-Driven Iterations & \begin{tabular}{c} Mean \\ Sd \\ Median \\ Min \\ Max \end{tabular} &
		\begin{tabular}{c} 8.63 \\ 3.20 \\ 8.5 \\ 3 \\ 14 \end{tabular} & 
		\begin{tabular}{c} 1.00 \\ 0.00 \\ 1.0 \\ 1 \\ 1 \end{tabular} & 
		\begin{tabular}{c} 13.87 \\ 2.78 \\ 14.0 \\ 9 \\ 21 \end{tabular} &	
		Continuous Feedback & \begin{tabular}{c} Mean \\ Sd \\ Median \\ Min \\ Max \end{tabular} &
		\begin{tabular}{c} 4.87 \\ 1.25 \\ 5.0 \\ 2 \\ 7 \end{tabular} & 
		\begin{tabular}{c} 1.00 \\ 0.00 \\ 1.0 \\ 1 \\ 1 \end{tabular} & 
		\begin{tabular}{c} 9.20 \\ 1.88 \\ 9.5 \\ 5 \\ 14 \end{tabular} \\ \hline		
		Continuous Integration & \begin{tabular}{c} Mean \\ Sd \\ Median \\ Min \\ Max \end{tabular} &
		\begin{tabular}{c} 21.97 \\ 4.40 \\ 21.0 \\ 11 \\ 31 \end{tabular} & 
		\begin{tabular}{c} 24.13 \\ 3.82 \\ 24.5 \\ 16 \\ 31 \end{tabular} & 
		\begin{tabular}{c} 48.10 \\ 4.23 \\ 48.5 \\ 40 \\ 56 \end{tabular} &	
		High-Bandwidth Communication & \begin{tabular}{c} Mean \\ Sd \\ Median \\ Min \\ Max \end{tabular} &
		\begin{tabular}{c} 36.73 \\ 4.11 \\ 38 \\ 29 \\ 42 \end{tabular} & 
		\begin{tabular}{c} 22.87 \\ 3.25 \\ 23 \\ 13 \\ 28 \end{tabular} & 
		\begin{tabular}{c} 60.30 \\ 5.69 \\ 60 \\ 51 \\ 75 \end{tabular} \\ \hline		
		Iteration Progress Tracking and Reporting & \begin{tabular}{c} Mean \\ Sd \\ Median \\ Min \\ Max \end{tabular} &
		\begin{tabular}{c} 21.67 \\ 6.42 \\ 22.5 \\ 8 \\ 35 \end{tabular} & 
		\begin{tabular}{c} 71.73 \\ 15.62 \\ 72.5 \\ 40 \\ 100 \end{tabular} & 
		\begin{tabular}{c} 31.73 \\ 1.55 \\ 32.0 \\ 27 \\ 35 \end{tabular} &	
		Iterative and Incremental Development & \begin{tabular}{c} Mean \\ Sd \\ Median \\ Min \\ Max \end{tabular} &
		\begin{tabular}{c} 27.10 \\ 2.71 \\ 27.0 \\ 22 \\ 34 \end{tabular} &
		\begin{tabular}{c} 8.43 \\ 2.11 \\ 8.5 \\ 4 \\ 13 \end{tabular} &
		\begin{tabular}{c} 14.47 \\ 2.13 \\ 15.0 \\ 11 \\ 18 \end{tabular} \\ \hline		
		Product Backlog & \begin{tabular}{c} Mean \\ Sd \\ Median \\ Min \\ Max \end{tabular} &
		\begin{tabular}{c} 1.00 \\ 0.00 \\ 1.0 \\ 1 \\ 1 \end{tabular} &
		\begin{tabular}{c} 4.97 \\ 0.85 \\ 5.0 \\ 3 \\ 6 \end{tabular} &
		\begin{tabular}{c} 15.80 \\ 2.14 \\ 15.5 \\ 12 \\ 19 \end{tabular}  &
		Refactoring & \begin{tabular}{c} Mean \\ Sd \\ Median \\ Min \\ Max \end{tabular} &
		\begin{tabular}{c} 2.03 \\ 0.85 \\ 2.0 \\ 1 \\ 4 \end{tabular} &
		\begin{tabular}{c} 10.80 \\ 2.27 \\ 11.0 \\ 6 \\ 14 \end{tabular} &
		\begin{tabular}{c} 20.67 \\ 3.66 \\ 20.5 \\ 14 \\ 28 \end{tabular}  \\ \hline
		Self-Organizing Teams & \begin{tabular}{c} Mean \\ Sd \\ Median \\ Min \\ Max \end{tabular} &
		\begin{tabular}{c} 3.6 \\ 1.19 \\ 3.5 \\ 2 \\ 6 \end{tabular} &
		\begin{tabular}{c} 62.9 \\ 6.57 \\ 63.0 \\ 48 \\ 75 \end{tabular} &
		\begin{tabular}{c} 36.5 \\ 5.20 \\ 37.0 \\ 26 \\ 45 \end{tabular}  &
		Smaller and Frequent Product Releases & \begin{tabular}{c} Mean \\ Sd \\ Median \\ Min \\ Max \end{tabular} &
		\begin{tabular}{c} 5.6 \\ 1.19 \\ 6 \\ 2 \\ 7 \end{tabular} &
		\begin{tabular}{c} 5.8 \\ 0.81 \\ 6 \\ 4 \\ 7 \end{tabular} &
		\begin{tabular}{c} 24.8 \\ 1.24 \\ 25 \\ 22 \\ 28 \end{tabular} \\ \hline		
		Software Configuration Management & \begin{tabular}{c} Mean \\ Sd \\ Median \\ Min \\ Max \end{tabular} &
		\begin{tabular}{c} 1 \\ 0 \\ 1 \\ 1 \\ 1 \end{tabular} &
		\begin{tabular}{c} 7 \\ 0 \\ 7 \\ 7 \\ 7 \end{tabular} &
		\begin{tabular}{c} 7 \\ 0 \\ 7 \\ 7 \\ 7 \end{tabular} &
		Test Driven Development & \begin{tabular}{c} Mean \\ Sd \\ Median \\ Min \\ Max \end{tabular} &
		\begin{tabular}{c} 10.90 \\ 2.90 \\ 10.5 \\ 6 \\ 17 \end{tabular} &
		\begin{tabular}{c} 6.57 \\ 3.28 \\ 6.0 \\ 3 \\ 15 \end{tabular} &
		\begin{tabular}{c} 9.10 \\ 1.97 \\ 9.0 \\ 6 \\ 13 \end{tabular} \\ \hline
		Minimal or Just Enough Documentation & \begin{tabular}{c} Mean \\ Sd \\ Median \\ Min \\ Max \end{tabular} &
		\begin{tabular}{c} 1.0 \\ 0.00 \\ 1 \\ 1 \\ 1 \end{tabular} &
		\begin{tabular}{c} 1.0 \\ 0.00 \\ 1 \\ 1 \\ 1 \end{tabular} &
		\begin{tabular}{c} 17.8 \\ 3.16 \\ 18 \\ 10 \\ 23 \end{tabular} &	
		Customer User Acceptance Testing & \begin{tabular}{c} Mean \\ Sd \\ Median \\ Min \\ Max \end{tabular} &
		\begin{tabular}{c} 17.37 \\ 7.04 \\ 17.5 \\ 5 \\ 33 \end{tabular} &
		\begin{tabular}{c} 1.00 \\ 0.00 \\ 1.0 \\ 1 \\ 1 \end{tabular} &
		\begin{tabular}{c} 1.00 \\ 0.00 \\ 1.0 \\ 1 \\ 1 \end{tabular} \\ \hline
		Evolutionary Requirements & \begin{tabular}{c} Mean \\ Sd \\ Median \\ Min \\ Max \end{tabular} &
		\begin{tabular}{c} 1.00 \\ 0.00 \\ 1 \\ 1 \\ 1 \end{tabular} &
		\begin{tabular}{c} 1.00 \\ 0.00 \\ 1 \\ 1 \\ 1 \end{tabular} &
		\begin{tabular}{c} 20.13 \\ 2.21 \\ 20 \\ 17 \\ 25 \end{tabular} &
		Constant Velocity & \begin{tabular}{c} Mean \\ Sd \\ Median \\ Min \\ Max \end{tabular} &
		\begin{tabular}{c} 1.00 \\ 0.00 \\ 1 \\ 1 \\ 1 \end{tabular} &
		\begin{tabular}{c} 5.93 \\ 1.01 \\ 6 \\ 4 \\ 7 \end{tabular} &
		\begin{tabular}{c} 1.00 \\ 0.00 \\ 1 \\ 1 \\ 1 \end{tabular} \\ \hline
\end{longtable}

In \textit{Continuous Feedback}, \ac{PAM} and \ac{OPS} have a moderate positive correlation of $\rho$ = 0.459. Both tools focus on getting feedback from the customer, while \ac{OPS} also checks whether the product is developed according to the customer's needs and expectations.

In \textit{Client-Driven Iterations}, \ac{PAM} and \ac{OPS} have a low positive correlation of $\rho$ = 0.161. Both tools check for the possibility of the requirements having been prioritized by the customer, while \ac{OPS} additionally focuses on the customers' requests and needs.

In \textit{Continuous Integration}, \ac{PAM} and \ac{OPS} have a low positive correlation of $\rho$ = 0.249. The common areas are continuous builds, multiple submits and story acceptance. There is a small difference regarding whether the developers should sync to the latest available code that is supported by the \ac{PAM}.

In \textit{Iterative and Incremental Development}, \ac{PAM} and \ac{OPS} have a low positive correlation of $\rho$ = 0.396. The \ac{OPS} focuses on the stories estimation and prioritization, while \ac{PAM} on the deadlines that have to be meet and on the software progress. 

In \textit{High Bandwidth Communication}, \ac{PAM} and \ac{TAA} have a low positive correlation of $\rho$ = 0.322. Both of them check for the team collocation, while \ac{TAA} also checks for the communication with the customers. \ac{PAM} and \ac{OPS} surprisingly have a correlation of $\rho$ = -0.023, which means that there is no correlation at all. They both focus on the communication, but \ac{OPS} does that to a huge extent, leading to this result. In addition, \ac{OPS} checks for effectively using the time for meetings. \ac{TAA} and \ac{OPS} have a positive correlation of $\rho$ = 0.237. It is worth mentioning that this is the only practice for which correlation can be calculated across the tools.

In \textit{Refactoring}, \ac{PAM} and \ac{TAA} have a correlation of $\rho$ = 0.097, which means there is almost no correlation at all. \ac{TAA} focuses on continuous refactoring, while on the other hand, \ac{PAM} focuses on the unit testing for refactoring.

\section[Direct Match Results]{Direct Match Questions Results}
\label{sec:direct_match_results}

The groups of direct match questions showed some unexpectedly amazing results. One would expect that questions which are considered to be the same would yield the same results. On the contrary, this did not happen for any of the question groups, apart from group \hyperref[G13]{G13}. The heatmaps (see Appendix~\ref{ch:heatmaps}) which were generated by the answers made it crystal clear that the respondents' answers resulted in different scores. 

\begin{table} [H]
	\begin{tabular}{| p{1cm} | p{2cm} | p{1cm} | p{2cm} | p{1cm} | p{2cm} |} \hline
		Group & Frequency & Group & Frequency & Group & Frequency \\ \hline
		\hyperref[G1]{G1} & 12 & \hyperref[G2]{G2} & 9 & \hyperref[G3]{G3} & 7 \\ \hline
		\hyperref[G4]{G4} & 8 & \hyperref[G5]{G5} & 12 & \hyperref[G6]{G6} & 16 \\ \hline
		\hyperref[G7]{G7} & 13 & \hyperref[G8]{G8} & 12 & \hyperref[G9]{G9} & 10 \\ \hline
		\hyperref[G10]{G10} & 13 & \hyperref[G11]{G11} & 19 & \hyperref[G12]{G12} & 16 \\ \hline
		\hyperref[G13]{G13} & 30 & \hyperref[G14]{G14} & 18 & \hyperref[G15]{G15} & 13 \\ \hline
		\hyperref[G16]{G16} & 12 & \hyperref[G17]{G17} & 6 & & \\ \hline
	\end{tabular}
	\caption{Frequency of Same Answers}
	\label{table:answers_frequency}
\end{table}

Table~\ref{table:answers_frequency} displays the group of questions and the frequency of the same answers given by the respondents. As it can be seen, \hyperref[G13]{G13} is the only group of questions to which all the respondents gave the exact same answer (the maximum is 30). \hyperref[G13]{G13} is about the existence of software control management and Company $A$ uses version control management software for every single line of code written. On the other hand, \hyperref[G17]{G17}, which is about backlog prioritization, had the lowest score with only six respondents giving the same answer. The maximum difference in answers was up to two likert-scale points. For a better view of the results, one can see the heatmaps in Appendix~\ref{ch:heatmaps}.

As far as the results from the ``Mann–Whitney U test" and ``Kruskal–Wallis one-way analysis of variance" are concerned, the p-values are presented in Table~\ref{table:direct_match_answers_pvalues}. As it can be seen, the p-values from the majority of the groups are more than the alpha level of 0.05. As a result, we cannot reject the $H_0$ hypothesis (\textit{There is no difference between the groups of the same questions}).
On the other hand, the p-value of group \hyperref[G13]{G13} cannot be computed, since all the answers are the same, while for the groups \hyperref[G6]{G6} and \hyperref[G16]{G16} the p-value is below the alpha level which means that the $H_0$ hypothesis can be rejected. Nevertheless, as it is was presented in Table~\ref{table:answers_frequency}, many of the respondents gave different answers to the same question, thus, we consider the results of the ``Mann–Whitney U test" and the ``Kruskal–Wallis one-way analysis of variance" as non-significant.

\begin{table} [H]
	\begin{tabular}{| p{1cm} | p{2cm} | p{1cm} | p{2cm} | p{1cm} | p{2cm} |} \hline
		Group & p-value & Group & p-value & Group & p-value \\ \hline
		\hyperref[G1]{G1} & 0.5271 & \hyperref[G2]{G2} & 0.2404 & \hyperref[G3]{G3} & 0.3837 \\ \hline
		\hyperref[G4]{G4} & 0.6715 & \hyperref[G5]{G5} & 0.503 & \hyperref[G6]{G6} & 0.01523 \\ \hline
		\hyperref[G7]{G7} & 0.1654 & \hyperref[G8]{G8} & 0.2984 & \hyperref[G9]{G9} & 0.1865 \\ \hline
		\hyperref[G10]{G10} & 0.6893 & \hyperref[G11]{G11} & 0.3246 & \hyperref[G12]{G12} & 0.2246 \\ \hline
		\hyperref[G13]{G13} & NA & \hyperref[G14]{G14} & 1 & \hyperref[G15]{G15} & 0.4957 \\ \hline
		\hyperref[G16]{G16} & 0.0007 & \hyperref[G17]{G17} & 0.0522 & & \\ \hline
	\end{tabular}
	\caption{P-Values of Same Questions Results}
	\label{table:direct_match_answers_pvalues}
\end{table}

\section[Practices' Coverage Results]{Tool's Agile Practices Coverage Results}
\label{sec:coverage_results}
One can see in Table~\ref{table:agile_practices_coverage} how many questions from each tool belong to each agile practice. \ac{TAA} has eight questions which do not belong to any agile practice, since those refer to product ownership. \ac{OPS} covers 18 agile practices, while \ac{TAA} 15 and \ac{PAM} 13. The agile practices which were found relevant to the case studies \cite{Williams_Microsoft, laurie_williams} were marked with the sign \CrossMaltese in the same table.

When it comes to the number of practices covered by the tools, \ac{OPS} comes first, second is \ac{TAA} and \ac{PAM} comes third. On the other hand, when it comes to the number of questions that exist for top agile practices from the case studies \cite{Williams_Microsoft, laurie_williams}, then \ac{TAA} comes first with 52 questions, second \ac{OPS} with 49 questions and \ac{PAM} ranks third with 26 questions. Table~\ref{table:agile_practices_coverage_summary} summarizes the results.

\begin{table}
	\begin{tabular}{| c | c | c | c |} \hline
		\textbf{Practice} & \textbf{TAA} & \textbf{PAM} & \textbf{OPS} \\ \hline
		Iterative and Incremental Development & 2 & 5 & 3 \\ \hline
		Product Backlog \CrossMaltese & 2 & & 3 \\ \hline
		Smaller and Frequent Product Releases \CrossMaltese & 1 & 1 & 4 \\ \hline
		Customer/User Acceptance Testing & & 5 & \\ \hline
		Constant Velocity \CrossMaltese & 1 & & \\ \hline
		Iteration Progress Tracking and Reporting \CrossMaltese & 17 & 5 & 5 \\ \hline
		Self-Organizing Teams \CrossMaltese & 11 & 1 & 7 \\ \hline
		Appropriate Distribution of Expertise & 2 & & 5 \\ \hline
		High-Bandwidth Communication & 4 & 8 & 13 \\ \hline
		Daily Progress Tracking Meetings \CrossMaltese & 1 & 5 & 1 \\ \hline
		Retrospective Meetings \CrossMaltese & 5 & 6 & 4 \\ \hline
		Test Driven Development \CrossMaltese & 3 & 3 & 4 \\ \hline
		Refactoring \CrossMaltese & 2 & 1 & 4 \\ \hline
		Software Configuration Management \CrossMaltese & 1 & & 1 \\ \hline
		Adherence to Standards \CrossMaltese & 3 & & 2 \\ \hline
		Continuous Integration \CrossMaltese & 5 & 5 & 10 \\ \hline
		Client-Driven Iterations & & 2 & 3 \\ \hline
		Minimal or Just Enough Documentation & & & 4 \\ \hline
		Continuous Feedback & & 1 & 2 \\ \hline
		Evolutionary Requirements \CrossMaltese & & & 4 \\ \hline
	 	None & 8 & & \\ \hline
	\end{tabular}
	\caption{Agile Practices Coverage By Tools}
	\label{table:agile_practices_coverage}
\end{table}

\begin{table}
	\begin{tabular}{| c | c |} \hline
		\textbf{Ordered by practices} & \textbf{Ordered by questions} \\ \hline
		OPS (18) & TAA (52) \\ \hline
		TAA (15) & OPS (49) \\ \hline
		PAM (13) & PAM (26) \\ \hline
	\end{tabular}
	\caption{Summary Of Agile Practice's Coverage}
	\label{table:agile_practices_coverage_summary}
\end{table}

%Validity threats?
%http://fluidsurveys.com/university/tips-for-avoiding-respondent-bias/
%http://fluidsurveys.com/university/tips-for-overcoming-researcher-bias/

%http://en.wikipedia.org/wiki/Social_desirability_bias

\section{Chapter Summary}
In this chapter, we presented the results of the surveys answered by Company $A$'s employees. As it was seen, there are very few and very low correlations among the agile practices. Moreover, many of them did not have a monotonic relationship. This fact indicates that convergent validity cannot be established. In addition, questions which we identified to be the same from the three tools did not always have the same answer from the respondents. Finally, we showed that \ac{OPS} covers more agile practices, while \ac{TAA} has more questions for the agile practices it covers, compared with the rest of the tools.
